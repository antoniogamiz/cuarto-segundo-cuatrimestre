\documentclass[12pt]{article}
 
\usepackage[margin=1in]{geometry} 
\usepackage{amsmath,amsthm,amssymb}
\usepackage[spanish]{babel}
\usepackage[utf8]{inputenc}
\usepackage{tikz-cd}
\usepackage{amsmath}
\usepackage[shortlabels]{enumitem}
\usepackage{config}

\newtheorem{theorem}{Teorema}
\newtheorem{lemma}[theorem]{Lema}
\newtheorem{prop}[theorem]{Proposición}
\newtheorem{coro}[theorem]{Corolario}
\newtheorem{conj}[theorem]{Conjetura}
\newtheorem{ejercicio}[theorem]{Ejercicio}
\theoremstyle{definition}
\newtheorem{definition}[theorem]{Definición}
\newtheorem{example}[theorem]{Ejemplo}
\theoremstyle{remark}
\newtheorem{remark}[theorem]{Nota}
\newtheorem{notacion}[theorem]{Notación}

\begin{document}

\title{Curvas y Superficies}
\author{Antonio Gámiz Delgado\\ Universidad de Granada} 
 
\maketitle

\begin{ejercicio}

\end{ejercicio}
Para cada función diferenciable $\funcion{f}{\R}{\R}$ se considera la curva $\funcion{\alpha_f}{\R}{\R^3}$ dada por $\alpha_f(t)=(\cos t,\sin t, f(t))$.
\begin{enumerate}[(a)]
\item Comprueba que $\alpha_f$ es regular.
Tenemos que comprobar $\valorabsoluto{\alpha'_f(t)}\neq 0 \; \forall t\in\R$:
\[
\valorabsoluto{\alpha'_f(t)}=\sqrt{(-\sin t)^2+(\cos t)^2+(f'(t))^2}=\sqrt{1+(f'(t))^2}
\]
Que nunca puede dar 0 ya que es la suma de dos números positivos.
\item Prueba que para cada $t_1,t_2\in\R$, con $t_1<t_2$, se cumple $\integral{t_1}{t_2}{\valorabsoluto{\alpha'_{f}(t)}dt}\geq t_2-t_1$.

\[
\integral{t_1}{t_2}{\valorabsoluto{\alpha'_{f}(t)}dt}=\integral{t_1}{t_2}{\sqrt{1+(f'(t))^2}dt}\geq\integral{t_1}{t_2}{dt}=t_2-t_1
\]

\item Construye el triedro de Frenet.

Como $\valorabsoluto{\alpha_f(t)}\neq 1$, tenemos que el triedro viene dado por:
\[
e_1(t)=\frac{\alpha'_f(t)}{\valorabsoluto{\alpha'_f(t)}}=\frac{1}{\sqrt{1+(f'(t))^2}}\alpha'_f(t)
\]
\[
\alpha''_f(t)=(-\cos t,-\sin t,f''(t)) \Rightarrow \tilde{e_2(t)}= \alpha''_f(t)-<\alpha''_f(t),e_1(t)>e_1(t)=
\]
\[
=\alpha''_f(t)-\frac{f'(t)f''(t)}{1+(f'(t))^2}\alpha'_f(t)=(-\cos t,-\sin t,f''(t))-\frac{f'(t)f''(t)}{1+(f'(t))^2}(-\sin t,\cos t,f'(t))
\]
Calculemos $\valorabsoluto{\tilde{e_2}(t)}^2$, saltándonos las cuentas debido a su gran longitud:
\[
\valorabsoluto{\tilde{e_2}(t)}^2 = \frac{1+f'(t)^2+f''(t)^2}{1+f'(t)^2}\Rightarrow e_2(t)=\sqrt{\frac{1+f'(t)^2+f''(t)^2}{1+f'(t)^2}}\tilde{e_2}(t)
\]
\[
e_3(t)=e_1(t)\times e_2(t)=(...\text{ cuentas }...) = 
\]
\[
=\frac{1}{\sqrt{1+f'(t)^2+f''(t)^2}}(f''(t)\cos t+f'(t)\sin t, f''(t)\sin t-f'(t)\cos t, 1)
\]

\item Calcula sus funciones curvatura y torsión.

Usando el ejercicio 3:
\[
K(t)=\frac{\valorabsoluto{\alpha'(t)\times\alpha''(t)}}{\valorabsoluto{\alpha'(t)}^3} \espacio \tau(t)=\frac{det(\alpha'(t),\alpha''(t),\alpha'''(t))}{|\alpha'(t)\times\alpha''(t)|^2}
\]
\[
\alpha'(t)\times\alpha''(t)=(\cos tf''(t)+\sin tf'(t), \sin tf''(t)-\cos tf'(t),1) \Rightarrow
\]
\[
\valorabsoluto{\alpha'(t)\times\alpha''(t)}^2=1+f'(t)^2+f''(t)^2 \Rightarrow
\]
\[
K(t)=\frac{\sqrt{1+f'(t)^2+f''(t)^2}}{(1+f'(t)^2)^{3/2}}
\]
\[
\alpha'''(t)=(\sin t, -\cos t,f'''(t)) \Rightarrow 
\]
\[
det(\alpha'(t),\alpha''(t),\alpha'''(t))=
\left|
\begin{array}{ccc}
-\sin t & \cos t & f'(t) \\
-\cos t & -\sin t & f''(t) \\
\sin t & -\cos t & f'''(t)
\end{array}
\right|=f'''(t)+f'(t)
\]

\item Determina las funciones $f$ para las cuales $\alpha_f$ es una curva plana.

Para que sea una curva plana, su torsión tiene que ser 0, es decir, $\tau(t)=0$. 
La expresión solo será 0 si el numerador lo es, es decir, $f'''(t)+f'(t)=0$. Cuya solución son las funciones $f(x)=c_1\cos x+c_2\sin x$, con $c_1,c_2\in\R$.

\end{enumerate}
\end{document}