\documentclass[12pt]{report}
 
%%%%%%%%%%%%%%%%%%%%%%%%%%%%%%%%%%%%%%%%%%%%%%%%%%%%%%%%%%%%%%%%%%%%%%%%%%%%%%%%%%%%\usepackage[margin=1in]{geometry} 
\usepackage{amsmath,amsthm,amssymb}
\usepackage[utf8]{inputenc}
\usepackage{amsmath}
\usepackage[shortlabels]{enumitem}
\usepackage{mathtools}
\usepackage{personalcommands}
\usepackage{amsfonts}
\usepackage{float}
\usepackage{epigraph}
\usepackage{lipsum}
\usepackage{parskip}
\usepackage[spanish]{babel}
\usepackage{tikz}
\usetikzlibrary{babel}
\usepackage{csquotes}
\usepackage{xcolor}
\usepackage[framemethod=tikz,xcolor=true]{mdframed}
\usepackage[new]{old-arrows}
%%%%%%%%%%%%%%%%%%%%%%%%%%%%%%%%%%%%%%%%%%%%%%%%%%%%%%%%%%%%%%%%%%%%%%%%%%%%%%%%%%
\begin{document}

\textbf{Ejercicio 3.}  Se dispone de una relación $R(a,b)$ donde $a$ es la clave de valores únicos por la que se mantiene ordenado el archivo y $b$ es un atributo con valores duplicados. Además se tiene $B=4096B$, $C=10B$, $P=10B$, $N(R)=1000$, $V(R,b)=200$, $size(a)=10B$, $size(b)=40B$. Se montan dos índices, $I_A$, $I_B$, uno por cada atributo. Calcula el tamaño en bloques de cada índice.

\begin{enumerate}[(a)]
\item Índice $I_A$: cada entrada del índice ocupará $size(a)+P=10B+10B=20B$. En cada bloque caben las siguientes entradas:
\[
Bfr(I_A)=\left\lfloor\frac{B-C}{20B}\right\rfloor=\left\lfloor\frac{4096B-10B}{20B}\right\rfloor=204
\]
Como $N(R)=1000$ y necesitamos una entrada en el índice por cada registro, necesitamos:
\[
nºbloques=\left\lceil\frac{N(R)}{Bfr(I_A)}\right\rceil=\left\lceil\frac{1000}{204}\right\rceil=5
\]
\item Índice $I_B$: como los registros están duplicados, solo necesitamos añadir una entrada en el índice por cada valor distinto, es decir, necesitamos $V(R,b)=200$ entradas.
\[
Bfr(I_B)=\left\lfloor\frac{B-C}{40B}\right\rfloor=\left\lfloor\frac{4096B-10B}{40B}\right\rfloor=102
\]
\[
nºbloques=\left\lceil\frac{V(R,b)}{Bfr(I_B)}\right\rceil=\left\lceil\frac{200}{102}\right\rceil=2
\]
\end{enumerate}

\end{document}

