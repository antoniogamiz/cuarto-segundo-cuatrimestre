\documentclass[12pt]{report}
 
%%%%%%%%%%%%%%%%%%%%%%%%%%%%%%%%%%%%%%%%%%%%%%%%%%%%%%%%%%%%%%%%%%%%%%%%%%%%%%%%%%%%\usepackage[margin=1in]{geometry} 
\usepackage{amsmath,amsthm,amssymb}
\usepackage[utf8]{inputenc}
\usepackage{amsmath}
\usepackage[shortlabels]{enumitem}
\usepackage{mathtools}
\usepackage{personalcommands}
\usepackage{amsfonts}
\usepackage{float}
\usepackage{epigraph}
\usepackage{lipsum}
\usepackage{parskip}
\usepackage[spanish]{babel}
\usepackage{tikz}
\usetikzlibrary{babel}
\usepackage{csquotes}
\usepackage{xcolor}
\usepackage[framemethod=tikz,xcolor=true]{mdframed}
\usepackage[new]{old-arrows}
%%%%%%%%%%%%%%%%%%%%%%%%%%%%%%%%%%%%%%%%%%%%%%%%%%%%%%%%%%%%%%%%%%%%%%%%%%%%%%%%%%
\begin{document}

\textbf{Ejercicio 5.} Indica si son ciertas o falsas las siguientes afirmaciones y, brevemente, explica por qué:
\begin{enumerate}[(a)]
\item El tiempo que se tarda en reorganizar un fichero ASI (Arcihvos Secuenciales Indexados) denso depende únicamente del tiempo necesario para reordenar el fichero de desbordamientoo y de reescribir ordenadamente el nuevo fichero maestro.

Falso. Eso sería si fuera un ASL, pero como indica la I de ASI, aquí también juegan los índices. Al reordenar el archivo, es necesario reconstruir el índice, luego también hay que tener en cuenta ese tiempo.
\item Para consultas por rango, los ASI (Archivos Secuenciales Indexados) son  más adecuados.

Falso también, para que quieres un índice si vas a consultar por rango. Para eso, te ahorras el índice y usas un ASL, que simplemente tienes que encontrar el del principio del rango, el del final, y leer todo lo del medio.
\end{enumerate} 
\end{document}

