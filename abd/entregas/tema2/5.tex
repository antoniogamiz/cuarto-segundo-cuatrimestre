\documentclass[12pt]{report}
 
%%%%%%%%%%%%%%%%%%%%%%%%%%%%%%%%%%%%%%%%%%%%%%%%%%%%%%%%%%%%%%%%%%%%%%%%%%%%%%%%%%%%\usepackage[margin=1in]{geometry} 
\usepackage{amsmath,amsthm,amssymb}
\usepackage[utf8]{inputenc}
\usepackage{amsmath}
\usepackage[shortlabels]{enumitem}
\usepackage{mathtools}
\usepackage{personalcommands}
\usepackage{amsfonts}
\usepackage{float}
\usepackage{epigraph}
\usepackage{lipsum}
\usepackage{parskip}
\usepackage[spanish]{babel}
\usepackage{tikz}
\usetikzlibrary{babel}
\usepackage{csquotes}
\usepackage{xcolor}
\usepackage[framemethod=tikz,xcolor=true]{mdframed}
\usepackage[new]{old-arrows}
%%%%%%%%%%%%%%%%%%%%%%%%%%%%%%%%%%%%%%%%%%%%%%%%%%%%%%%%%%%%%%%%%%%%%%%%%%%%%%%%%%
\begin{document}

\textbf{Ejercicio 5.} Indica si son ciertas o falsas las siguientes afirmaciones y, brevemente, explica por qué:
\begin{enumerate}[(a)]
\item El tiempo que se tarda en reorganizar un fichero ASI (Arcihvos Secuenciales Indexados) denso depende únicamente del tiempo necesario para reordenar el fichero de desbordamiento y de reescribir ordenadamente el nuevo fichero maestro.

Falso, a parte de depender de esos dos factores, debido a la existencia de un índice, también depende del tiempo que se tarda en generar el índice

\item Para consultas por rango, los ASI (Archivos Secuenciales Indexados) son más adecuados.

Verdadero, pero esto sólo ocurre cuando se consulta por el mismo atributo del índice. Si no, es altamente ineficiente.
\end{enumerate} 
\end{document}

