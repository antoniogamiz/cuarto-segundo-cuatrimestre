\documentclass[12pt]{report}
 
%%%%%%%%%%%%%%%%%%%%%%%%%%%%%%%%%%%%%%%%%%%%%%%%%%%%%%%%%%%%%%%%%%%%%%%%%%%%%%%%%%%%\usepackage[margin=1in]{geometry} 
\usepackage{amsmath,amsthm,amssymb}
\usepackage[utf8]{inputenc}
\usepackage{amsmath}
\usepackage[shortlabels]{enumitem}
\usepackage{mathtools}
\usepackage{personalcommands}
\usepackage{amsfonts}
\usepackage{float}
\usepackage{epigraph}
\usepackage{lipsum}
\usepackage{parskip}
\usepackage[spanish]{babel}
\usepackage{tikz}
\usetikzlibrary{babel}
\usepackage{csquotes}
\usepackage{xcolor}
\usepackage[framemethod=tikz,xcolor=true]{mdframed}
\usepackage[new]{old-arrows}
%%%%%%%%%%%%%%%%%%%%%%%%%%%%%%%%%%%%%%%%%%%%%%%%%%%%%%%%%%%%%%%%%%%%%%%%%%%%%%%%%%
\begin{document}

\textbf{Ejercicio 4.} Propón dos planes lógicos para la siguiente consulta e indica de qué dependerá que se escoja uno u otro de sus planes físicos asociados:
\[
\Pi_{C}\left(\sigma_{A=a\wedge B=b}(R)\right)
\]
Dos planes lógicos posibles podrían ser

\begin{tikzpicture}[scale=0.4]
\hspace{5cm}
\draw (0,0) node[anchor=north] {$\Pi_{C}$};
\draw (0,-1.3)-- (0,-3);
\draw (0,-3) node[anchor=north] {$\sigma_{A=a}$};
\draw (0,-4.3)-- (0,-6);
\draw (0,-6) node[anchor=north] {$\sigma_{B=b}$};
\draw (0,-7.3)-- (0,-9);
\draw (0,-9) node[anchor=north] {$R$};

\hspace{3cm}

\draw (0,0) node[anchor=north] {$\Pi_{C}$};
\draw (0,-1.3)-- (0,-3);
\draw (0,-3) node[anchor=north] {$\sigma_{B=b}$};
\draw (0,-4.3)-- (0,-6);
\draw (0,-6) node[anchor=north] {$\sigma_{A=a}$};
\draw (0,-7.3)-- (0,-9);
\draw (0,-9) node[anchor=north] {$R$};
\end{tikzpicture}

La relación $R$ es la misma para ambas consultas (es decir, mismo número de tuplas, tamaño, etc). La diferencia estará en la primera selección y el criterio de elección dependerá del tamaño de la relación intermedia resultando de la primera selección en cada plan.

Luego podemos saber cuál se escogerá comparando el número de registros resultantes:
\[
N_1=\frac{N(R)}{V(R,A)}, \espacio N_2=\frac{N(R)}{V(R,B)}
\]
Se escogerá aquel plan que minimize el número de tuplas en la primera selección, como $V(R,A)$ y $V(R,B)$ se encuentran en el denominador, se escogerá aquel que sea mayor.

\end{document}

