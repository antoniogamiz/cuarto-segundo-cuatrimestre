\documentclass[12pt]{report}
 
%%%%%%%%%%%%%%%%%%%%%%%%%%%%%%%%%%%%%%%%%%%%%%%%%%%%%%%%%%%%%%%%%%%%%%%%%%%%%%%%%%%%\usepackage[margin=1in]{geometry} 
\usepackage{amsmath,amsthm,amssymb}
\usepackage[utf8]{inputenc}
\usepackage{amsmath}
\usepackage[shortlabels]{enumitem}
\usepackage{mathtools}
\usepackage{personalcommands}
\usepackage{amsfonts}
\usepackage{float}
\usepackage{epigraph}
\usepackage{lipsum}
\usepackage{parskip}
\usepackage[spanish]{babel}
\usepackage{tikz}
\usetikzlibrary{babel}
\usepackage{csquotes}
\usepackage{xcolor}
\usepackage[framemethod=tikz,xcolor=true]{mdframed}
\usepackage[new]{old-arrows}
%%%%%%%%%%%%%%%%%%%%%%%%%%%%%%%%%%%%%%%%%%%%%%%%%%%%%%%%%%%%%%%%%%%%%%%%%%%%%%%%%%
\begin{document}

\textbf{Ejercicio 1.} Sean las relaciones $R$ y $S$ con los siguientes parámetros:

\begin{center}
\begin{tabular}{|c|c|}
\hline 
R(a,b,c) & S(c,d) \\ 
\hline 
N(R)=5000 & N(S)=200 \\ 
\hline 
V(R,a)=5000 &   \\ 
\hline 
V(R,b)=3000 &   \\ 
\hline 
V(R,c)=5 & V(S,c)=5 \\ 
\hline 
  & V(S,d)=40 \\ 
\hline 
Size(a)=20 &   \\ 
\hline 
Size(b)=60 &   \\ 
\hline 
Size(c)=20 & Size(c)=20 \\ 
\hline 
  & Size(d)=40 \\ 
\hline 
\end{tabular} 
\end{center} 

Teniendo en cuenta que el tamaño de bloque es $B=2KB$, que la cabecera es de $20B$ y que en memoria sólo cabe un bloque, determina el número de operaciones de E/S que supondría la ejecución de la consulta:
\[
\Pi_{a,d}(R\;JOIN\;S)
\]
Primero la longitud de los registros:
\[
L(R)=20B+60B+20B=100B, \espacio L(S)=20B+40B=60B
\]
y el factor de bloqueo para cada tabla:
\[
Bfr(R)=\left\lfloor\frac{T(B)-C}{L(R)}\right\rfloor=\left\lfloor\frac{2048B-20B}{100B}\right\rfloor=20
\]
\[
Bfr(S)=\left\lfloor\frac{T(B)-C}{L(S)}\right\rfloor=\left\lfloor\frac{2048B-20B}{60B}\right\rfloor=33
\]
y el número de bloques de cada tabla:
\[
B(R)=\left\lceil\frac{N(R)}{Bfr(R)}\right\rceil=\left\lceil\frac{5000}{20}\right\rceil= 250
\]
\[
B(S)=\left\lceil\frac{N(S)}{Bfr(S)}\right\rceil=\left\lceil\frac{200}{33}\right\rceil= 7
\]
Para que la unión sea eficiente, $R$ y $S$ deberían estar por el atributo por el que se va a hacer la unión. Eso requiere $n\log_2(n)$ bloques en cada caso:
\[
250\log_2(250)+7\log_2(7)\approx 2011
\]

Ahora, para hacer la unión, necesitamos leer las dos tablas enteras, luego necesitaremos $250+7$ bloques de E/S. Hacemos la unión por $c$. Calculamos el número de registros:
\[
N(R\;JOIN\;S)=\frac{N(R)N(S)}{\max\{V(R,c),V(S,c)\}}=\frac{5000\cdot 200}{\max\{5,5)\}}=200000
\]
\[
L(R\;JOIN\;S)=L(R)+L(S)-size(c)=100B+60B-20B=140B
\]
\[
Bfr(R\;JOIN\;S)=\left\lfloor\frac{T(B)-C}{L(R\;JOIN\;S)}\right\rfloor=\left\lfloor\frac{2048B-20B}{140B}\right\rfloor=14
\]
\[
B(R\;JOIN\;S)=\left\lceil\frac{N(R\;JOIN\;S)}{Bfr(R\;JOIN\;S)}\right\rceil=\left\lceil\frac{200000}{14}\right\rceil= 14286
\]
Luego para escribir en disco el resultado de la unión hemos necesitado 14286 E/S. Ahora necesitamos otros 14286 bloques de E/S para leerlos y hacer la proyección. Por simplicidad, notamos $\Pi_{a,d}(R\;JOIN\;S)$.
\[
L(P)=size(a)+size(d)=20B+40B=60B 
\]
\[
N(P)=N(R\;JOIN\;S)
\]
\[
Bfr(P)=\left\lfloor\frac{T(B)-C}{L(P)}\right\rfloor=\left\lfloor\frac{2048B-20B}{60B}\right\rfloor=33
\]
\[
B(P)=\left\lceil\frac{N(P)}{Bfr(P)}\right\rceil=\left\lceil\frac{200000}{33}\right\rceil= 6061
\]
Luego necesitamos 6061 bloques E/S para escribir el resultado de la proyección. En total:
\[
2011+2(250+7)+2\cdot 14286+6061=37158
\]
\end{document}

