\documentclass[11pt]{report}
 
%%%%%%%%%%%%%%%%%%%%%%%%%%%%%%%%%%%%%%%%%%%%%%%%%%%%%%%%%%%%%%%%%%%%%%%%%%%%%%%%%%%%\usepackage[margin=1in]{geometry} 
\usepackage{amsmath,amsthm,amssymb}
\usepackage[utf8]{inputenc}
\usepackage{amsmath}
\usepackage[shortlabels]{enumitem}
\usepackage{mathtools}
\usepackage{amsfonts}
\usepackage{float}
\usepackage{epigraph}
\usepackage{lipsum}
\usepackage{parskip}
\usepackage[spanish]{babel}
\usepackage{tikz}
\usetikzlibrary{babel}
\usepackage{csquotes}
\usepackage{xcolor}
\usepackage[framemethod=tikz,xcolor=true]{mdframed}
\usepackage[new]{old-arrows}
%%%%%%%%%%%%%%%%%%%%%%%%%%%%%%%%%%%%%%%%%%%%%%%%%%%%%%%%%%%%%%%%%%%%%%%%%%%%%%%%%%
\begin{document}

\hrule

\textbf{Tema en el que se plantea}: Tema 3

\textbf{Transparencia:} 25

\textbf{Respuesta correcta:}  (c)

\textbf{Con respecto a las extensiones, es cierto que: }
\begin{enumerate}[(a)]
\item Todas contienen el mismo número de bloques y del mismo tamaño, pero los bloques no tienen por qué estar contiguos en memoria.
\item Todas contienen el mismo número de bloques y del mismo tamaño, pero los bloques estan contiguos en memoria.
\item  no tienen por qué tener el mismo número de bloques, ni del mismo tamaño, y los bloques no tienen por qué estar contiguos en memoria.
\item Todas son falsas.
\end{enumerate}

\hrule 

\textbf{Tema:} Tema 1

\textbf{Transparencia(s):} 80,81,82 

\textbf{Texto:}  Enuncie tres estrategias diferentes para resolver colisiones en un Archivo de Acceso Directo (AAD). Expliquelas brevemente.

\textbf{Respuesta:} Tenemos varias opciones disponibles:
\begin{itemize}
\item Direccionamiento cerrado: el espacio de posiciones consiste de un único archivo. Dos estrategias:
\begin{itemize}
\item Búsqueda lineal: una colisión se almacena en una posicición libre del mismo bloque.
\item Realeatorización: una colisión se almacena en otra posición reaplicando la transformación de clave.
\end{itemize}
\item Direccionamiento abierto: el espacio de posiciones se compone de más de un fichero (principal y desbordamiento). Dos estrategias:
\begin{itemize}
\item Listas enlazadas: las colisiones se almacenan en un fichero de desbordamiento (ASI).
\item Bloques de desbordamiento: colisiones en bloques del fichero de desbordamiento.
\end{itemize}
\item Hashing dinámico: el espacio de posiciones y la transformación se adaptan dinámicamente.
\end{itemize}

\[
\sigma_{c<c_k}\left(\Pi_c(R\;JOIN\;S)\right)
\]
\[
\Pi_c(\sigma_{c<c_k}(R)JOIN\;S)
\]

\end{document}

