\documentclass[12pt]{report}
 
%%%%%%%%%%%%%%%%%%%%%%%%%%%%%%%%%%%%%%%%%%%%%%%%%%%%%%%%%%%%%%%%%%%%%%%%%%%%%%%%%%%%\usepackage[margin=1in]{geometry} 
\usepackage{amsmath,amsthm,amssymb}
\usepackage[utf8]{inputenc}
\usepackage{amsmath}
\usepackage[shortlabels]{enumitem}
\usepackage{mathtools}
\usepackage{personalcommands}
\usepackage{amsfonts}
\usepackage{float}
\usepackage{epigraph}
\usepackage{lipsum}
\usepackage{parskip}
\usepackage[spanish]{babel}
\usepackage{tikz}
\usetikzlibrary{babel}
\usepackage{csquotes}
\usepackage{xcolor}
\usepackage[framemethod=tikz,xcolor=true]{mdframed}
\usepackage[new]{old-arrows}
%%%%%%%%%%%%%%%%%%%%%%%%%%%%%%%%%%%%%%%%%%%%%%%%%%%%%%%%%%%%%%%%%%%%%%%%%%%%%%%%%%
\begin{document}

Sean las relaciones $R$ y $S$ con los siguientes parámetros:

\begin{center}
\begin{tabular}{|c|c|}
\hline 
R(a,b,c) & S(a,d,e) \\ 
\hline 
N(R)=1000 & N(S)=10000 \\ 
\hline 
Size(a)=20 & Size(a)=20 \\ 
\hline 
Size(b)=40 &   \\ 
\hline 
Size(c)=100 &   \\ 
\hline 
  & Size(d)=20 \\ 
\hline 
  & Size(e)=40 \\ 
\hline 
V(R,a)=1000 & V(S,a)=1000 \\ 
\hline 
V(R,b)=200 &   \\ 
\hline 
V(R,c)=20 &   \\ 
\hline 
  & V(S,d)=??? \\ 
\hline 
  & V(S,e)=40 \\ 
\hline 
\end{tabular} 
\end{center}
donde $a$ es la llave primaria de $R$ y $(a,d)$ es la llave primaria de $S$, y donde el atributo $S.a$ es llave externa a $R.a$.


Teniendo en cuenta que $B=4KB$, $C=40B$, que se usa \textbf{bloqueo fijo}, que los \textbf{bloques} son \textbf{homogéneos}, que en memoria cabe únicamente un bloque de cada relación o resultado de operación intermedia, y considerando que las operaciones de \textbf{proyección y selección 'respetan' índices} (es decir, si las relaciones sobre las que se aplica la operación tienen un índice, el resultado de la misma también lo tendrá).

\textbf{Ejercicio 6.} Considerando la estructura de llaves (primarias y externa) de $R$ y $S$, deduce la variabilidad $V(S,d)$.

Vemos que $N(S)=10000$, luego $(a,d)$, al ser llave primaria, debe tener exactamente 10000 valores distintos. Como $a$ es llave primaria en $R$, clave externa en $S$ y $V(S,a)=1000$, deducimos que todos los valores de $a$ en $R$ también se encuentran en $S$. 

Si ahora suponemos que $V(S,d)=1$, tendríamos solamente $1000\cdot 1 = 1000$ pares $(a,d)$ distintos, luego no podríamos cubrir los 10000 necesarios. Luego $V(S,d)$, por lo menos tiene que ser:
\[
V(S,d)\geq \frac{N(S)}{V(S,a)}=\frac{10000}{1000}=10
\]
¿Lo podemos acotar superiormente? A lo bruto, $d$ puede tener tantos valores distintos como tuplas haya en la relación, luego:
\[
V(S,d)\leq N(S) = 10000
\]
En resumen:
\[
10 \leq V(S,d) \leq 10000
\]

\end{document}

