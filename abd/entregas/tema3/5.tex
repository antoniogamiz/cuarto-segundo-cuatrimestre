\documentclass[12pt]{report}
 
%%%%%%%%%%%%%%%%%%%%%%%%%%%%%%%%%%%%%%%%%%%%%%%%%%%%%%%%%%%%%%%%%%%%%%%%%%%%%%%%%%%%\usepackage[margin=1in]{geometry} 
\usepackage{amsmath,amsthm,amssymb}
\usepackage[utf8]{inputenc}
\usepackage{amsmath}
\usepackage[shortlabels]{enumitem}
\usepackage{mathtools}
\usepackage{personalcommands}
\usepackage{amsfonts}
\usepackage{float}
\usepackage{epigraph}
\usepackage{lipsum}
\usepackage{parskip}
\usepackage[spanish]{babel}
\usepackage{tikz}
\usetikzlibrary{patterns}
\usetikzlibrary{babel}
\usepackage{csquotes}
\usepackage{xcolor}
\usepackage[framemethod=tikz,xcolor=true]{mdframed}
\usepackage[new]{old-arrows}
%%%%%%%%%%%%%%%%%%%%%%%%%%%%%%%%%%%%%%%%%%%%%%%%%%%%%%%%%%%%%%%%%%%%%%%%%%%%%%%%%%
\begin{document}

Se dispone de un archivo secuencial indexado con un factor de bloqueo de 4 registros para almacenar registros de longitud fija, con la siguiente estructura en el fichero maestro (de datos):

\begin{center}
\begin{tikzpicture}[scale=0.4]
\draw[draw=black] (-0.5,0) rectangle (0,3);
\draw[draw=black] (0,0) rectangle (3,3);
\draw[draw=black] (3,0) rectangle (6,3);
\draw[draw=black] (6,0) rectangle (9,3);
\draw[draw=black] (9,0) rectangle (12,3);
\pattern[pattern=north east lines] (-0.5,0)--(-0.5,3)--(0,3)--(0,0)--cycle;
\end{tikzpicture}
\end{center}

\textbf{Ejercicio 5.} Indica qué ocurre cuando se añaden los registros con valores de clave 6,4,8,9 y 1, tanto el fichero maestro como en el fichero de índice.

El fichero maestro es un ASL, luego los registros se encuentran ordenados por una clave física, la que nos dan. 

\begin{center}
\begin{tikzpicture}[scale=0.4]
%
\draw[draw=black] (-0.5,0) rectangle (0,3);
\draw[draw=black] (0,0) rectangle (3,3);
\draw[draw=black] (3,0) rectangle (6,3);
\draw[draw=black] (6,0) rectangle (9,3);
\draw[draw=black] (9,0) rectangle (12,3);
\pattern[pattern=north east lines] (-0.5,0)--(-0.5,3)--(0,3)--(0,0)--cycle;
%
\draw (-4.5,1.5) node[] {$Maestro$};
\draw (1.5,1.5) node[] {$1$};
\draw (4.5,1.5) node[] {$4$};
\draw (7.5,1.5) node[] {$6$};
\draw (10.5,1.5) node[] {$8$};
\draw (1.5,-1) node[] {$0$};
\draw (4.5,-1) node[] {$1$};
\draw (7.5,-1) node[] {$2$};
\draw (10.5,-1) node[] {$3$};
%
\draw[draw=black] (-0.5+15,0) rectangle (0+15,3);
\draw[draw=black] (0+15,0) rectangle (3+15,3);
\draw[draw=black] (3+15,0) rectangle (6+15,3);
\draw[draw=black] (6+15,0) rectangle (9+15,3);
\draw[draw=black] (9+15,0) rectangle (12+15,3);
\pattern[pattern=north east lines] (-0.5+15,0)--(-0.5+15,3)--(0+15,3)--(0+15,0)--cycle;
%
\draw (1.5+15,1.5) node[] {$9$};
\draw (1.5+15,-1) node[] {$4$};
\draw (4.5+15,-1) node[] {$5$};
\draw (7.5+15,-1) node[] {$6$};
\draw (10.5+15,-1) node[] {$7$};
%
\draw[->, thick] (12,1.5) -- (14.5,1.5);
\end{tikzpicture}
\end{center}

\begin{center}
\begin{tikzpicture}[scale=0.4]
%
\draw[draw=black] (-0.5,0) rectangle (0,3);
\draw[draw=black] (0,0) rectangle (3,3);
\draw[draw=black] (3,0) rectangle (6,3);
\draw[draw=black] (6,0) rectangle (9,3);
\draw[draw=black] (9,0) rectangle (12,3);
\pattern[pattern=north east lines] (-0.5,0)--(-0.5,3)--(0,3)--(0,0)--cycle;
%
\draw (-4.5,1.5) node[] {$Indice$};
\draw (1.5,1.5) node[] {$(1,0)$};
\draw (4.5,1.5) node[] {$(4,1)$};
\draw (7.5,1.5) node[] {$(6,2)$};
\draw (10.5,1.5) node[] {$(8,3)$};
%
\draw[draw=black] (-0.5+15,0) rectangle (0+15,3);
\draw[draw=black] (0+15,0) rectangle (3+15,3);
\draw[draw=black] (3+15,0) rectangle (6+15,3);
\draw[draw=black] (6+15,0) rectangle (9+15,3);
\draw[draw=black] (9+15,0) rectangle (12+15,3);
\pattern[pattern=north east lines] (-0.5+15,0)--(-0.5+15,3)--(0+15,3)--(0+15,0)--cycle;
%
\draw (1.5+15,1.5) node[] {$(9,4)$};
%
\draw[->, thick] (12,1.5) -- (14.5,1.5);
\end{tikzpicture}
\end{center}

En este tengo dudas, porque no se si al insertar el último registro.

\end{document}

