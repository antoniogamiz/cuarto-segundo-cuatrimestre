\documentclass[12pt]{report}
 
%%%%%%%%%%%%%%%%%%%%%%%%%%%%%%%%%%%%%%%%%%%%%%%%%%%%%%%%%%%%%%%%%%%%%%%%%%%%%%%%%%%%\usepackage[margin=1in]{geometry} 
\usepackage{amsmath,amsthm,amssymb}
\usepackage[utf8]{inputenc}
\usepackage{amsmath}
\usepackage[shortlabels]{enumitem}
\usepackage{mathtools}
\usepackage{personalcommands}
\usepackage{amsfonts}
\usepackage{float}
\usepackage{epigraph}
\usepackage{lipsum}
\usepackage{parskip}
\usepackage[spanish]{babel}
\usepackage{tikz}
\usetikzlibrary{patterns}
\usetikzlibrary{babel}
\usepackage{csquotes}
\usepackage{xcolor}
\usepackage[framemethod=tikz,xcolor=true]{mdframed}
\usepackage[new]{old-arrows}
%%%%%%%%%%%%%%%%%%%%%%%%%%%%%%%%%%%%%%%%%%%%%%%%%%%%%%%%%%%%%%%%%%%%%%%%%%%%%%%%%%
\begin{document}

Se dispone de un archivo secuencial indexado con un factor de bloqueo de 4 registros para almacenar registros de longitud fija, con la siguiente estructura en el fichero maestro (de datos):

\begin{center}
\begin{tikzpicture}[scale=0.4]
\draw[draw=black] (-0.5,0) rectangle (0,3);
\draw[draw=black] (0,0) rectangle (3,3);
\draw[draw=black] (3,0) rectangle (6,3);
\draw[draw=black] (6,0) rectangle (9,3);
\draw[draw=black] (9,0) rectangle (12,3);
\pattern[pattern=north east lines] (-0.5,0)--(-0.5,3)--(0,3)--(0,0)--cycle;
\end{tikzpicture}
\end{center}

\textbf{Ejercicio 4.} Rellena sobre el bloque del enunciado el resultado de insertar los registros con valores de clave 7,2,5,3.

El fichero maestro es un ASL, luego los registros se encuentran ordenados por una clave física, la que nos dan. 

\begin{center}
\begin{tikzpicture}[scale=0.4]
\draw[draw=black] (-0.5,0) rectangle (0,3);
\draw[draw=black] (0,0) rectangle (3,3);
\draw[draw=black] (3,0) rectangle (6,3);
\draw[draw=black] (6,0) rectangle (9,3);
\draw[draw=black] (9,0) rectangle (12,3);
\pattern[pattern=north east lines] (-0.5,0)--(-0.5,3)--(0,3)--(0,0)--cycle;
\draw (-4.5,1.5) node[] {$Maestro$};
\draw (1.5,1.5) node[] {$2$};
\draw (4.5,1.5) node[] {$3$};
\draw (7.5,1.5) node[] {$5$};
\draw (10.5,1.5) node[] {$7$};
\draw (1.5,-1) node[] {$0$};
\draw (4.5,-1) node[] {$1$};
\draw (7.5,-1) node[] {$2$};
\draw (10.5,-1) node[] {$3$};
\end{tikzpicture}
\end{center}

%\begin{center}
%\begin{tikzpicture}[scale=0.4]
%\draw[draw=black] (-0.5,0) rectangle (0,3);
%\draw[draw=black] (0,0) rectangle (3,3);
%\draw[draw=black] (3,0) rectangle (6,3);
%\draw[draw=black] (6,0) rectangle (9,3);
%\draw[draw=black] (9,0) rectangle (12,3);
%\pattern[pattern=north east lines] (-0.5,0)--(-0.5,3)--(0,3)--(0,0)--cycle;
%\draw (-4.5,1.5) node[] {$Indice$};
%\draw (1.5,1.5) node[] {$(2,1)$};
%\draw (4.5,1.5) node[] {$(3,3)$};
%\draw (7.5,1.5) node[] {$(5,2)$};
%\draw (10.5,1.5) node[] {$(7,0)$};
%\end{tikzpicture}
%\end{center}

\end{document}

