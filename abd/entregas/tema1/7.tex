\documentclass[12pt]{report}
 
%%%%%%%%%%%%%%%%%%%%%%%%%%%%%%%%%%%%%%%%%%%%%%%%%%%%%%%%%%%%%%%%%%%%%%%%%%%%%%%%%%%%\usepackage[margin=1in]{geometry} 
\usepackage{amsmath,amsthm,amssymb}
\usepackage[utf8]{inputenc}
\usepackage{amsmath}
\usepackage[shortlabels]{enumitem}
\usepackage{mathtools}
\usepackage{amsfonts}
\usepackage{float}
\usepackage{epigraph}
\usepackage{lipsum}
\usepackage{parskip}
\usepackage[spanish]{babel}
\usepackage{tikz}
\usetikzlibrary{babel}
\usepackage{csquotes}
\usepackage{xcolor}
\usepackage[framemethod=tikz,xcolor=true]{mdframed}
\usepackage[new]{old-arrows}
%%%%%%%%%%%%%%%%%%%%%%%%%%%%%%%%%%%%%%%%%%%%%%%%%%%%%%%%%%%%%%%%%%%%%%%%%%%%%%%%%%
\begin{document}

\textbf{Ejercicio 7.} Supón una tabla con \textit{nombre\_paciente},  que es un \textit{varchar(55)} que ocupa 56B, una fecha que ocupa 10B, un peso de tipo real que ocupa 8B, un numero\_interveciones que es un entero y ocupa 4B, un numero\_hijos que es un entero y ocupa 4B, un atributo fumador que es lógico y ocupa 1B, y un R de 83B. Calcula el factor de bloqueo y el porcentaje de utilización en caso de tratarse de bloqueo fijo en los casos de:

Del enunciado deducimos que trata de registros de longitud fija con $R=83B$. Como no dice nada de cuanto ocupa la cabecera, voy a suponer que ocupa tanto como un registro (no sé si debería suponer tamaño 0), luego $C=R=83B$.

\begin{enumerate}[(a)]
\item bloque de 2KB
\[
Bfr=\left\lfloor\frac{B-C}{R}\right\rfloor=\left\lfloor\frac{2KB-83B}{83B}\right\rfloor=\left\lfloor\frac{2\cdot 1024B-83B}{83B}\right\rfloor=23
\]
\item bloque de 4KB
\[
Bfr=\left\lfloor\frac{B-C}{R}\right\rfloor=\left\lfloor\frac{4KB-83B}{83B}\right\rfloor=\left\lfloor\frac{4\cdot 1024B-83B}{83B}\right\rfloor=48
\]
\end{enumerate}


\end{document}

