\documentclass[12pt]{report}
 
%%%%%%%%%%%%%%%%%%%%%%%%%%%%%%%%%%%%%%%%%%%%%%%%%%%%%%%%%%%%%%%%%%%%%%%%%%%%%%%%%%%%
\usepackage[margin=1in]{geometry} 
\usepackage{amsmath,amsthm,amssymb}
\usepackage[utf8]{inputenc}
\usepackage{amsmath}
\usepackage[shortlabels]{enumitem}
\usepackage{mathtools}
\usepackage{amsfonts}
\usepackage{float}
\usepackage{epigraph}
\usepackage{lipsum}
\usepackage{parskip}
\usepackage[spanish]{babel}
\usepackage{tikz}
\usetikzlibrary{babel}
\usepackage{csquotes}
\usepackage{xcolor}
\usepackage[framemethod=tikz,xcolor=true]{mdframed}
\usepackage[new]{old-arrows}
\decimalpoint

\newcommand{\floor}[1]{\left\lfloor #1 \right\rfloor}
\newcommand{\ceil}[1]{\left\lceil #1 \right\rceil}
%%%%%%%%%%%%%%%%%%%%%%%%%%%%%%%%%%%%%%%%%%%%%%%%%%%%%%%%%%%%%%%%%%%%%%%%%%%%%%%%%%
\begin{document}

\textbf{Ejercicio 8.} Se tienen registros con un nombre que es un varchar (29), una dirección que es un varchar (255), una fecha que ocupa 10B, un valor para sexo que es un lógico y ocupa 1B, y un tamaño de bloque $B=4B$. Calcula el factor de bloqueo y el porcentaje de utilización en caso de tratarse de bloqueo fijo. Si el bloque contiene 10B de cabecera y un directorio de entradas en el bloque.

Calculamos lo que ocupa un registro:
\[
R=29B+255B+10B+1B=295B
\]
Calculamos el factor de bloqueo:
\[
Bfr=\floor{\frac{B-C}{R}}=\floor{\frac{4096B-10B}{295B}}=13
\]
El porcentaje de utiliación es el complementario del desperdicio, luego:
\[
\% utilizacion = 1-W = 1-\frac{B-13*R-10B}{B}=0.9387 \; (93.87\%)
\]

\end{document}

