\documentclass[12pt]{report}
 
%%%%%%%%%%%%%%%%%%%%%%%%%%%%%%%%%%%%%%%%%%%%%%%%%%%%%%%%%%%%%%%%%%%%%%%%%%%%%%%%%%%%
\usepackage[margin=1in]{geometry} 
\usepackage{amsmath,amsthm,amssymb}
\usepackage[utf8]{inputenc}
\usepackage{amsmath}
\usepackage[shortlabels]{enumitem}
\usepackage{mathtools}
\usepackage{amsfonts}
\usepackage{float}
\usepackage{epigraph}
\usepackage{lipsum}
\usepackage{parskip}
\usepackage[spanish]{babel}
\usepackage{tikz}
\usetikzlibrary{babel}
\usepackage{csquotes}
\usepackage{xcolor}
\usepackage[framemethod=tikz,xcolor=true]{mdframed}
\usepackage[new]{old-arrows}
\decimalpoint

\newcommand{\floor}[1]{\left\lfloor #1 \right\rfloor}
\newcommand{\ceil}[1]{\left\lceil #1 \right\rceil}
%%%%%%%%%%%%%%%%%%%%%%%%%%%%%%%%%%%%%%%%%%%%%%%%%%%%%%%%%%%%%%%%%%%%%%%%%%%%%%%%%%
\begin{document}

\textbf{Ejercicio 6.} Indice por qué mejorran las consultas mediante índices:
\begin{enumerate}[(a)]
\item El número de bloques del índice es menor

Como es menor, es menos costoso consultar el índice que el fichero de datos (menos accesos a disco).

\item Las claves están ordenadas por valor de clave en el índice

Al estar ordenados, podemos hacer consultas por rango y búsquedas de forma muy eficiente, porque en lugar de tener que recorrer todo el fichero de datos simplemente necesitamos hacer esas búsquedas entre los valores de clave del índice.

\item Si son suficientemente pequeños están en memoria

Si caben en memoria, se pueden traer de disco directamente y consultarlo de forma mucho más rápida. Esto implica que si queremos hacer una búsqueda por valor de clave, simplemente consultamos el índice en memoria y después nos traemos explicitamente el bloque que contiene ese registro.
\end{enumerate}
\end{document}

