\documentclass[12pt]{report}
 
%%%%%%%%%%%%%%%%%%%%%%%%%%%%%%%%%%%%%%%%%%%%%%%%%%%%%%%%%%%%%%%%%%%%%%%%%%%%%%%%%%%%\usepackage[margin=1in]{geometry} 
\usepackage{amsmath,amsthm,amssymb}
\usepackage[utf8]{inputenc}
\usepackage{amsmath}
\usepackage[shortlabels]{enumitem}
\usepackage{mathtools}
\usepackage{personalcommands}
\usepackage{amsfonts}
\usepackage{float}
\usepackage{epigraph}
\usepackage{lipsum}
\usepackage{parskip}
\usepackage[spanish]{babel}
\usepackage{tikz}
\usetikzlibrary{babel}
\usepackage{csquotes}
\usepackage{xcolor}
\usepackage[framemethod=tikz,xcolor=true]{mdframed}
\usepackage[new]{old-arrows}
%%%%%%%%%%%%%%%%%%%%%%%%%%%%%%%%%%%%%%%%%%%%%%%%%%%%%%%%%%%%%%%%%%%%%%%%%%%%%%%%%%
\begin{document}

Considera el plan de ejecución de transacciones entrelazadas y la tabla de modificaciones de la derecha:

Lee (T1, A), Lee (T2, A), Escribe (T1, A=20), Lee (T3, B),
Escribe (T2, A=30), Escribe (T3, B=15), Escribe (T3, D=25),
Escribe (T2, E=35)


\textbf{Ejercicio 4.} Si no consideramos el uso de concurrencia (sin abortar transacciones) y los valores iniciales de los datos son $A=10$, $B=0$, $D=8$ y $E=35$, completa la tabla de modificaciones, considerando que se incluye un \textit{start} cuando comienza una transacción y un \textit{commit} cuando termina.

\begin{center}
\begin{tabular}{|c|c|c|c|c|c|}
\hline 
$T_i$ & Estado & Operación & Dato & V antiguo & V nuevo \\ 
\hline 
$T_1$ & start &   &   &   &   \\ 
\hline 
$T_2$ & start &   &   &   &   \\ 
\hline 
$T_1$ &  & update & A & 10 & 20 \\ 
\hline 
$T_1$ & commit &  &  &  &  \\ 
\hline 
$T_3$ & start &  &  &  &  \\ 
\hline 
$T_2$ &  & update & A & 20 & 30 \\ 
\hline 
 & savepoint &  &  &  &  \\ 
\hline 
$T_3$ &  & update & B & 0 & 15 \\ 
\hline 
$T_3$ &  & update & D & 8 & 25 \\ 
\hline 
$T_3$ & commit &  &  &  &  \\ 
\hline 
$T_2$ &  & update  & E & 35 & 35 \\ 
\hline 
$T_2$ & commit &  &  &  &  \\ 
\hline 
\end{tabular} 
\end{center}

\end{document}

