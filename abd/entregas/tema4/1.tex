\documentclass[12pt]{report}
 
%%%%%%%%%%%%%%%%%%%%%%%%%%%%%%%%%%%%%%%%%%%%%%%%%%%%%%%%%%%%%%%%%%%%%%%%%%%%%%%%%%%%\usepackage[margin=1in]{geometry} 
\usepackage{amsmath,amsthm,amssymb}
\usepackage[utf8]{inputenc}
\usepackage{amsmath}
\usepackage[shortlabels]{enumitem}
\usepackage{mathtools}
\usepackage{personalcommands}
\usepackage{amsfonts}
\usepackage{float}
\usepackage{epigraph}
\usepackage{lipsum}
\usepackage{parskip}
\usepackage[spanish]{babel}
\usepackage{tikz}
\usetikzlibrary{babel}
\usepackage{csquotes}
\usepackage{xcolor}
\usepackage[framemethod=tikz,xcolor=true]{mdframed}
\usepackage[new]{old-arrows}
%%%%%%%%%%%%%%%%%%%%%%%%%%%%%%%%%%%%%%%%%%%%%%%%%%%%%%%%%%%%%%%%%%%%%%%%%%%%%%%%%%
\begin{document}

Considera el plan de ejecución de transacciones entrelazadas:

Lee (T3, C), Lee (T1, A), Lee (T2, B), Escribe (T3, B),
Lee (T1, B), Escribe (T1, A), Escribe (T3, C), Escribe (T2, B)

Suponiendo que:
\begin{itemize}
\item el plan de ejecución resulta en una serialización de las transacciones en el orden T3, T2 y T1
(es decir, que se ejecuta primero la transacción 3 completa, después la transacción 2
completa y, por último la transacción 1 completa), 
\item antes de la primera operación de cada transacción se inicia la misma, que justo después de la última operación de cada transacción se realiza el commit de la misma, 
\item los valores iniciales de los datos son $A=100$, $B=50$, $C=150$.
\end{itemize}

\textbf{Ejercicio 1.} Indica el contenido de la tabla de modificaciones.

El plan de ejecución, según las suposiciones hechas, sería:

\begin{center}
\begin{tabular}{|c|}
\hline 
Lee(T3,C) \\ 
\hline 
Escribe(T3,B) \\ 
\hline 
Escribe(T3,C) \\ 
\hline 
Lee(T2,B) \\ 
\hline 
Escribe(T2,B) \\ 
\hline 
Lee(T1,A) \\ 
\hline 
Lee(T1,B) \\ 
\hline 
Escribe(T1,A) \\ 
\hline 
\end{tabular} 
\end{center}

La tabla de modificaciones quedaría:
\begin{center}
\begin{tabular}{|c|c|c|c|c|c|}
\hline 
$T_i$ & Estado & Operación & Dato & V antiguo & V nuevo \\ 
\hline 
3 & activa &   &   &   &   \\ 
\hline 
3 &   & update & B & 50 & 50 \\ 
\hline 
3 &   & update & C & 150 & 150 \\ 
\hline 
3 & commit &   &   &   &   \\ 
\hline 
2 & activa &   &   &   &   \\ 
\hline 
2 &   & update & B & 50 & 50 \\ 
\hline 
2 & commit &   &   &   &   \\ 
\hline 
1 & activa &   &   &   &   \\ 
\hline 
1 &   & update & A & 100 & 100 \\ 
\hline 
1 & commit &  &  &  &  \\ 
\hline 
\end{tabular} 
\end{center}


\end{document}

