\documentclass[12pt]{report}
 
%%%%%%%%%%%%%%%%%%%%%%%%%%%%%%%%%%%%%%%%%%%%%%%%%%%%%%%%%%%%%%%%%%%%%%%%%%%%%%%%%%%%\usepackage[margin=1in]{geometry} 
\usepackage{amsmath,amsthm,amssymb}
\usepackage[utf8]{inputenc}
\usepackage{amsmath}
\usepackage[shortlabels]{enumitem}
\usepackage{mathtools}
\usepackage{personalcommands}
\usepackage{amsfonts}
\usepackage{float}
\usepackage{epigraph}
\usepackage{lipsum}
\usepackage{booktabs}
\usepackage{parskip}

\usepackage[spanish]{babel}
\usepackage{tikz}
\usetikzlibrary{babel}
\usepackage{csquotes}
\usepackage{xcolor}
\usepackage[framemethod=tikz,xcolor=true]{mdframed}
\usepackage[new]{old-arrows}
\usepackage{ctable} % for \specialrule command

%%%%%%%%%%%%%%%%%%%%%%%%%%%%%%%%%%%%%%%%%%%%%%%%%%%%%%%%%%%%%%%%%%%%%%%%%%%%%%%%%%
\begin{document}

\begin{center}
\begin{tabular}{|c|c|c|c|c|c|}
\hline 
$T_i$ & Estado & Operación & Dato & V antiguo & V nuevo \\ 
\hline 
$T_1$ & start &   &   &   &   \\ 
\hline 
$T_2$ & start &   &   &   &   \\ 
\hline 
$T_1$ &  & update & A & 10 & 20 \\ 
\hline 
$T_1$ & commit &  &  &  &  \\ 
\hline 
$T_3$ & start &  &  &  &  \\ 
\hline 
$T_2$ &  & update & A & 20 & 30 \\ 
\hline 
 & savepoint &  &  &  &  \\ 
\hline 
$T_3$ &  & update & B & 0 & 15 \\ 
\hline 
$T_3$ &  & update & D & 8 & 25 \\ 
\specialrule{.2em}{.2em}{.2em}
$T_3$ & commit &  &  &  &  \\ 
\hline
$T_2$ &  & update  & E & 35 & 35 \\ 
\hline 
$T_2$ & commit &  &  &  &  \\ 
\hline 
\end{tabular} 
\end{center}

\textbf{Ejercicio 5.} Si ocurriera un \textbf{fallo donde se muestra la doble línea} de la tabla de modificaciones, ¿qué haría el sistema después de recuperarse con cada una de las transacciones y cuáles serían los valores de los datos después de la recuperación?

Si hay un error del sistema, las tres últimas entradas de la tabla de modificaciones no estarían ahí, así que suponemos que no están.

Primeramente vemos que pasa con $T_1$. $T_1$ empieza y hace commit de sus cambios, luego la transacción se ha completado entera. Como el bloque podría haber estado en memoria y no haberse volcado a disco, puede que el valor nuevo de $A$ no haya sido salvado. Por suerte, antes del fallo hay un \textit{savepoint}, luego podemos asegurar que el sistema ha volcado a disco los bloques actualizados de memoria. 

Con $T_2$ y $T_3$ pasa lo mismo, no ha dado tiempo a hacer \textit{commit} antes de la caída del sistema, luego debemos hacer una operación UNDO, es decir, traer los bloques de disco y sobreescribir los valores que haya por los valores antiguos de la tabla de modificaciones.

Dicho esto, los valores tras la recuperación serían:
\[
A=20, \espacio B=0, \espacio D=8, \espacio E=35, \espacio
\]

\end{document}

