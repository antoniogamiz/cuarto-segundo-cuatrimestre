\documentclass[12pt]{article}
 
\usepackage[margin=1in]{geometry} 
\usepackage{amsmath,amsthm,amssymb}
\usepackage[spanish]{babel}
\usepackage[utf8]{inputenc}
\usepackage{tikz-cd}
\usepackage{amsmath}
\usepackage[shortlabels]{enumitem}
\usepackage{mathtools}

% cosas entre comillas 
\usepackage{csquotes}

\usepackage{tikz}

\decimalpoint
\usepackage{xcolor}

\usepackage{personalcommands}
\newcommand{\flechita}[1]{\overset{\rightarrow}{ #1 }}
\newtheorem{theorem}{Teorema}[section]
\newtheorem{lemma}[theorem]{Lema}
\newtheorem{prop}[theorem]{Proposición}
\newtheorem{coro}[theorem]{Corolario}
\newtheorem{conj}[theorem]{Conjetura}
\newtheorem{ejercicio}{Ejercicio}
\newtheorem*{ejercicio*}{Ejercicio}
\theoremstyle{definition}
\newtheorem{definition}[theorem]{Definición}
\newtheorem{example}[theorem]{Ejemplo}
\theoremstyle{remark}
\newtheorem{remark}[theorem]{Nota}
\newtheorem{notacion}[theorem]{Notación}
\newcommand{\continuas}[1][]{C^{ #1 }[a,b]}
\newcommand{\continuasabierto}[1][]{C^{ #1 }(a,b)}
\newcommand{\soportecompacto}{\mathcal{D}(a,b)}
\newcommand{\xcero}{(a,b)}
\newcommand{\xcerocerrado}{[a,b]}
\newcommand{\fvariaciones}{F(x,y(x),y'(x))}

\begin{document}

\textbf{Antonio Gámiz Delgado}

%=================================================== EJERCICIO 1

\begin{ejercicio}
Sea $\funcion{F}{\sobolevcero[-1,1]{1}}{\R}$ definido por
\[
F(y)=\frac{1}{2}\integral{-1}{1}{y'(x)^2dx}+y(0)
\]
Encontrar el mínimo de $F$ en $\sobolevcero[-1,1]{1}$.
\end{ejercicio}

%===================================================

Sea $\funcion{A}{\sobolevcero[-1,1]{1}\times \sobolevcero[-1,1]{1}}{\R}$, definida por $A(u,u)=\frac{1}{2}\integral{-1}{1}{u'(x)^2dx}$. Evidentemente, $A$ es una forma cuadrática y es coerciva:
\[
A(u,u)\geq \frac{1}{2}\norm{u} \espacio \forall u\in\sobolevcero[-1,1]{1}
\]
Ahora sea $\funcion{R}{\sobolevcero[-1,1]{1}}{\R}$, definida por $R(y)=-y(0)$. $R$ es una aplicación lineal y continua, luego el funcional $F(y)=A(y,y)-R(y)$, es cuadrático. El teorema de Lax-Milgran nos dice que existe su mínimo absoluto y que debe cumplir:
\[
A(\phi,y)-R(\phi)=0 \espacio \forall \phi\in\sobolevcero[-1,1]{1}
\]
Desarrollando la expresión anterior:
\[
\frac{1}{2}\integral{-1}{1}{\phi'(x)y'(x)dx}=R(\phi)=-\phi(0)
\]
Si tomamos $\phi\in \mathcal{D}(-1,0)$, obtenemos que 
\[
\integral{-1}{0}{\phi'(x)y'(x)dx}+\integral{-1}{0}{0\phi(x)dx}=0
\]
Luego $z=y'$ tiene derivada débil 0 en $(-1,0)$, es decir, $z\in\sobolevcero[-1,0]{1}$. Hacemos lo mismo con la parte de la derecha, y obtenemos $z\in\sobolevcero[0,1]{1}$. La función $z\notin\mathcal{C}[-1,1]$, pero se cumple $z\in\mathcal{C}[-1,0]\cap\mathcal{C}[0,1]$ y existen los límites laterales de $z$ en $0$. Volviendo a la fórmula anterior:
\[
\frac{1}{2}\integral{-1}{1}{\phi'(x)z(x)dx}=-\phi(0)
\]
Como $z$ no es continua en $(a,b)$, hay que partir la integral en dos:
\[
\integral{-1}{1}{\phi'(x)z(x)dx}=\integral{-1}{0}{\phi'(x)z(x)dx}+\integral{0}{1}{\phi'(x)z(x)dx}
\]
Desarrollando cada una por separado, usando la regla de la cadena en cada término:
\[
\integral{-1}{0}{z(x)\phi'(x)dx}=z(x)\phi(x)\Big|^0_{-1}-\integral{-1}{0}{z'(x)\phi(x)dx}=z(0^-)\phi(0)-\integral{-1}{0}{z'(x)\phi(x)dx}
\]
\[
\integral{0}{1}{z(x)\phi'(x)dx}=z(x)\phi(x)\Big|_0^{1}-\integral{0}{1}{z'(x)\phi(x)dx}=-z(0^+)\phi(0)-\integral{0}{1}{z'(x)\phi(x)dx}
\]
Sumando ambos resultados y sustituyendo en la expresión del principio:
\[
z(0^+)\phi(0)-z(0^-)\phi(0)=-\phi(0)
\]
Como $\phi$ es arbitraria, podemos tomarla de forma que $\phi(\alpha)=1$, quedando:
\[
z(0^+)-z(0^-)=-1
\]
Ahora con estas condiciones, se puede calcular la expresión de $y$. En los intervalos, $[-1,0),(0,1]$, $y$ es una recta. La recta de la izquierda tiene que pasar por $(-1,0)$ y la de la derecha debe pasar por $(1,0)$. Además, la resta de sus pendientes tiene que ser $-1$. Serán de la forma:
\[
y_1(x)=ax+a, \espacio y_2(x)=bx-b, \espacio a,b\in\R
\]
Como $y$ debe ser continua en $\alpha$, se tiene que cumplir $y_1(0)=y_2(0)$, luego nos queda el sistema:
\[
\begin{array}{l}
a\alpha+a=b\alpha+b\\
b-a=-1
\end{array}
\]
Y resolviendo el sistema, obtenemos: $a=\frac{1}{2}$, $b=\frac{-1}{2}$. Luego la función $y$ buscada es:
\[
y(x)=\left\{
\begin{array}{ll}
\frac{1}{2}(x+1) & x\in[-1,0]\\
-\frac{1}{2}(x-1) & x\in(0,1]
\end{array}
\right.
\]

%=================================================== EJERCICIO 2

\begin{ejercicio}
Demostrar que el problema de contorno
\[
-y''+3xy=2x, \espacio y(0)=y(1)=0
\]
tiene una única solución.
\end{ejercicio}

%===================================================

Sean $K(x)=3x$ y $q(x)=-2x$ y el funcional:
\[
L(y)=\frac{1}{2}\integral{0}{1}{y'(x)^2dx}+\frac{1}{2}\integral{0}{1}{K(x)y(x)^2dx}+\integral{0}{1}{q(x)y(x)dx}
\]
Lo anterior se puede expresar como: $F(x,y,p)=\frac{1}{2}p^2+\frac{1}{2}K(x)y^2+q(x)y$. Derivando respecto de $p$ e $y$, y suponiendo que $y$ es un punto extremal, se llega a:
\[
\integral{0}{1}{y'(x)\phi'(x)dx}+\integral{0}{1}{(K(x)y(x)+q(x))\phi(x)dx}=0, \espacio \forall \phi \in \sobolevcero[0,1]{1}
\]
Llamando $z=y'$, se tiene que la derivada débil de $z$ es: $z'=K(x)y+q(x)$. Luego se tiene la siguiente ecuación diferencial igulando ambas expresiones:
\[
-y''+K(x)y+q(x)=0\Rightarrow -y''+3xy-2x=0
\]
Al igual que se hizo en teoría, ahora solo se tiene que encontrar un producto escalar para ver que se tiene un espacio de Hilbert y usar el teorema de representación de Riesz. La proposición $(3.12)$ dice justamente lo que necesitamos.

%=================================================== EJERCICIO 3

\begin{ejercicio}
Sea
\[
p(x)=\left\{
\begin{array}{lr}
3 & \text{ si } x\in(-3,0)\\
1 & \text{ si } x\in(0,2)
\end{array}
\right.
\]
y consideramos $\funcion{L}{\sobolevcero[-3,2]{1}}{\R}$ definido por:
\[
L(y)=\frac{1}{2}\integral{-3}{2}{p(x)y'(x)^2dx}-\integral{-3}{2}{y(x)dx}
\]
\begin{enumerate}
\item Demuestra la existencia de mínimo.
\item Encuentra la expresión en casi todo punto de la función minimizante. Indicación: $z(x)=p(x)y(x)$ admite una extensión continua. 
\end{enumerate}
\end{ejercicio}

%===================================================

Al igual que se hizo en el ejercicio 1, se toma $\funcion{A}{\sobolevcero[-3,2]{1}\times \sobolevcero[-3,2]{1}}{\R}$ definida por $A(u,v)=\frac{1}{2}\integral{-3}{2}{p(x)u'(x)v'(x)dx}$. Al ser $p(x)>0$ y estar acotada superiormente, $A$ es evidentemente coercivo. Sea $R(y)=\integral{-3}{2}{y(x)dx}$, es un funcional lineal continuo (por las propiedades de la integral). En resumen, $L$ es un funcional cuadrático coercivo cuyo dominio es un espacio de Hilbert, luego el teorema de Lax-Milgram asegura la existe de un único mínimo.

Para ver su expresión, se usa la propiedad que proporciona el teorema de representación de Riesz. Sea $y\in\sobolevcero[-3,2]{1}$ dicha solución:
\[
A(\phi,y)-R(\phi)=0 \espacio \forall \phi\in\sobolevcero[-3,0]{1}
\]
Desarrollando esa expresión:
\[
\integral{-3}{2}{p(x)y'(x)\phi'(x)dx}+\integral{-3}{2}{-\phi(x)dx}=0
\]
Luego $\widetilde{z}=py'$ tiene derivada débil -1 en $(-3,2)$. $\widetilde{z}$ admite una extensión continua en $(-3,2)$, $z$. Imponiendo ahora las condiciones que se tienen:
\[
z'(x)=-1\Rightarrow z(x)=y'(x)=-x+a
\]
Al integrar esa ecuación en $(-3,2)$ y $(0,2)$:
\[
\begin{array}{l}
y_1(x)=-\frac{x^2}{6}+\frac{k}{3}x+b\\
y_2(x)=-\frac{x^2}{2}+ax+c
\end{array}
\]
Para resolver ese sistema, se usan las condiciones de contorno:
\[
\left\{
\begin{array}{l}
y_1(-3)=0\\
y_2(2)=0\\
y_1(0^-)=y_2(0^+)
\end{array}
\right.
\]
Resolviendo el sistema resultante, se obtiene: $a=\frac{1}{6}$ y $b=c=\frac{5}{3}$. La expresión del mínimo es, por tanto:
\[
y(x)=\left\{
\begin{array}{ll}
-\frac{x^2}{6}+\frac{x}{18}+\frac{5}{3} & x\in(-3,0)\\
-\frac{x^2}{2}+\frac{x}{6}+\frac{5}{3} & x\in(0,2)
\end{array}
\right.
\]


%================================================= EJERCICIO 4

\begin{ejercicio}
Dado
\[
E=\{u\in\sobolev[0,1]{1}: \; u(0)=0, \; u(1)=1\}, \espacio L(u)=\integral{0}{1}{u'(x)^2dx}+\integral{0}{1}{u(x)dx}
\]
Encontrar $\min_{u\in E}L(u)$.
\end{ejercicio}

%===================================================
\begin{itemize}
\item $E$ es un espacio afín embebido en $\sobolevcero[0,1]{1}$.

Sea $\funcion{\overset{\rightarrow}{\cdot}}{E\times E}{\sobolevcero[0,1]{1}}$, definida por $\flechita{uv}=v-u$ y cumple:
\begin{itemize}
\item Si $p,q,r\in E$, entonces $\flechita{pq}+\flechita{qr}=\flechita{pr}$. Esta condición se cumple trivialmente.
\item Dado $p\in E$ y $v\in\sobolevcero[0,1]{1}$, existe un único $p_v$ tal que $\flechita{pp_v}=v$. La unicidad se obtiene por definición ($p_v=v+p$).
\end{itemize}
\end{itemize}

Sea $p_v=v+x$ y $x\in E$ fijo. Como $E$ está embebido en $\sobolevcero[0,1]{1}$, buscar el mínimo deseado es equivalente a encontrar el mínimo de $L(v+x)$ en $\sobolevcero[0,1]{1}$, donde:
\[
\begin{array}{l}
L(p_v)=L(v+x)=\integral{0}{1}{(v'(x)+1)^2dx}+\integral{0}{1}{(v(x)+x)dx}=
\integral{0}{1}{v'(x)^2dx}+2\integral{0}{1}{v'(x)dx}+1+\\
\integral{0}{1}{v(x)dx}+\integral{0}{1}{xdx}=\integral{0}{1}{(v'(x)+1)^2dx}+\integral{0}{1}{(v(x)+x)dx}=
\integral{0}{1}{(v'(x))^2dx}+\integral{0}{1}{v(x)dx}+\frac{3}{2}
\end{array}
\]
Minimizar el anterior funcional es equivalente a minimizar:
\[
L(v)=\frac{1}{2}A(v,v)-R(v), \;\;
A(u,v)=2\integral{0}{1}{u'(x)v'(x)dx}, \;\; R(\phi)=-\integral{0}{1}{\phi(x)dx}
\]
con $A$ una forma lineal cuadrática y $R$ un funcional lineal. Además, $A$ es evidentemente coercivo, luego también $L$ también lo es.

Ahora ya se tienen las condiciones necearias para aplciar el teorema de Lax-Milgram, el cual asegura la existencia de un único $y\in\sobolevcero[0,1]{1}$, cumpliendo:
\[
A(y,\phi)=R(\phi) \Rightarrow \integral{0}{1}{2y'(x)\phi'(x)dx}+\integral{0}{1}{\phi(x)dx}=0
\]
Luego la derivada débil de $2y'$ es 1, luego:
\[
y''=\frac{1}{2} \Rightarrow y'=\frac{x}{2}+a\Rightarrow y=\frac{x^2}{4}+ax+b
\]
Aplicando las condiciones de contornos se obtienen los valores de $a$ y $b$: $a=-\frac{1}{4}$ y $b=0$. Todavía queda sumarle el término $x$ del principio a $y$, luego la solución final del problema es:
\[
u(x)=\frac{x^2}{4}-\frac{x}{4}+x=\frac{x^2+3x}{4}
\]

\begin{ejercicio}
Sea $E$ un espacio afín embebido en $H$, espacio de Hilbert. Supongamos que $L$ es un funcional cuadrático coercivo y que $E$ es cerrado. Demostrar que el problema 
\[
\min_{u\in E}L(u)
\]
tiene una única solución.
\end{ejercicio}

Como $E$ es un espacio afín embebido en $H$, debe existir una aplicacion $\funcion{\flechita{\cdot}}{E\times E}{H}$ definida por $\flechita{(p,q)}=\flechita{pq}=q-p$. Sea $p\in E$ fijo y $p_v=v+p$. Usando el último ejercicio sobre espacios embebidos se tiene que:
\[
\min_{u\in E}L(u)=\min_{v\in H}L(p_v)=\min_{v\in H}L(p+v)
\]
Como $\funcion{L}{E}{\R}$ es un coercivo, se expresará como:
\[
L(u)=\frac{1}{2}A(u,u)-R(u)
\]
donde $\funcion{A}{E\times E}{\R}$ es una forma cuadrática coerciva y $\funcion{R}{E}{\R}$ es un funcional lineal. Ahora, si se define:
\[
\overline{L}(p_v)=L(v+p)=\frac{1}{2}A(p+v,p+v)-R(p+v)=
\]
\[
=\frac{1}{2}\left(\underbrace{A(p,p)}_{constante}+2A(p,v)+A(v,v)\right)-\underbrace{R(p)}_{constante}-R(v)
\]
Al estar $p$ fijo, esos dos términos son constantes, luego minimizar $\overline{L}$, equivale a minimizar $\widehat{L}$ dado por:
\[
\widehat{L}(v)=\frac{1}{2}A(v,v)-(R(v)-A(v,v))
\]
El último término es un funcional lineal continuo y $A$ sigue siendo una forma cuadrática coerciva, luego $\widehat{L}$ es coercivo también. La clave ahora es que el dominio de $\widehat{L}$ es $H$, un espacio de Hilbert, es decir, se puede aplicar el teorema de Lax-Milgram que asegura la existencia de un único mínimo. Como todos los problemas de mínimo hasta aquí eran equivalentes, el del principio también tiene una única solución. 

%================================================= EJERCICIO 6

\begin{ejercicio}
Encontrar $f_n\in\mathcal{D}(\R)$ tales que:
\[
f_n'(-1)=1, \; f_n'(1)=1, \; \frac{1}{2}\integral{-1}{1}{f_n^2(x)dx}+\frac{1}{2}\integral{-1}{1}{f_n'(x)^2dx}\longrightarrow 0
\]
\end{ejercicio}

Usando la indicación del ejercicio, sea $f_n(x)=\frac{x^{2n+1}}{2n+1}$. Evidentemente $f_n\in\mathcal{D}(\R)$ y se tiene $f_n'(x)=x^{2n}$ con $f_n'(-1)=f_n'(1)=1$. Para la última condición:
\[
\frac{1}{2}\integral{-1}{1}{\left(\frac{1}{2n+1}x^{2n+1}\right)^2dx}+\frac{1}{2}\integral{-1}{1}{x^{4n}dx}=\frac{x^{4n+3}}{2(2n+1)^2(4n+3)}\Big|^1_{-1}+\frac{x^{4n+1}}{2(4n+1)}\Big|^1_{-1}=
\]
\[
= \frac{1}{(2n+1)^2(4n+3)}+\frac{1}{4n+1}\overset{n\to +\infty}{\longrightarrow}0
\]

Ya solo quedaría multiplifar la función $f_n$ por una función meseta para que $\widetilde{f_n}=f_n\phi_\varepsilon\in\mathcal{D}(\R)$.

%================================================= EJERCICIO 7

\begin{ejercicio}
Demostrar que el problema de contorno
\[
-u''+u=0, \; u'(-1)=1, \; u'(1)=1
\]
tiene una única solución.
\end{ejercicio}

Este apartado puede resolverse fácilmente unando teoría de ecuaciones diferenciales. Se tiene que $u_1(x)=e^x$ y $u_2(x)=e^{-x}$ es una base del espacio de soluciones de la ecuación homogénea: $-u''+u=0$. Luego la solución que se tiene que buscar será de la forma $u(x)=au_1(x)+bu_2(x)$, cumpliendo:
\[
\begin{array}{l}
u'(-1)=ae^{-1}-be=1\\
u'(1)=ae-be^{-1}=1
\end{array}
\]
Ese sistema tiene solución ya que el determinante de la matriz de coeficientes es distinto de 0. La solución es única por ser un elemento de la base.

%================================================= EJERCICIO 8

\begin{ejercicio}
Estudia y en su caso calcula:
\[
\min\{L(u):u\in E\}, \espacio L(u)=\frac{1}{2}\integral{-1}{1}{u(x)^2dx}+\frac{1}{2}\integral{-1}{1}{u'(x)^2dx}
\]
con $E=\{u\in\sobolev[-1,1]{2}: u'(-1)=u'(1)=1\}$
\end{ejercicio}

Sea $f_n$ una función cumpliendo las condiciones del ejercico 6. Es trivial comprobar que $f_n\in E$ $\forall n\in\N$ y que $\{L(f_n)\}\longrightarrow 0$ (es de hecho lo que enuncia el ejercicio 6). Como $L(u)\geq 0$ para todo $u\in\sobolev[-1,1]{2}$, se tiene que $\inf\{L(u):u\in E\}=0$. 

¿Es 0 el mínimo también, en caso de existir? Suponiendo que existe $f\in\sobolev[-1,1]{2}$ tal que:
\[
L(f)=\frac{1}{2}\integral{-1}{1}{f(x)^2dx}+\frac{1}{2}\integral{-1}{1}{f'(x)^2dx}=0
\]
Como es suma de términos cuadráticos, la única opción posible es que ambos sean 0 para casi todo punto. Pero entonces, el representante continuo de $f$ sería la función constante 0, cosa imposible ya que en $-1$ y $1$ debe valer uno. En conclusión, no existe el mínimo.

\end{document}