\documentclass[12pt]{article}
 
\usepackage[margin=1in]{geometry} 
\usepackage{amsmath,amsthm,amssymb}
\usepackage[spanish]{babel}
\usepackage[utf8]{inputenc}
\usepackage{tikz-cd}
\usepackage{amsmath}
\usepackage[shortlabels]{enumitem}
\usepackage{mathtools}

% cosas entre comillas 
\usepackage{csquotes}

\usepackage{tikz}


\usepackage{xcolor}

%\usepackage{config}

\newtheorem{theorem}{Teorema}[section]
\newtheorem{lemma}[theorem]{Lema}
\newtheorem{prop}[theorem]{Proposición}
\newtheorem{coro}[theorem]{Corolario}
\newtheorem{conj}[theorem]{Conjetura}
\newtheorem{ejercicio}{Ejercicio}[section]
\newtheorem*{ejercicio*}{Ejercicio}
\theoremstyle{definition}
\newtheorem{definition}[theorem]{Definición}
\newtheorem{example}[theorem]{Ejemplo}
\theoremstyle{remark}
\newtheorem{remark}[theorem]{Nota}
\newtheorem{notacion}[theorem]{Notación}
\newcommand{\continuas}[1][]{C^{ #1 }[a,b]}
\newcommand{\continuasabierto}[1][]{C^{ #1 }(a,b)}
\newcommand{\soportecompacto}{\mathcal{D}(a,b)}
\newcommand{\xcero}{(a,b)}
\newcommand{\xcerocerrado}{[a,b]}
\newcommand{\fvariaciones}{F(x,y(x),y'(x))}

\begin{document}

\section{Ejercicios tema 1}

\begin{ejercicio}
  Sean $\Phi\in \mathcal{C}^1(\mathbb{R}^3), y(x)\in \mathcal{C}^2\xcero, y'(x) \neq 0$. Dado el funcional

  \[
    F[y] = \int_{a}^{b}{\Phi(x,y(x),y'(x))dx}
  \]

  demuéstrese la equivalencia de las dos formas siguientes de las
  ecuaciones de Euler-Lagrange.

  \begin{itemize}
  \item $\frac{\partial\Phi}{\partial y}-\frac{d}{dx}\frac{\partial\Phi}{\partial p} = 0 $
  \item $\frac{\partial\Phi}{\partial x} - \frac{d}{dx}(\Phi-y'\frac{\partial\Phi}{\partial p}) = 0$
  \end{itemize}

  \begin{proof}
    Comenzamos viendo el caso $a) \implies b)$. Para ello en primer
    lugar tenemos que comprobar que efectivamente podemos derivar
    $\Phi$ respecto al tercer parámetro. Del apartado $a)$ sabemos que

    \[
      \frac{\partial\Phi}{\partial y} = \frac{d}{dx}\frac{\partial\Phi}{\partial p}
    \]
    
    y del enunciado sabemos que $\Phi$ es de calse $\mathcal{C}^1$
    luego $\frac{\partial\Phi}{\partial p}$ es de $\mathcal{C}^1$.

    Cuando derivamos $\Phi(x,y(x),y'(x))$ obtenemos

    \begin{align*}
      \Phi(x,y(x),y'(x))' & = \frac{\partial\Phi}{\partial x}(x, y(x), y'(x)) \\
                          & = \frac{\partial\Phi}{\partial y}(x, y(x), y'(x))y'(x) \\
                          & = \frac{\partial\Phi}{\partial p}(x, y(x), y'(x))y''(x)
    \end{align*}

    Recordemos que $\frac{\partial\Phi}{\partial p}(x, y(x), y'(x)) = z(x)$ luego

    \begin{align*}
      \frac{d}{dx}(\Phi-y'\frac{\partial\Phi}{\partial p}) & = \Phi_x + \Phi_yy' + \Phi_py'' - y''z -y'z'
    \end{align*}

    Tenemos que el 3º y 4º termino son iguales t el 2º y 4º son
    iguales entre ellos luego obtenemos

        \begin{align*}
      \frac{d}{dx}(\Phi-y'\frac{\partial\Phi}{\partial p}) = \Phi_x
        \end{align*}

        que es lo que queríamos.

        $b)\implies a)$
        
        Todos los pasos que hemos dado son reversibles pero
        necesitamos ver que
        $\frac{\partial\Phi}{\partial p}(x, y(x), y'(x))$ es $\mathcal{C}^1$.

        Llamando $H(x) = \Phi-y'(x)z(x)$, por hipótesis tenemos que
        $H$ es derivable. Despejando tenemos que

        \[
          z = \frac{\Phi-H}{y'}
        \]

        luego se verifica cómo queríamos.

        \begin{align*}
          \Phi_x & = \Phi_x + \Phi_yy' + \Phi_py'' - y''z -y'z' \\
                 & \implies y'(\Phi_Y-z') = 0
        \end{align*}

        e $y' \neq 0$ por hipótesis luego $\Phi_Y - z' = 0$.
  \end{proof}
\end{ejercicio}

\end{document}