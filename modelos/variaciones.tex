\documentclass[12pt]{article}
 
\usepackage[margin=1in]{geometry} 
\usepackage{amsmath,amsthm,amssymb}
\usepackage[spanish]{babel}
\usepackage[utf8]{inputenc}
\usepackage{tikz-cd}
\usepackage{amsmath}
\usepackage[shortlabels]{enumitem}

% cosas entre comillas 
\usepackage{csquotes}
\usepackage{hyperref}

\newcommand{\N}{\mathbb{N}}
\newcommand{\Z}{\mathbb{Z}}
\newcommand{\R}{\mathbb{R}}
\newcommand{\C}{\mathbb{C}}

\newcommand\restr[2]{{% we make the whole thing an ordinary symbol
  \left.\kern-\nulldelimiterspace % automatically resize the bar with \right
  #1 % the function
  \vphantom{\big|} % pretend it's a little taller at normal size
  \right|_{#2} % this is the delimiter
  }}
 
% lo de section sirve en los numeros de las cosas aparezca el numero de la seccion y un punto y luego el otro numero
\newtheorem{theorem}{Teorema}[section]
\newtheorem{lemma}[theorem]{Lema}
\newtheorem{prop}[theorem]{Proposición}
\newtheorem{coro}[theorem]{Corolario}
\newtheorem{conj}[theorem]{Conjetura}

\theoremstyle{definition}
\newtheorem{definition}[theorem]{Definición}
\newtheorem{example}[theorem]{Ejemplo}
\newtheorem{xca}[theorem]{Ejercicio}

\theoremstyle{remark}
\newtheorem{remark}[theorem]{Nota}

\newtheorem{notacion}[theorem]{Notación}

\newcommand{\continuas}[1][]{C^{ #1 }[x_0,x_1]}
\newcommand{\continuasabierto}[1][]{C^{ #1 }(x_0,x_1)}
\newcommand{\soportecompacto}{\mathcal{D}(x_0,x_1)}
\newcommand{\derivada}[2]{\frac{\partial #1 }{\partial #2 }}
\newcommand{\integral}[3]{\displaystyle\int_{ #1 }^{ #2 } #3 }
\DeclareMathOperator\supp{soporte}
% comandos chorra
\newcommand{\xcero}{(x_0,x_1)}
\newcommand{\xcerocerrado}{[x_0,x_1]}
\newcommand{\fvariaciones}{F(x,y(x),y'(x))}
\newcommand{\espacio}{\;\;\;}
\begin{document}

\title{Modelos Matemáticos II}
\author{Antonio Gámiz Delgado\\ Universidad de Granada} 
 
\maketitle

\section{Cálculo de variaciones}

\subsection{Herramientas previas}

\begin{definition}
\label{funcionestest}
Dado $I$ intervalo, se llama \textit{espacio de funciones test} al conjunto:

\[
\mathcal{D} = \{\phi\in C^{\infty}(a,b): \; \exists J\subset(a,b) \text{ compacto: } \phi(x)=0 \text{ si } x \in J\}
\]
\end{definition}

\begin{lemma}
\label{existanciaphi}
Dado $x_0\in(a,b)$ y $\varepsilon>0$ tal que $[x_0-\varepsilon, x_0+\varepsilon]\subset(a,b)$, existe $\phi\in\mathcal{D}(a,b)$ tal que $\phi(x)>0$ si $x\in(x_0-\varepsilon,x_0+\varepsilon)$ y $\phi(x)=0$ en otro caso.
\end{lemma}

\begin{theorem}
\label{theorem:1.3}
Sea $f\in \continuas$ tal que

\[
\int f(x)\phi(x)dx=0 \hspace{1cm} \forall \phi \in \soportecompacto
\]

Entonces $f(x)=0 \;$  $\forall x\in [x_0,x_1]$.

\end{theorem}

\begin{proof}

Sea $\bar{x}\in \xcero$ y supongamos por reducción al absurdo que $f(\bar{x})\neq 0$. Podemos suponer $f(\bar{x})>0$. Aplicando el teorema de conservación del signo, obtenemos $\varepsilon>0 \; f(x)>0 \text{ si } \; (\bar{x}-\varepsilon,\bar{x}+\varepsilon)$.

Por el lema \ref{existanciaphi}, existe una función test $\phi$ tal que $\phi(x)>0$ si $x\in(\bar{x}-\varepsilon, \bar{x}+\varepsilon)$ y $0$ en otro caso. Luego:

\[
0=\int_{x_0}^{x_1}f(x)\phi(x)dx=\int_{\bar{x}-\varepsilon}^{\bar{x}+\varepsilon}f(x)\phi(x)dx>0 \Rightarrow f(\bar{x})=0
\]

Como $\bar{x}$ era arbitrario, tenemos que $f(\bar{x})=0 \; \forall \bar{x}\in\xcero$, y por la continuidad de $f$ podemos extenderlo a los extremos también, es decir, $f(x_0)=f(x_1)=0$.

\end{proof}

\subsection{Problema general del cálculo de variaciones}

Sea $\Omega\subset\R^3$, definamos $F:\Omega \longrightarrow \R$ tal que $(x,y,p)\longmapsto F(x,y,p)$. Supongamos que $F\in C^1(\Omega)$  respecto de las dos últimas variables, es decir, existen $\derivada{F}{y}$ y $\derivada{F}{p}$, continuas. 

Usando la función anterior, podemos definir el siguiente funcional:

\begin{equation}\label{funcional}
L(y) = \int_{x_0}^{x_1}\fvariaciones dx 
\end{equation}


Nuestro objetivo en este apartado será encontrar \textit{extremales} de ese funcional, es decir, máximos o mínimos.

\begin{notacion}
Normalmente, a las derivadas parciales las denotaremos por:

\[
\derivada{F}{x}=F_x
\]

\end{notacion} 

Los \textit{extremales} los buscaremos entre los elementos de un conjunto de funciones cumpliendo ciertas propiedades:

\begin{definition} 

Sea $\Omega\subset\R^3$ y $F:\Omega\longrightarrow \R$ funcional en las condiciones anteriores. Definimos entonces el siguiente conjunto:

\begin{equation}\label{espaciofunciones}
D=\{y\in\continuasabierto\cap\continuas[1]: \text{ se cumplen (a),(b) y (c)}\}
\end{equation}

\begin{enumerate}[(a)]

\item $(x,y(x),y'(x))\in\Omega \espacio \forall x\in\xcero$
\item $y(x_0)=y_0$ e $y(x_1)=y_1$ (\textit{Condición de contorno})
\item $\displaystyle\int_{x_0}^x\fvariaciones dx<+\infty \espacio \forall x\in \xcero$ 
\end{enumerate}

\end{definition}

El siguiente teorema nos proporcionará una condición sobre las derivadas parciales de $F$, que nos ayudará a buscar \textit{extremales}. Para su demostración necesitaremos el siguiente lema:

\begin{lemma}
\label{lemmatecnico}
Sea $\{s_n\}\longrightarrow 0$ una sucesión de números reales, existe $n_0\in\N$ tal que si $y\in D$, entonces $y+s_n\phi\in D \espacio \forall n\geq n_0$.
\end{lemma}

Lo que nos asegura este lema es que podamos sumar una perturbación \textit{pequeña} a nuestro extremal sin \textit{salirnos} de $D$.

\begin{theorem}
\label{theorem:1.7}
Si $\bar{y}\in D$ es un extremal, entonces:
\[
\integral{x_0}{x_1}{F_y(x,\bar{y}(x), \bar{y}'(x))}\phi(x)dx+\integral{x_0}{x_1}{F_p(x,\bar{y}(x), \bar{y}'(x))}\phi'(x)dx=0 \espacio \forall \phi\in \soportecompacto
\]

A $\bar{y}$ se le suele llamar función crítica.
\end{theorem}

\begin{proof}
Sean $\bar{y}\in D$ extremal y $\phi\in\soportecompacto$. Definimos el funcional $g:\R\longrightarrow\R$ tal que $g(s)=L(\bar{y}+s\phi)$.

Por el lema anterior, existe $\varepsilon>0$ tal que $g$ está bien definida en $(-\varepsilon,\varepsilon)$.

Ahora queremos derivar $g$ respecto de $s$, pero necesitamos que esté definida en un intervalo cerrado (por el teorema de derivación). Para ello, tomamos un intervalo cerrado $J$ de forma que $\supp (\phi)\subset J$. \textbf{(AQUI TENGO DUDAS MIRAR NOTAS CLASE)}

Derivamos $g$ respecto de $s$:

\[
g'(s)=\left(
\integral{\xcerocerrado\backslash J}{}{F(x,\bar{y},\bar{y}')dx}+
\integral{J}{}{F(x,\bar{y}+s\phi(x),\bar{y}'+s\phi'(x))dx}
\right)'=
\]
\[
= \integral{J}{}{\left(F_y(x,\bar{y}+s\phi(x),\bar{y}'+s\phi'(x))\phi(x)+F_p(x,\bar{y}+s\phi(x),\bar{y}'+s\phi'(x))\phi'(x)\right)dx}=
\]
\[
\integral{\xcerocerrado}{}{\left(F_y(x,\bar{y}+s\phi(x),\bar{y}'+s\phi'(x))\phi(x)+F_p(x,\bar{y}+s\phi(x),\bar{y}'+s\phi'(x))\phi'(x)\right)dx}
\]

Si evaluamos ahora $g'$ en 0, tenemos:

\[
g'(0)=\integral{x_0}{x_1}{\left(F_y(x,\bar{y},\bar{y}')\phi+F_p(x,\bar{y},\bar{y}')\phi'\right)dx}=0 \espacio(\bar{y}\text{ extremal})
\]

\end{proof}

El teorema anterior da pie a la siguiente definición:

\begin{definition}
\label{gateaux}

Sea $\Omega\subset X$ un abierto de un espacio de Banach, $X$. Sean $L:\Omega\longrightarrow\R$, $y\in \Omega$, $\phi\in X$, se define la \textit{derivada de Gateaux} como:

\[
Dg(L(\bar{y}))(\phi)=\frac{d}{ds}\Big|_{s=0}L(\bar{y}+s\phi)
\]

\end{definition}

\subsubsection{Ecuación de Euler}

Usando el teorema anterior vamos a llegar a una ecuación diferencial de segundo orden que nos ayudará a resolver este problema. Supongamos que tenemos $y\in C^2$, función crítica y $F\in C^2(\Omega)$, definimos $Z(x)=F_p(x,y(x),y'(x))\in C^1(x_0,x_1)$. Para continuar necesitamos un lema previo:

\begin{lemma}
Sea $Z\in C^1(x_0,x_1)$, entonces:

\[
\integral{x_0}{x_1}{Z(x)\phi'(x)dx}=-\integral{x_0}{x_1}{Z'(x)\phi(x)dx} \espacio \forall \phi\in\soportecompacto
\]
\end{lemma}

Este lema nos permite \enquote{intercambiar la derivada de sitio}.

Usando ahora el Teorema \ref{theorem:1.7} (podemos usarlo porque $y$ es función crítica) y el lema anterior, tenemos:

\[
0=\integral{x_0}{x_1}{F_y(x,y, y')}\phi(x)dx+\integral{x_0}{x_1}{F_p(x,y,y')}\phi'(x)dx= 
\]
\[
=\integral{x_0}{x_1}{F_y(x,y, y')}\phi(x)dx+\integral{x_0}{x_1}{Z(x)}\phi'(x)dx=
\]
\[
=\integral{x_0}{x_1}{F_y(x,y, y')}\phi(x)dx-\integral{x_0}{x_1}{Z'(x)}\phi(x)dx=
\]
\[
=\integral{x_0}{x_1}{\Big(F_y(x,y, y')-Z'(x)\Big)}\phi(x)dx=0 \espacio \forall\phi\in\soportecompacto
\]

Y usando ahora el Teorema \ref{theorem:1.3} nos queda:

\[
F_y(x,y,y')-Z'(x)=0 \espacio \forall x \in\xcero
\]

Que denoteramos por:

\[
\frac{d}{dx}F_p-F_y(x,y,y')=0 \espacio \textbf{(Ecuación de Euler)}
\]

Las condiciones sobre $F$ se pueden rebajar con el siguiente teorema:

\begin{theorem} 

Si $F\in C^1_{yp}$, $y'\in C^1$, función crítica, entonces:
\[
Z(x)=F_p(x,y(x),y'(x))\in C^1
\]
\end{theorem}

\textbf{AHORA CREO QUE QUEREMOS RESOLVER LA ECUACION NO¿?¿?¿?}


\begin{lemma}
Sea $\phi\in\mathcal{D}(a,b)$, entonces:
\[
\phi \text{ admite primitiva } \Longleftrightarrow \integral{a}{b}{\phi(x)dx}=0
\]
\end{lemma}

\begin{lemma}
Sea $f\in C(a,b)$ tal que $\integral{}{}{f\phi'(x)dx=0}\espacio\forall\phi\in\mathcal{D}(a,b)\Longrightarrow f \text{ es constante.}$
\end{lemma}

\end{document}
