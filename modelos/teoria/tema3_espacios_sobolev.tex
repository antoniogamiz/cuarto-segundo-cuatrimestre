\chapter{Espacios de Sobolev}

A lo largo de los temas 1 y 2, hemos construido una serie de herramientas básicas en la resolución de ciertos modelos matemáticos. A continuación, vamos a enlazar los conceptos vistos anteriormente y desarrollar los primeros modelos de esta asignatura.

\section{Enlace}

Antes de definir los \textit{espacios de Sobolev}, recordemos algunas propiedades del espacio 
\[L^2(a,b)=\displaystyle\{\funcion{f}{(a,b)}{\R}\;:\; \integral{a}{b}{|f(x)|^2dx<+\infty}\displaystyle\}\]
La norma de este espacio venía dada por $\norm{f}_2=\sqrt{\integral{a}{b}{f(x)^2dx}}$ e inducía el siguiente producto escalar: $<f,g>=\integral{a}{b}{f(x)g(x)dx}$ (nótese que no usamos el conjugado de $g$ ya que estamos trabajando sobre $\R$, no sobre $\C$). Además, $L^2(a,b)\subset L^1(a,b)$ ya que 
\[
\integral{a}{b}{f(x)}=\integral{a}{b}{|f(x)|\cdot 1dx}\leq \left(\integral{a}{b}{f(x)^2dx}\right)^{1/2}\left(b-a\right)^{1/2}
\]
donde hemos usado la desigualdad de Schwartz para el último paso. Con estas propiedades en mente, podemos definir el primer espacio de Sobolev:
\[
\sobolev{1}=\conjunto{f\in\lebesgue{2}: f \text{ tiene derivada débil }f' \text{ y } f'\in\lebesgue{2}}
\]
Este espacio tiene la particularidad de que también cuenta con un producto escalar:
\[
<f,g>=\integral{a}{b}{f(x)g(x)dx}+\integral{a}{b}{f'(x)g'(x)dx}
\]
Por supuesto, $\sobolev{1}$ es un espacio de Hilbert con la norma $\norm{f}=\sqrt{\norm{f}^2_2+\norm{f'}_2^2}$. Es importante recordar que si $f\in\sobolev{1}$, no tiene por qué ser continua (como muestra \textit{The devil staircase}), pero podemos elegir un representante continuo en su clase de equivalencia, es decir, $\sobolev{1}\hookrightarrow C[a,b]$. De hecho:
\begin{prop}\label{inclusion continua}
La aplicación inclusión $\funcion{i}{\sobolev{1}}{C[a,b]}$ es continua.
\end{prop}
\begin{proof}
Pendiente.
\end{proof}
Además, si $f\in C^1[a,b]$, tiene derivada clásica, luego también tiene derivada débil, luego $f\in\sobolev{1}$. Resumiendo, hemos construido la siguiente cadena:
\[
C^1[a,b]\subset\sobolev{^1}\subset C[a,b]\subset \lebesgue{2}
\]
Definimos ahora el siguiente espacio de Sobolev:
\[
\sobolev{2}=\conjunto{f\in\lebesgue{2}: \exists f',f''\in\lebesgue{2}\text{ (débiles)}}
\]
Usando la propiedad \ref{derivadanesima}, tenemos que $\sobolev{2}\subset C^1[a,b]$ y repitiendo el argumento anterior $C^2[a,b]\subset \sobolev{2}$, es decir:
\[
C^2[a,b]\subset \sobolev{2}\subset C^1[a,b]
\]
Repitiendo este proceso iterativamente, podemos definir el $n$-ésimo espacio de Sobolev:
\[
\sobolev{n}=\conjunto{f\in\lebesgue{2}: \exists f',\dots,f^{n)}\in\lebesgue{2} \text{ (débiles)}}
\]
con la propiedad:
\[
C^{n}[a,b]\subset\sobolev{n}\subset C^{n-1}[a,b]\subset\sobolev{n-1}\subset\dots\subset\sobolev{1}\subset\lebesgue{2}
\]

¿Qué tienen de particular estos espacios? Que son de Hilbert, es decir, vamos a poder usar todas las herramientes que tenemos de Análisis Funcional para resolver algunos problemas como veremos en los siguientes apartados.

\begin{definition}
Diremos que $\funcion{F}{[a,b]\times\R^2}{\R}$ es una \textit{función de Carathéodory} si cumple:
\begin{enumerate}[(a)]
\item Para casi todo punto $x\in(a,b)$, $F(x,y,p)$ es continua en $(y,p)$.
\item Para casi todo punto $(y,p)\in\R^2$, la función $x\mapsto F(x,y,p)$ es medible.
\item Dado $K\subset[a,b]\times\R^2$ compacto, existe una función $m_k(x)\in\lebesgue{1}$ tal que $\valorabsoluto{F(x,y,p)}\leq m_k(x)$ $\forall(x,y,p)\in K$.
\end{enumerate}
\end{definition}

Para que la teoría que vamos a desarrollar a continuación tenga sentido, vamos a imponer de ahora en adelante que $F$, $F_y$ y $F_p$ sean de Carathéodory. Haciendo uso de la propiedad (c), podemos definir el funcional $\funcion{L}{\sobolev{1}}{\R}$:
\[
L(y)=\integral{a}{b}{F(x,y(x),y'(x))dx}
\]
que está bien definido ya que que $F$ es medible y está acotada por una función que es integrable. Tomando $\phi\in\soportecompacto$, podemos definir otra función $\funcion{g}{[a,b]}{\R}$ por $g(s)=L(y+s\phi)$, derivarla respecto a $s$ y evaluarla en 0 (a esta expresión la llamamos \textit{derivada de y a lo largo de $\phi$}):
\[
g'(0)=DL_y(\phi)=\integral{a}{b}{F_y(x,y(x),y'(x)dx}+\integral{a}{b}{F_p(x,y(x),y'(x)dx}
\]
\newpage

\section{Modelos de cuerdas}

Procedemos a la introducción del \textit{modelo de cuerdas}. Comenzaremos realizando el planteamiento más simple, densidad constante y extremos fijos. Posteriormente, eliminaremos la primera hipótesis y resolveremos dos casos: en el primero partimos de una función de densidad dada explícitamente. Por el contrario, en el segundo caso, lo resolveremos dada una función de densidad arbitraria. A continuación, supondremos que el puente se haya \textit{sujeto} por varias cuerdas elásticas. Finalmente, no supondremos fijos los extremos.

Aunque ya hemos desarrollado una cantidad de resultados considerable, todavía precisamos de ciertos resultados, mayoritariamente del \textit{Análisis Funcional}, que iremos introduciendo al mismo tiempo que el modelo. Como estamos trabajando sobre espacios de Sobolev, que son de Hilbert, podremos usarlos sin mucha complicación.

\begin{figure}[h]
   \center
  \includegraphics[scale=0.5]{img/cuerdanormal.png}
\end{figure}

\subsection{Extremos fijos y densidad constante}

Supongamos tener una \textit{cuerda elástica}, colgada entre dos puntos, 0 y 1, es decir, nuestra cuerda está representada por una función $\funcion{y}{[0,1]}{\R}$ en $\sobolev{1}$ tal que $y(0)=y(1)=1$. El objetivo del modelo será encontrar la \textit{cuerda} de mínima energía. En este caso, vamos a suponer que la \textit{densidad} de la cuerda es una constante $m\in\R^+$. 

Denotamos por $E_p$ a la energía potencial y por $E_e$ a la energía elástica:
\[
E_e=\frac{1}{2}\integral{0}{1}{y'(x)^2dx} \espacio\espacio E_p=\integral{0}{1}{my(x)dx}
\]
Minimizar la energía total de la cuerda, es lo mismo que minimizar el funcional:
\[
L(y)=\frac{1}{2}\integral{0}{1}{y'(x)^2dx}+\integral{0}{1}{my(x)dx}
\]
Usando la notación usual:
\[
L(y)=\integral{0}{1}{F(x,y,p)dx} \; \text{ donde } \;\; F(x,y,p)=\frac{1}{2}p^2+my
\]
Supongamos que $y\in\sobolev{1}$ es un punto crítico del funcional $L:\sobolev{1}\longrightarrow\R$ para poder usar la teoría de la \textit{ecuación de Euler}.
Si calculamos las derivadas parciales de $F(x,y,p)$:
\[
Z(x)=F_p(x,y,p)=p=y'(x) \espacio Z'(x)=F_y(x,y,p)=m
\]
vemos que $Z(x)$ tiene derivada débil ($Z'(x)$), como consecuencia, $y'(x)$ es continua (proposición \ref{representantecontinuo}). Pero además, $Z'(x)=(y'(x))'=y''(x)=m$, continua, por lo tanto $y\in C^2[0,1]$. En resumen, tenemos que resolver la siguiente EDO de segundo orden:
\[
\left\{
\begin{array}{rl}
y''(x) & = m \\
y(0) & = 0 \\
y(1) & = 0
\end{array}
\right.
\]
Si recordamos algo de ecuaciones diferenciales, nos damos cuenta de que las condiciones iniciales están \textit{mal planteadas}. Para poder resolver la ecuación, necesitamos dos conciones sobre el mismo punto: una en $y$ y otra en $y'$. Lo podemos solucionar usando el llamado \textit{método de tiro}, que consiste en darle un valor arbitrario a la condición que nos falta, resolver la ecuación y despejar el valor posteriormente. Asumiendo que $y'(0)=\alpha\in\R$, nos queda:
\[
\left\{
\begin{array}{rl}
y''(x) & = m \\
y'(0) & = \alpha \\
y(0) & = 0
\end{array}
\right.
\]
con solución $y(x)=\integral{0}{x}{\left(\alpha+msds\right)}=m\frac{x^2}{2}+\alpha x$. Usando $y(1)=0$, obtenemos que $\alpha=-\frac{m}{2}$. Por lo que la solución del modelo es:
\[
y(x)=\frac{m}{2}x(x-1) \;\; \forall x \in[0,1]
\] 

\subsection{Extremos fijos y densidad no constante}

El planteamiento es igual al anterior, pero suponemos que la densidad en lugar de ser constante es la función $\funcion{q}{[0,1]}{\R}$ dada por:
\[
q(x)=\left\{\begin{array}{cc}
1 & x \in[0,\frac{1}{2}] \\
\frac{1}{2} & x \in(\frac{1}{2}, 1] 
\end{array}
\right.
\]
Cabe resaltar que $q(x)$ no es continua (presenta un salto en $x=\frac{1}{2}$, pero no importa a la hora de resolverlo. En este caso, el funcional viene dado por:
\[
L(y)=\integral{0}{1}{\frac{1}{2}y'(x)^2+q(x)y(x)dx}
\]
con $F(x,y,p)=\frac{1}{2}p^2+q(x)y$, $F_p(x,y,p)=p$, $F_y(x,y,p)=q(x)$. La ecuación diferencial a resolver es (usando de nuevo el método de tiro):
\[
\left\{
\begin{array}{rl}
y''(x) & = q(x) \\
y(0)  = 0,y'(0) & = \alpha \\
y(1)= 0, y'(1) & = \beta
\end{array}
\right.
\]
Tenemos el problema de que $y\notin C^2[0,1]$. Intentemos arreglarlo:
\begin{itemize}
\item Si $x\in(0,\frac{1}{2}) \Rightarrow y''(x)=1 \Rightarrow y\in C^2(0,\frac{1}{2}) \Rightarrow y'(x)=\integral{0}{x}{1dx}+y'(0)=x+\alpha$\\
 $\Rightarrow y(x)=\integral{0}{x}{x+\alpha dx}=\frac{x^2}{2}+x\alpha$.
\item Si $x\in(\frac{1}{2},1) \Rightarrow y''(x)=1/2 \Rightarrow y\in C^2(\frac{1}{2},1) \Rightarrow y'(x)=y'(1)-\integral{x}{1}{\frac{1}{2}dx}=\beta+\frac{x}{2}-\frac{1}{2}$\\
$\Rightarrow y(1)-y(x)=\integral{x}{1}{\frac{x}{2}-\frac{1}{2}+\beta dx}\Rightarrow y(x)=\frac{x^2}{2}-\frac{x}{2}+\beta(x-1)+\frac{1}{4}$
\end{itemize}
Quedando:
\begin{equation}\label{y}
y(x)=\left\{
\begin{array}{cc}
\frac{x^2}{2}+x\alpha & x\in[0,1/2]) \\
\frac{x^2}{2}-\frac{x}{2}+\beta(x-1)+\frac{1}{4} & x\in(1/2,1]
\end{array}
\right.
\end{equation}

\begin{equation}\label{yprima}
y'(x)=\left\{
\begin{array}{cc}
x+\alpha & x\in[0,1/2]) \\
\frac{x}{2}-\frac{1}{2}+\beta & x\in(1/2,1]
\end{array}
\right.
\end{equation}

Ahora tenemos que resolver el sistema de $\alpha$ y $\beta$. Obtendremos dos ecuaciones de imponer que $y$ e $y'$ sean continuas en $\frac{1}{2}$, es decir, de que las dos partes evaluadas en ese punto coincidan. Por lo que a partir de \eqref{y} y \eqref{yprima} conseguimos,evaluando en $\frac{1}{2}$ e igualando:
\begin{equation}
\left\{
\begin{array}{cc}
\alpha+\beta & =0 \\
\alpha-\beta & =-\frac{3}{4}
\end{array}
\right.
\end{equation}
Resolviendo el sistema, $\alpha=-\frac{3}{8}$, $\beta=\frac{3}{8}$. Quedando finalmente la siguiente solución al modelo: (esta mal)
\begin{equation}\label{y}
y(x)=\left\{
\begin{array}{cc}
\frac{x^2}{2}-\frac{x}{4} & x\in[0,1/2]) \\
\frac{x^2}{2}-\frac{x}{2}+\frac{1}{4} & x\in(1/2,1]
\end{array}
\right.
\end{equation}

\textbf{ESTAS CUENTAS NO ESTAN ACTUALIZADAS, AUNQUE ESTAN MAL :D}

\begin{figure}[h]
   \center
  \includegraphics[scale=0.6]{img/puenteflotante.png}
\end{figure}

Que como vemos, floa, asi que algo hay mal :D.

\subsection{Extremos fijos y densidad arbitraria}

El planteamiento es igual al caso más simple, pero ahora la función de densidad, $q(x)$, la consideraremos en $C^\infty$. Sin la expresión de la función $q(x)$, no podemos resolver el problema de forma explícita, pero sí podemos asegurar la existencia de solución. El siguiente procedimiento lo repetiremos varias veces: definimos un espacio en donde buscaremos nuestra solución ($\sobolevcero[0,1]{1}$), plantearemos el problema (condición de extremal), definiremos un producto escalar y demostraremos que el espacio definido con ese producto escalar es de Hilbert, para usar el Teorema de Riesz-Fréchet.

Comencemos definiendo nuestro espacio de soluciones:
\[
\sobolevcero[a,b]{1}=\conjunto{y\in\sobolev{1}: y(a)=y(b)=0}
\]
Recordemos que en este espacio están las funciones de $\lebesgue{2}$, que tienen derivada débil. Por lo tanto, por la proposición \ref{representantecontinuo} podemos elegir una $y$ que sea continua.
\begin{prop}
El espacio vectorial $\sobolevcero{1}$, es cerrado.
\end{prop}   
\begin{proof}
Sea una secuencia de funciones convergentes en $\sobolevcero{1}$, $f_n\longrightarrow f$. Usando la proposición \ref{inclusion continua} y $\sobolevcero{1}\subset\sobolev{1}$, tenemos que $f_n(a)\longrightarrow f(a)$ y $f_n(b)\longrightarrow f(b)$. Luego $f_n(a)=f_n(b)=0 \;\forall n\in\N \Rightarrow f(a)=f(b)=0 \Rightarrow f\in\sobolevcero{1}$.\\
\end{proof}

Una vez definido nuestro nuevo espacio, pasamos a resolver el modelo. Como de costumbre, suponemos que $y\in\sobolevcero[0,1]{1}$ es extremal y usamos la teoría de Euler:
\[
L(y)=\frac{1}{2}\integral{0}{1}{y'(x)^2dx}+\integral{0}{1}{q(x)y(x)dx}
\]
Usando la notación usual:
\[
L(y)=\integral{0}{1}{F(x,y,p)dx} \; \text{ donde } \;\; F(x,y,p)=\frac{1}{2}p^2+q(x)y
\]
Calculando sus derivadas parciales
\[
F_p(x,y,p)=p \espacio F_y(x,y,p)=q(x)
\]
podemos calcular $\funcion{DL_y}{\sobolevcero[0,1]{1}}{\R}$ y usar la concidición de extremal de $y$ (teorema \ref{theorem:1.7}):
\[
DL_y(\phi)=\integral{0}{1}{y'(x)\phi'(x)dx}+\integral{0}{1}{q(x)\phi(x)dx}=0 \espacio \forall \phi\in\sobolevcero[0,1]{1}
\]
Si lo miramos como una derivada débil, vemos que si $Z(x)=y'(x) \Rightarrow Z'(x)=q(x) \Rightarrow y''(x)=q(x)$, en sentido débil.
\begin{definition}
\label{soluciondebil}
$y\in\sobolevcero{1}$ es \textit{solución débil} de $y''(x)=q(x) \; \forall x 
\in [a,b]$, si se verifica:
\[
\integral{0}{1}{y'(x)\phi'(x)dx}+\integral{0}{1}{q(x)\phi(x)dx}=0 \espacio \forall \phi\in\sobolevcero[0,1]{1}
\]
\end{definition}
Cabe resaltar, el hecho de que en la anterior definición, $y$ está en $\sobolevcero[0,1]{1}$ y estamos resolviendo una EDO de segundo orden, es decir, no sabemos nada sobre la segunda derivada de $y$ (solo de la primera).

Definimos un producto escalar en $\sobolevcero[a,b]{1}$ y veamos que es de Hilbert:
\begin{prop}
El espacio $\sobolevcero[a,b]{1}$ con el siguiente producto escalar:
\[
f,g\in\sobolevcero[a,b]{1}, \espacio <f,g>=\integral{a}{b}{f'(x)g'(x)dx}
\]
es un espacio de Hilbert.
\end{prop}
\begin{proof}
Sea $f_n\in\sobolevcero{1} \;\forall n\in\N$ una sucesión de Cauchy, entonces:
\[
\forall \varepsilon>0 \; \exists n_0: \; n,m\geq n_0 \; \norm{f_n-f_m}_{\sobolevcero{1}}<\varepsilon
\]
Recordemos que si tenemos un producto escalar definido, podemos expresar la norma de un elemento del espaico como la raíz cuadrada del producto escalar de el elemento consigo mismo, es decir:
\[
\norm{f}_{\sobolevcero{1}}=\sqrt{<f,f>}
\]
Usando esa propiedad:
\[
\varepsilon > \norm{f_n-f_m}_{\sobolevcero{1}}=\sqrt{<f_n-f_m,f_n-f_m>}=\sqrt{\integral{a}{b}{(f_n(x)-f_m(x))^2dx}}=\norm{f_n'-f_m'}_2
\]
Luego $f_n'$ converge a una cierta $g$ en $\lebesgue{2}$. Definiendo:
\[
\tilde{f_n}(x)=\integral{a}{b}{f_n'(s)ds} \Rightarrow \tilde{f_n}(x) \longrightarrow \integral{a}{x}{g(s)ds} \Rightarrow f(x) = \limitemasinfinito{n}{\tilde{f_n}(x)}
\]
Ya tenemos nuestro candidato a límite, comprobemos que pertenece a $\sobolevcero{1}$:
\begin{itemize}
\item $f(a)=0$ y $f(b)=\integral{a}{b}{g(s)ds}=f_n'(b)-f_n'(a)=0$.
\item $f$ tiene derivada débil (igual a $g$) por el teorema \ref{fundamentalcalculo}.
\end{itemize}
Luego $f\in\sobolevcero{1}$.
Por último, vemos que converge:
\[
\norm{f_n-f}_{\sobolevcero{1}}=\norm{f_n'-g}_2\longrightarrow 0
\]
\end{proof}

Una vez comprobado que el espacio es de Hilbert, solo nos falta recordar el Teorema de Riesz-Fréchet:
\begin{theorem}[Riesz-Fréchet]
\label{riesz-frechet}
Si $H$ es un espacio de Hilbert y $f\in H^*$, existe $y\in H$ tal que:
\[
f(x)=<x,y> \; \forall x\in H
\]
\end{theorem}

Sea ahora $\phi\in\sobolevcero{1}$, definimos $\funcion{R}{\sobolevcero{1}}{\R}$ por $R(\phi)=-\integral{a}{b}{q(x)\phi(x)dx}\in\left(\sobolevcero{1}\right)^*$. Por el teorema de Riesz-Fréchet, existe $y\in\sobolevcero{1}$ tal que $R(\phi)=<y,\phi>$.

Si desarrollamos la última expresión, nos damos cuenta de que es igual a la condición en la definición \ref{soluciondebil}, que es justo lo que buscábamos.

\subsection{Puente sujeto por cuerdas}

Al modelo del puente anterior le vamos a añadir unas cuerdas elásticas para soportarlo, cada una con una constante de elasticidad distinta, dado por $K(x)\geq 0$.
Si $K(x)=0$ en algún punto $x\in[0,1]$, significa que en ese punto no hay cuerda sujetando a la de abajo.
\begin{figure}[H]
   \center
  \includegraphics[scale=0.6]{img/puentecuerdas.png}
\end{figure}
En esta nueva versión del modelo tenemos que tener en cuentra otra energía más, la aportada por los cables, $E_c$:
\[
E_c=\frac{1}{2}\integral{0}{1}{K(x)y^2(x)dx}
\]
Con lo que el funcional quedaría:
\[
L(y)=\frac{1}{2}\integral{0}{1}{y'(x)^2dx}+\frac{1}{2}\integral{0}{1}{K(x)y(x)^2dx}+\integral{0}{1}{q(x)y(x)dx}
\]
Usando la notación usual:
\[
L(y)=\integral{0}{1}{F(x,y,p)dx} \; \text{ donde } \;\; F(x,y,p)=\frac{1}{2}p^2+\frac{1}{2}K(x)y^2+q(x)y
\]
Suponiendo que $y\in\sobolevcero[0,1]{1}$ es extremal, la condición de punto crítico es:
\[
DL_y(\phi)=\integral{0}{1}{y'(x)\phi'(x)dx}+\integral{0}{1}{\left(K(x)y(x)+q(x)\right)\phi(x)dx}=0 \;\; \forall \phi\in\sobolevcero[0,1]{1}
\]
Denotando $Z(x)=y'(x)$ y viendo la expresión anterior en sentido débil, tenemos que $Z'(x)=K(x)y(x)+q(x)$ es su derivada débil. Luego la EDO de orden 2 de este modelo es (que la obtenemos a partir de la ecuación de Euler):
\[
-y''(x)=+K(x)y(x)+q(x)=0
\]
Ahora, al igual que en el modelo anterior, vamos a definir un cierto producto escalar de forma que el teorema de Riesz-Frechét nos de la existencia de solución.
\begin{prop}
El espacio $\sobolevcero[a,b]{1}$ con el siguiente producto escalar:
\[
f,g\in\sobolevcero[a,b]{1}, \espacio <f,g>=\integral{a}{b}{f'(x)g'(x)dx}+\integral{a}{b}{K(x)f(x)g(x)dx}
\]
con $K\in\lebesgue{2}$, $K(x)\geq 0$ $\forall x\in[a,b]$, es un espacio de Hilbert.
\end{prop}
\begin{proof}

\end{proof}

