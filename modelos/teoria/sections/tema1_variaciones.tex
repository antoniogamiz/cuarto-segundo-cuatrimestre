\chapter{Cálculo de variaciones}

\section{Herramientas previas y repaso}

Necesitaremos recordar algunas nociones y teoremas básicos sobre derivabilidad:

\begin{theorem}[derivada de una integral respecto de un parámetro]
\label{derivadaparametro}
Sea $X\subset\R^n$ y $(X,\mathcal{A},\mu)$ un espacio medible, $I$ un intervalo cerrado y sea $\funcion{f}{I\times X}{\R}$ tal que:

\begin{enumerate}[(a)]
\item $\forall t \in I$, $x\mapsto f(t,x)$ es integrable.
\item $\forall x\in X$, la función $t\mapsto f(t,x)$ es derivable en $t\in I$.
\item Existe $\funcion{g}{X}{\R}$ integrable tal que
\[
\left|\derivada{f}{t}(t,x)\right|\leq |g(x)| \espacio \forall x\in X\; \forall t \in I
\]
Entonces la función $F(t)=\int_Xf(t,x)dx$ es derivable en $t_0$ y la derivada es 
\[
F'(t_0)=\displaystyle\int_X\derivada{f}{t}(t_0,x)dx
\]
\end{enumerate}

\end{theorem}

TODO: añadir resto de resultados usados de otras asignaturas que no recordábamos

\section{Problema general del cálculo de variaciones}

Primeramente definimos un tipo de funciones llamadas \textit{funciones test} o \textit{funciones de la clase de Schwartz}, junto con algunos resultados que usaremos bastante para trabajar con ellas.

\begin{definition}
\label{funcionestest}
Dado $I$ intervalo, se llama \textit{espacio de funciones test} al conjunto:
\[
\mathcal{D} = \mathcal{D}(a,b) = \{\phi\in C^{\infty}(a,b): \; \exists J\subset(a,b) \text{ compacto: } \phi(x)=0 \text{ si } x \notin J\}
\]
\end{definition}

\begin{lemma}
\label{existenciaphi}
Dado $x_0\in(a,b)$ y $\varepsilon>0$ tal que $[x_0-\varepsilon, x_0+\varepsilon]\subset(a,b)$, existe $\phi\in\mathcal{D}(a,b)$ tal que $\phi(x)>0$ si $x\in(x_0-\varepsilon,x_0+\varepsilon)$ y $\phi(x)=0$ en otro caso.
\end{lemma}

\begin{proof}
La demostración la vamos a hacer por construcción. Sea $\funcion{g}{\R}{\R}$ definida por:
\[
g(x)=\left\{
\begin{array}{cc}
e^{1/x} & x<0 \\
0 & x\geq 0
\end{array}
\right.
\]
Tenemos que $\limite{x}{0^-}{g(x)}=0$, luego $g$ es continua. Veamos que de hecho $g\in C^{\infty}$. Su derivada es:
\[
g'(x)=\left\{
\begin{array}{cc}
e^{1/x}\left(-\frac{1}{x^2}\right) & x<0 \\
0 & x> 0
\end{array}
\right.
\]
Mediante un proceso iterativo llegamos a:
\[
g^{n)}(x)=\left\{
\begin{array}{cc}
e^{1/x}\frac{R(x)}{x^{2n}} & x<0 \\
0 & x> 0
\end{array}
\right.
\]
donde $R(x)$ es un cierto polinomio que no nos interesa calcular. Queremos ver que 
\[\limite{x}{0^-}{g^{n)}(x)}=0,\]
con lo cual, $g\in C^n$. Haciendo el cambio $y=-\frac{1}{x}$, el anterior límite equivale a 
\[
\limite{y}{+\infty}{\frac{y^{2n}}{e^y}}=0 \Rightarrow g\in C^{n} \;\forall n \in\N \Rightarrow g\in C^{\infty}
\]
Con lo anterior, solo nos queda definir la función buscada de forma que sea una función test, es decir, la definimos como:
\[
\phi(x)=g(x-(x_0+\varepsilon))g((x_0-\varepsilon)-x) \espacio \varepsilon>0 
\]

\end{proof}


\begin{theorem}
\label{theorem:1.3}
Sea $f\in \continuas$ tal que
\[
\int f(x)\phi(x)dx=0 \hspace{1cm} \forall \phi \in \soportecompacto
\]
Entonces $f(x)=0 \;$  $\forall x\in [a,b]$.
\end{theorem}

\begin{proof}
Sea $\bar{x}\in \xcero$ y supongamos por reducción al absurdo que $f(\bar{x})\neq 0$. Podemos suponer $f(\bar{x})>0$. Aplicando el teorema de conservación del signo, obtenemos $\varepsilon>0$ tal que $f(x)>0 \text{ si } x \in (\bar{x}-\varepsilon,\bar{x}+\varepsilon)$.

Por el lema \ref{existenciaphi}, existe una función test $\phi$ tal que $\phi(x)>0$ si $x\in(\bar{x}-\varepsilon, \bar{x}+\varepsilon)$ y $0$ en otro caso. Luego:
\[
0=\int_{a}^{b}f(x)\phi(x)dx=\int_{\bar{x}-\varepsilon}^{\bar{x}+\varepsilon}f(x)\phi(x)dx>0 \Rightarrow f(\bar{x})=0
\]
Como $\bar{x}$ era arbitrario, tenemos que $f(\bar{x})=0 \; \forall \bar{x}\in\xcero$, y por la continuidad de $f$ podemos extenderlo a los extremos también, es decir, $f(a)=f(b)=0$.

\end{proof}

\subsection{Cálculo de extremales}

Sea $\Omega\subset\R^3$, definamos $F:\Omega \longrightarrow \R$ tal que $(x,y,p)\longmapsto F(x,y,p)$. Supongamos que $F\in C^1(\Omega)$  respecto de las dos últimas variables, es decir, existen $\derivada{F}{y}$ y $\derivada{F}{p}$, continuas. 

Usando la función anterior, podemos definir el siguiente funcional:

\begin{equation}\label{funcional}
L(y) = \int_{a}^{b}\fvariaciones dx 
\end{equation}


Nuestro objetivo en este apartado será encontrar \textit{extremales} de ese funcional, es decir, máximos o mínimos.

\begin{notacion}
Normalmente, a las derivadas parciales las denotaremos por:
\[
\derivada{F}{x}=F_x
\]
\end{notacion}

Los \textit{extremales} los buscaremos entre los elementos de un conjunto de funciones cumpliendo ciertas propiedades:

\begin{definition}
\label{espaciofuncionesbuenas}

Sea $\Omega\subset\R^3$ y $F:\Omega\longrightarrow \R$ funcional en las condiciones anteriores. Definimos entonces el siguiente conjunto:

\begin{equation}\label{espaciofunciones}
D=\{y\in\continuasabierto\cap\continuas[1]: \text{ se cumplen (a),(b) y (c)}\}
\end{equation}

\begin{enumerate}[(a)]
\item $(x,y(x),y'(x))\in\Omega \espacio \forall x\in\xcero$
\item $y(a)=y_0$ e $y(b)=y_1$ (\textit{Condición de contorno})
\item $\displaystyle\int_{a}^b\Big|\fvariaciones\Big|dx<+\infty$ 
\end{enumerate}

\end{definition}

El siguiente teorema nos proporcionará una condición sobre las derivadas parciales de $F$, que nos ayudará a buscar \textit{extremales}. Para su demostración necesitaremos el siguiente lema:

\begin{lemma}
\label{lemmatecnico}
Sea $\{s_n\}\longrightarrow 0$ una sucesión de números reales y $\phi\in \soportecompacto$, existe $n_0\in\N$ tal que si $y\in D$, entonces $y+s_n\phi\in D \espacio \forall n\geq n_0$.
\end{lemma}
\begin{proof}

Sea $\phi\in\soportecompacto$ y $K=\supp\phi$.Tenemos que comprobar que $y+s_n\phi$ cumple las condiciones de la definición \ref{espaciofuncionesbuenas}.

\begin{enumerate}[(a)]
\item Razonemos por reducción al absurdo. Supongamos que hay infinitos valores de $n\in\mathbb{N}$ tales que $(x_n,y(x_n)+s_n\phi(x_n),y'(x_n)+s_n\phi'(x_n))\notin \Omega$ para algún $x_n\in(a,b)$. Tomando una parcial si es necesario, podemos suponer una sucesión $\{x_n\}$ tal que para cada $n\in\mathbb{N}$, $(x_n,y(x_n)+s_n\phi(x_n),y'(x_n)+s_n\phi'(x_n))\notin \Omega \Rightarrow x_n\in\supp\phi$, ya que si no estuviera, tendríamos que $\phi(x_n)=0$ y $(x_n,y(x_n),y'(x_n))\in \Omega$. Como el soporte es compacto, podemos suponer que $\{x_n\}\longrightarrow\bar{x}\in\supp\phi\in(a,b)$. Si tomamos límite:
\[
\limitemasinfinito{n}{(x_n,y(x_n)+s_n\phi(x_n),y'(x_n)+s_n\phi'(x_n))}=(\bar{x},y(\bar{x}),y'(\bar{x}))\notin\Omega 
\]
porque el complementario de $\Omega$ es cerrado y el límite se queda fuera, llegando así a un absurdo. Por tanto, debe existir un $n_0\in\mathbb{N}$ tal que $(x_n,y(x_n)+s_n\phi(x_n),y'(x_n)+s_n\phi'(x_n))\in \Omega\quad\forall n\geq n_0$.
\item Evidente, ya que $\phi(a)=\phi(b)=0$.
\item Tomando $n\geq n_0$,de (a) tenemos:
\[
\integral{a}{b}{\Big|F(x,y(x)+s_n\phi(x),y'(x)+s_n\phi'(x))\Big|dx}=\]
\[
=\integral{(a,b)\backslash K}{}{\Big|F(...)\Big|dx}+\integral{K}{}{\Big|F(...)\Big|dx} < +\infty
\]
El primer término es finito ya que, fuera de $K$, $\phi=0$, luego ese término coincide con $\integral{(a,b)\backslash K}{}{\Big|F(x,y(x),y'(x))\Big|dx}$, que
es finito porque $y\in D$.
El segundo término lo es porque ser la integal de una función continua en un compacto. 
\end{enumerate}
\end{proof}
Lo que nos asegura este lema es que podamos sumar una perturbación \textit{pequeña} a nuestro extremal sin \textit{salirnos} de $D$.

\begin{theorem}
\label{theorem:1.7}
Si $\bar{y}\in D$ es un extremal, entonces:
\[
\integral{a}{b}{F_y(x,\bar{y}(x), \bar{y}'(x))}\phi(x)dx+\integral{a}{b}{F_p(x,\bar{y}(x), \bar{y}'(x))}\phi'(x)dx=0 \espacio \forall \phi\in \soportecompacto
\]
A $\bar{y}$ se le suele llamar \textbf{función crítica}.
\end{theorem}

\begin{proof}
Sean $\bar{y}\in D$ extremal y $\phi\in\soportecompacto$. Definimos el funcional $g:\R\longrightarrow\R$ tal que $g(s)=L(\bar{y}+s\phi)$.

Por el lema anterior, existe $\varepsilon>0$ tal que $g$ está bien definida en $(-\varepsilon,\varepsilon)$.

Ahora queremos derivar $g$ respecto de $s$, pero necesitamos que esté definida en un intervalo cerrado (por el teorema \ref{derivadaparametro}). Para ello, tomamos un intervalo cerrado $J$ de forma que $\supp (\phi)\subset J\subset [a,b]$. 

Derivamos $g$ respecto de $s$:
\[
g'(s)=\left(
\integral{[a,b]}{}{F(x,\bar{y}+s\phi(x),\bar{y}'+s\phi'(x))dx}
\right)'=
\]
\[
=\left(
\integral{[a,b]\backslash J}{}{F(x,\bar{y}+s\phi(x),\bar{y}'+s\phi'(x))dx}+
\integral{J}{}{F(x,\bar{y}+s\phi(x),\bar{y}'+s\phi'(x))dx}
\right)'
\]
Usando que $\integral{[a,b]\backslash J}{}{F(x,\bar{y}+s\phi(x),\bar{y}'+s\phi'(x))dx}\phi(x)$ es una constante (por que $\bar{y}\in D$ y $\phi = 0$ en $[a,b]\backslash J$) cuya derivada es cero, obtenemos que:
\[
g'(s)= \integral{J}{}{\Big(F_y(x,\bar{y}+s\phi(x),\bar{y}'+s\phi'(x))\phi(x)+F_p(x,\bar{y}+s\phi(x),\bar{y}'+s\phi'(x))\phi'(x)\Big)dx}
\]
Como $\phi=0=\phi'$ fuera de $J$
\[
g'(s)= \integral{[a,b]}{}{\Big(F_y(x,\bar{y}+s\phi(x),\bar{y}'+s\phi'(x))\phi(x)+F_p(x,\bar{y}+s\phi(x),\bar{y}'+s\phi'(x))\phi'(x)\Big)dx}
\]
Si evaluamos ahora $g'$ en 0 y usamos que $\bar{y}$ es extremal, tenemos:
\[
g'(0)=\integral{a}{b}{\Big(F_y(x,\bar{y},\bar{y}')\phi+F_p(x,\bar{y},\bar{y}')\phi'\Big)dx}=0 
\]
\end{proof}

\subsection{Ecuación de Euler}

Usando el teorema anterior vamos a llegar a una ecuación diferencial de segundo orden que nos ayudará a resolver este problema. Definimos $Z(x)=F_p(x,y(x),y'(x))$, nuestro objetivo ahora es imponer condiciones suficientes para que $Z(x)\in C^1(a,b)$, para poder derivarla y obtener una ecuación diferencial en $y'$.

Para continuar necesitamos un lema previo:

\begin{lemma}
Sea $Z\in C^1(x_0,x_1)$, entonces:
\[
\integral{a}{b}{Z(x)\phi'(x)dx}=-\integral{a}{b}{Z'(x)\phi(x)dx} \espacio \forall \phi\in\soportecompacto
\]
\end{lemma}

\begin{proof}

Resolviendo la integal por partes tenemos:
\[
\left.\begin{array}{cc}
u=Z(x) & du=Z'(x)dx\\
dv=\phi'(x)dx & v=\phi
\end{array}\right\} \Rightarrow\integral{a}{b}{Z(x)\phi'(x)dx}=Z(x)\phi(x)\Big|_a^b-\integral{a}{b}{Z'(x)\phi(x)dx}=
\]
\[
=-\integral{a}{b}{Z'(x)\phi(x)dx}
\]

\end{proof}

Este lema nos permite \enquote{intercambiar la derivada de sitio}.
Usando ahora el Teorema \ref{theorem:1.7} (podemos usarlo porque $y$ es función crítica) y el lema anterior, tenemos:
\[
0=\integral{a}{b}{F_y(x,y, y')}\phi(x)dx+\integral{a}{b}{F_p(x,y,y')}\phi'(x)dx= 
\]
\[
=\integral{a}{b}{F_y(x,y, y')}\phi(x)dx+\integral{a}{b}{Z(x)}\phi'(x)dx=
\]
\[
=\integral{a}{b}{F_y(x,y, y')}\phi(x)dx-\integral{a}{b}{Z'(x)}\phi(x)dx=
\]
\[
=\integral{a}{b}{\Big(F_y(x,y, y')-Z'(x)\Big)}\phi(x)dx=0 \espacio \forall\phi\in\soportecompacto
\]
Y usando ahora el Teorema \ref{theorem:1.3} nos queda:
\[
F_y(x,y,y')-Z'(x)=0 \espacio \forall x \in\xcero
\]
Que denoteramos por:
\[
\frac{d}{dx}F_p-F_y(x,y,y')=0 \espacio \textbf{(Ecuación de Euler)}
\]
Las condiciones sobre $F$ se pueden rebajar con el siguiente teorema:

\begin{theorem} 
\label{theorem:12}
Si $F\in C^1_{yp}$, $y'\in C^1$, función crítica, entonces:
\[
Z(x)=F_p(x,y(x),y'(x))\in C^1
\]
\[
Z'(x)=F_y(x,y(x),y'(x))
\]
\end{theorem}

Ya tenemos la ecuación que queremos resolver. La demostración del teorema es consecuencia de los anteriores resultados.

\begin{definition}
Una función $\phi\in\soportecompacto$ admite primitiva si existe otra función $\Psi\in\soportecompacto$ tal que $\Phi'=\phi$. 
\end{definition}

\begin{lemma}
\label{lemma:13}
Sea $\phi\in\mathcal{D}(a,b)$, entonces:
\[
\phi \text{ admite primitiva } \Longleftrightarrow \integral{a}{b}{\phi(x)dx}=0
\]
\end{lemma}

\begin{proof}
\hfill\\
$(\Rightarrow)$ Por hipótesis, supongamos que existe $\Psi$ tal que $\phi=\Psi'$. Ahora solo tenemos que integrarla y usar que $\Psi$ vale 0 en los extremos por ser una función de soporte compacto:
\[
\integral{a}{b}{\phi(s)ds}=\integral{a}{b}{\Psi'(s)ds}=\Psi\Big|_a^b=0
\]
$(\Leftarrow)$  Si $\integral{a}{b}{\phi(s)ds}=0$, entonces tenemos la siguiente situación: $\supp\phi\subset[a',b']$ tal que $a<a'\leq b'<b$.

\begin{center}
\includegraphics[scale=0.4]{./img/testfuncion.png}
\end{center}

Definimos $\Psi(x)=\integral{a'}{x}{\phi(s)ds}$ y tenemos $\Psi'=\phi$. Falta probar que $\Psi\in\mathcal{D}(a,b)$.
Si $x\leq a'$, claramente $\Psi(x)=0$ \big($\phi(x)=0\quad\forall x\in(a,a')$\big). Si $x\geq b'$,
\[
\int_{a'}^x\phi(s)ds=\int_{a'}^{b'}\phi(s)ds=\int_{a}^{b}\phi(s)ds=0
\]
Por tanto $\supp\Psi\subset[a',b']$.

\end{proof}

\begin{lemma}
\label{lemma:14}
Sea $f\in C(a,b)$ tal que $\integral{a}{b}{f(x)\phi'(x)dx=0}\espacio\forall\phi\in\mathcal{D}(a,b)\Longrightarrow f \text{ es constante.}$
\end{lemma}

\begin{proof}
Sea $x_0\in(a,b)$, tomamos una función $\phi_{x_0}\in \soportecompacto$ cumpliendo la tesis del lema \ref{existenciaphi}. Podemos tomarla de forma que $\integral{a}{b}{\phi_{x_0}(s)ds}=1$ (multiplicándola por cierta constante). 

Definimos $\Psi(x)=\integral{a}{x}{\phi_{x_0}(s)ds}$ y tomamos $c$ de forma que:
\[
\integral{a}{b}{f(s)\Psi'(s)}=c\integral{a}{b}{\Psi'(s)ds}=c\Psi\Big|_a^b=c
\]
Tomamos $\phi\in\soportecompacto$ y veamos que la función $\phi-\lambda\Psi'$ tiene primitiva para cierto $\lambda\in\R$. Supongamos que tiene media 0, y despejemos $\lambda$:
\[
\integral{a}{b}{\phi(s)-\lambda\Psi'(s)}=0 \Longrightarrow \lambda=\integral{a}{b}{\phi(s)ds}
\]
Haciéndo esa elección de $\lambda$, $\phi-\lambda\Psi'$ tiene primitiva por el lema \ref{lemma:13}.
\[
0=\integral{a}{b}{f(s)(\phi(s)-\lambda\Psi'(s))ds}=\integral{a}{b}{f(s)\phi(s)ds}-\lambda\integral{a}{b}{f(s)\Psi'(s)ds}=
\]
\[
=\integral{a}{b}{f(s)\phi(s)ds}-c\integral{a}{b}{\phi(s)ds}=\integral{a}{b}{(f(s)-c)\phi(s)ds} \Rightarrow f(s)-c=0 \Rightarrow f\equiv c
\]
\end{proof}

\begin{lemma}
Sean $f$,$g$ funciones continuas en $(a,b)$, entonces es equivalente:
\[
\integral{a}{b}{g\phi}+\integral{a}{b}{f\phi'}=0 \espacio \forall\phi\in\soportecompacto \Longleftrightarrow f\in C^1(a,b), g=f'
\]
\end{lemma}

\begin{proof}

($\Rightarrow$)Tomamos $x_0\in(a,b)$ y definimos $\tilde{f}(x)=\integral{a}{x}{g(s)ds}\Rightarrow \tilde{f}\in C^1$.
\[
\integral{a}{b}{\tilde{f}(s)\phi'(s)+g(s)\phi(s)ds}=\integral{a}{b}{\tilde{f}(s)\phi'(s)+\tilde{f}'(s)\phi(s)ds}=\integral{a}{b}{(\tilde{f}\phi)}=\tilde{f}\phi\Big|_a^b=0
\]
Restando la expresión de la hipótesis menos la anterior, obtenemos:
\[
0=\integral{a}{b}{f(s)\phi'(s)ds}-\integral{a}{b}{\tilde{f}(s)\phi'(s)ds}=\integral{a}{b}{(f-\tilde{f})(s)\phi'(s)ds}
\]
Y usnado el lema \ref{lemma:14}, tenemos que $f-\tilde{f}$ es constante, luego $f\in C^1$ ya que $\tilde{f}$ también pertenece a $C^1$.

($\Leftarrow$) 
\[
0=f\phi(x)\Big|_a^b=\integral{a}{b}{(f(x)\phi(x))'dx}=\integral{a}{b}{f(x)\phi'(x)dx+f'(x)\phi(x)dx}
\]
\end{proof}

\section{Problema de Braquistocrona}

Este problema se centra principalmente en averiguar que forma (curva) tenemos que darle a un tobogán para que este sea el más rápido.

A esa curva la vamos a denotar por $Y(t)$ y vamos a suponer que  pertenece a $C^1(0,L)$, es decir, que no tenga picos. Además, vamos a suponer que el tobogán tiene altura máximo 1, y mínima 0, es decir, $Y(0)=1$ e $Y(L)=0$. También necesitamos que el tobogán tenga sentido, es decir, que no tenga subidas ni bajadas muy bruscas, luego necesitamos imponer $Y(x)<1 \espacio \forall x \in (0,L)$.

Recordemos primero algunas nociones de física. Vamos a denotar por $(x(t),y(t))$ a la posición de una persona en el tobogán en el instante $t\in[0,L]$, por $m$ a su masa y por $g$ a la gravedad. Recordemos que la expresión de la energía es $mgy(t)$, la de la velocidad es $v(t)=\sqrt{x'(t)^2+y'(t)^2}$ y la de la energía cinética es $\frac{mv(t)^2}{2}$.

Si hacemos el tobogán suficientemente suave, por el teorema de conservación de la energía, se cumple:

\begin{equation}
\label{equationff}
mgy(t)+\frac{m}{2}(x'(t)^2+y'(t)^2)\equiv cte
\end{equation}
En el primer momento nos dejamos caer, luego en $t=0$, $(\ref{equationff})=mg$

Tenemos entonces $y(t)=Y(x(t)) \Rightarrow y'(t)=Y'(x(t))x'(t)$, sustituyendo en (\ref{equationff}):
\[
mgY(x(t))+\frac{m}{2}x'(t)^2+\frac{m}{2}(Y'(x(t))x'(t))^2=mg
\]
\[
x'(t)^2\left(1+Y'(x(t))^2\right)=2g\left(1-Y(x(t))\right)\Rightarrow x'(t)=\sqrt{2g}\sqrt{\frac{1-Y(x(t))}{1+Y'(x(t))^2}}
\]
Estamos buscando el tiempo de llegada, $T$, ¿cómo lo hacemos? Aplicamos un truco típico de ecuaciones diferenciales:
\[
T = \integral{0}{T}{dt}=\integral{0}{T}{\frac{x'(t)}{x'(t)}dt}=
\frac{1}{\sqrt{2g}}\integral{0}{T}{\frac{\sqrt{1+Y'(x(t))^2}}{\sqrt{1-Y(x(t))}}x'(t)dt}
\]
Que tras el cambio de variable $x=x(t)$ nos queda:
\[
T=\frac{1}{\sqrt{2g}}\integral{0}{L}{\frac{\sqrt{1+Y'(x)^2}}{\sqrt{1-Y(x)}}dx}
\]
Definimos ahora el conjunto de funciones donde vamos a buscar nuestro mínimo:
\[
D=\{y\in C^1(0,L),y(0)=1,y(L)=0, y'(x)<1 \espacio \forall x \in(0,L)\}
\]
Y definimos nuestro funcional:
\[
L(y)=\integral{0}{L}{\frac{\sqrt{1+Y'(x)^2}}{\sqrt{1-Y(x)}}dx}
\]
Usando la notación del principio, tenemos una función $F$ tal que $F(x,y,p)=\sqrt{\frac{1+p^2}{1-y}}$

Usando el Teorema \ref{theorem:12} podemos definir:
\[
Z(x)=\derivada{F}{p}(x,y,y')=\frac{1}{\sqrt{1-y(x)}}\frac{y'(x)}{\sqrt{1+y'(x)^2}} \in C^1(0,T)
\]
Ahora, en lugar de despejar $y'(x)$, vamos a ver que $y\in C^2$. Definiendo $\Psi(y')=\frac{y'}{\sqrt{1+y'^2}}=Z(x)\sqrt{1+y'^2}$. Como $\Psi(s)=\frac{s}{\sqrt{1+s^2}}$ tiene inversa de clase 1, tenemos:
\[
y'(x)=\Psi^{-1}(Z(x)\sqrt{1-y(x)})\Rightarrow y\in C^2
\]
En general, tenemos que si $F$ no depende de $x$, podemos hacer:
\[D(x)=F(y(x),y'(x))-Z(x)y'(x)\Rightarrow D'(x)=
\]
\[=\derivada{F}{y}(y(x),y'(x))y'(x)+\derivada{F}{p}(y(X),y'(x))y''(x)-Z'(x)y'(x)-Z(x)y''(x)=\]
\[
=y'(Z'(x)-Z'(x))=0
\]
Es decir, para alguna constante $C\in\R$, tenemos que:
\[
F(y(x),y'(x))-Z(x)y'=C
\]
En nuestro caso particular, nos quedaría:
\[
\frac{\sqrt{1+y'(x)^2}}{1-y(x)}-\frac{1}{\sqrt{1-y(x)}}\frac{y'(x)^2}{\sqrt{1+y'(x)^2}}=0 \Rightarrow \frac{1}{\sqrt{1-y(x)}}\frac{1}{\sqrt{1+y'(x)^2}}=C
\]
Esa ecuación diferencial, es de variables separadas, deberíamos resolverla con las condiciones iniciales $y(0)=1,y(L)=0$, pero la solución es trascendente, es decir, no tiene expresión explícita, es un cicloide.

\begin{figure}[!ht]
   \center
  \includegraphics[scale=0.5]{img/cicloide.png}
  \caption{Ejemplo de cicloide}
\end{figure}

\section{Relación de ejercicios}

\begin{ejercicio}
Encontrar la curva en la que el siguiente funcional podría alcanzar su extremo:
\[
I(y)=\integral{1}{2}{\Big(y'(x)^2-2xy(x)\Big)dx}
\]
con condiciones de contorno $y(1)=0,y(2)=-1$.
\end{ejercicio}
\textbf{Solución:}
Definimos $F(x,y,p)=p^2-2xy$ y obtenemos:
\[
\derivada{F}{p}=2p, \espacio \derivada{F}{y}=-2x \Rightarrow Z(x)=2y'(x), \espacio Z'(x)=-2x
\]
Resolviendo la ultima ecuación, obteniendo:
\[
Z(x)=-x^2+C \Rightarrow 2y'(x)=-x^2+C \Rightarrow y'(x)=\frac{-x^2+C}{2}\Rightarrow y(x)=-\frac{x^3}{6}+\frac{c}{2}x+D
\]
Usando las condiciones de contorno, podemos calcular el valor de $C$ y $D$, obteniendo:
\[
y(x)=\frac{x-x^3}{6}
\]
\begin{ejercicio}
Encuentra las curvas que unen $(1,3)$ con $(2,5)$, que puedan ser extremos del funcional:
\[
I(y)=\integral{1}{2}{\Big(y'(x)+x^2y'(x)^2\Big)dx}
\]
\end{ejercicio}

\textbf{Solución} $y(x)=-\frac{4}{x}+7$

\begin{ejercicio}
  Sean $\Phi\in \mathcal{C}^1(\mathbb{R}^3), y(x)\in \mathcal{C}^2\xcero, y'(x) \neq 0$. Dado el funcional
  \[
    F[y] = \int_{a}^{b}{\Phi(x,y(x),y'(x))dx}
  \]
  demuéstrese la equivalencia de las dos formas siguientes de las
  ecuaciones de Euler-Lagrange.
  \begin{itemize}
  \item[$a)$] $\frac{\partial\Phi}{\partial y}-\frac{d}{dx}\frac{\partial\Phi}{\partial p} = 0 $
  \item[$b)$] $\frac{\partial\Phi}{\partial x} - \frac{d}{dx}(\Phi-y'\frac{\partial\Phi}{\partial p}) = 0$
  \end{itemize}
  \begin{proof}
    Comenzamos viendo el caso $a) \implies b)$. Del apartado $a)$ sabemos que
    \[
      \frac{\partial\Phi}{\partial y} = \frac{d}{dx}\frac{\partial\Phi}{\partial p}
    \]
    y del enunciado sabemos que $\Phi$ es de calse $\mathcal{C}^1$
    luego $\frac{\partial\Phi}{\partial p}$ es de $\mathcal{C}^1$.

    Cuando derivamos $\Phi(x,y(x),y'(x))$ obtenemos
    \begin{align*}
      \frac{d\Phi}{dx}(x,y(x),y'(x)) & = \frac{\partial\Phi}{\partial x}(x, y(x), y'(x)) \\
                          & + \frac{\partial\Phi}{\partial y}(x, y(x), y'(x))y'(x) \\
                          & + \frac{\partial\Phi}{\partial p}(x, y(x), y'(x))y''(x)
    \end{align*}
    Recordemos que $\frac{\partial\Phi}{\partial p}(x, y(x), y'(x)) = z(x)$ luego
    \begin{align*}
      \frac{d}{dx}(\Phi-y'\frac{\partial\Phi}{\partial p}) & = \Phi_x + \Phi_yy' + \Phi_py'' - y''z -y'z'
    \end{align*}
    Tenemos que el 3º y 4º termino son opuestos, al igual que el 2º y 5º, luego obtenemos
        \begin{align*}
      \frac{d}{dx}(\Phi-y'\frac{\partial\Phi}{\partial p}) = \Phi_x
        \end{align*}

        que es lo que queríamos.

        $b)\implies a)$
        
        Todos los pasos que hemos dado son reversibles pero
        necesitamos ver que
        $\frac{\partial\Phi}{\partial p}(x, y(x), y'(x))$ es $\mathcal{C}^1$.

        Llamando $H(x) = \Phi-y'(x)z(x)$, por hipótesis tenemos que
        $H$ es derivable. Despejando tenemos que

        \[
          z = \frac{\Phi-H}{y'}
        \]

        luego $z$ es derivable por ser cociente de derivables ($y'\neq 0$).

        \begin{align*}
          \Phi_x & = \Phi_x + \Phi_yy' + \Phi_py'' - y''z -y'z' \\
                 & \implies y'(\Phi_y-z') = 0
        \end{align*}

        e $y' \neq 0$ por hipótesis, luego $\Phi_y - z' = 0$.
  \end{proof}
\end{ejercicio}
\begin{ejercicio}
Obténgase la forma que adopta la ecuación de Euler-Lagrange en los siguientes casos particulares:

\begin{enumerate}[a)]
\item $\Phi$ sólo depende de $y'$.
    \begin{proof}
      Usando el apartado a) del ejercicio anterior y sustituyendo $\frac{\partial\Phi}{\partial y}=0$ obtenemos
      \[
        \frac{d}{dx}\Phi_p = 0
      \]

      De donde deducimos que $\Phi_p(y'(x))$ es una constante.
    \end{proof}
\item $\Phi$ no depende de $y$.
\item $\Phi$ no depende explícitamente de $x$.
\item $\Phi=G(x,y)\sqrt{1+y'^2}$.
\end{enumerate}
\end{ejercicio}

\begin{ejercicio}
Aplíquense los resultados anteriores a los ejemplos siguientes:
\begin{enumerate}[a)]
\item $\mathcal{F}(y(x))=\integral{}{}{y(2x-y)dx}$, $y(0)=0$, $y(\pi/2)=\pi/2$.
\item $\mathcal{F}[y(x)] =\integral{}{}{y(2x-y)dx}, \quad y(0) = 0, \quad y(\pi/2) = \pi/2$

    Definimos $F(x,y,p) = y(2x-y)$. Se cumple

    \begin{align*}
      F_p &= 0\\
      z(x) &= 0\\
      F_y(x,y(x)) &= 2y(x)-2x
    \end{align*}

    De aquí obtenemos $2x-2y = 0 \implies y(x) = x$. Ahora tenemos que
    comprobar que la solución cumple las condiciones de contorno.

    \[
      y(0) = 0, \quad y(\pi/2) = \pi/2
    \]
    
    Luego se cumplen ambas condiciones.
\item $\mathcal{F}(y(x))=\integral{}{}{(y^2+2xyy')dx}$, $y(a)=A$, $y(b)=B$.
\item $\mathcal{F}(y(x))=\integral{}{}{y'(1+x^2y')dx}$, $y(1)=3$, $y(2)=5$.
\end{enumerate}
\end{ejercicio}

