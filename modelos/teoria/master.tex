\documentclass[12pt]{report}
 
%%%%%%%%%%%%%%%%%%%%%%%%%%%%%%%%%%%%%%%%%%%%%%%%%%%%%%%%%%%%%%%%%%%%%%%%%%%%%%%%%%%%%%%
\usepackage[margin=1in]{geometry} 
\usepackage{amsmath,amsthm,amssymb}
\usepackage[utf8]{inputenc}
\usepackage{amsmath}
\usepackage[shortlabels]{enumitem}
\usepackage{mathtools}
\usepackage{amsfonts}
\usepackage{float}
\usepackage{epigraph}
\usepackage{lipsum}
\usepackage{parskip}
\usepackage[spanish]{babel}
\usepackage{tikz}
\usetikzlibrary{babel}
\usepackage{csquotes}
\usepackage{xcolor}
\usepackage[framemethod=tikz,xcolor=true]{mdframed}
\usepackage[new]{old-arrows}
%%%%%%%%%%%%%%%%%%%%%%%%%%%%%%%%%%%%%%%%%%%%%%%%%%%%%%%%%%%%%%%%%%%%%%%%%%%%%%%%%%%%%%%

%%%%%%%%%%%%%%%%%%%%%%%%%%%%%%%%%%%%%%%%%%%%%%%%%%%%%%%%%%%%%%%%%%%%%%%%%%%%%%%%%%%%%%%
% mis comandos
\usepackage{personalcommands}
\newtheorem{theorem}{Teorema}[chapter]
\newtheorem{lemma}[theorem]{Lema}
\newtheorem{prop}[theorem]{Proposición}
\newtheorem{coro}[theorem]{Corolario}
\newtheorem{conj}[theorem]{Conjetura}
\newtheorem{ejercicio}{Ejercicio}[section]
\newtheorem*{ejercicio*}{Ejercicio}
\theoremstyle{definition}
\newtheorem{definition}[theorem]{Definición}
\newtheorem{example}[theorem]{Ejemplo}
\theoremstyle{remark}
\newtheorem{remark}[theorem]{Nota}
\newtheorem{notacion}[theorem]{Notación}
\newcommand{\continuas}[1][]{\mathcal{C}^{ #1 }[a,b]}
\newcommand{\continuasabierto}[1][]{\mathcal{C}^{ #1 }(a,b)}
\newcommand{\soportecompacto}{\mathcal{D}(a,b)}
\newcommand{\xcero}{(a,b)}
\newcommand{\xcerocerrado}{[a,b]}
\newcommand{\fvariaciones}{F(x,y(x),y'(x))}
\newcommand*\circled[1]{\tikz[baseline=(char.base)]{\node[shape=circle,draw,inner sep=1pt] (char) {#1};}}
%%%%%%%%%%%%%%%%%%%%%%%%%%%%%%%%%%%%%%%%%%%%%%%%%%%%%%%%%%%%%%%%%%%%%%%%%%%%%%%%%%%%%%%
            
%%%%%%%%%%%%%%%%%%%%%%%%%%%%%%%%%%%%%%%%%%%%%%%%%%%%%%%%%%%%%%%%%%%%%%%%%%%%%%%%%%%%%%%
% Titulos
\makeatletter
\@ifundefined{@chapapp}{\def\@chapapp{\chaptername}}{}
\makeatother
\usepackage[Lenny]{fncychap}
\ChTitleVar{\Huge\bfseries}
\setcounter{chapter}{0}
%%%%%%%%%%%%%%%%%%%%%%%%%%%%%%%%%%%%%%%%%%%%%%%%%%%%%%%%%%%%%%%%%%%%%%%%%%%%%%%%%%%%%%%

\begin{document}

\tableofcontents

%\chapter{Cálculo de variaciones}

\section{Herramientas previas y repaso}

Necesitaremos recordar algunas nociones y teoremas básicos sobre derivabilidad:

\begin{theorem}[derivada de una integral respecto de un parámetro]
\label{derivadaparametro}
Sea $X\subset\R^n$ y $(X,\mathcal{A},\mu)$ un espacio medible, $I$ un intervalo cerrado y sea $\funcion{f}{I\times X}{\R}$ tal que:

\begin{enumerate}[(a)]
\item $\forall t \in I$, $x\mapsto f(t,x)$ es integrable.
\item $\forall x\in X$, la función $t\mapsto f(t,x)$ es derivable en $t\in I$.
\item Existe $\funcion{g}{X}{\R}$ integrable tal que
\[
\left|\derivada{f}{t}(t,x)\right|\leq |g(x)| \espacio \forall x\in X\; \forall t \in I
\]
Entonces la función $F(t)=\int_Xf(t,x)dx$ es derivable en $t_0$ y la derivada es 
\[
F'(t_0)=\displaystyle\int_X\derivada{f}{t}(t_0,x)dx
\]
\end{enumerate}

\end{theorem}

TODO: añadir resto de resultados usados de otras asignaturas que no recordábamos

\section{Problema general del cálculo de variaciones}

Primeramente definimos un tipo de funciones llamadas \textit{funciones test} o \textit{funciones de la clase de Schwartz}, junto con algunos resultados que usaremos bastante para trabajar con ellas.

\begin{definition}
\label{funcionestest}
Dado $I$ intervalo, se llama \textit{espacio de funciones test} al conjunto:
\[
\mathcal{D} = \mathcal{D}(a,b) = \{\phi\in C^{\infty}(a,b): \; \exists J\subset(a,b) \text{ compacto: } \phi(x)=0 \text{ si } x \notin J\}
\]
\end{definition}

\begin{lemma}
\label{existenciaphi}
Dado $x_0\in(a,b)$ y $\varepsilon>0$ tal que $[x_0-\varepsilon, x_0+\varepsilon]\subset(a,b)$, existe $\phi\in\mathcal{D}(a,b)$ tal que $\phi(x)>0$ si $x\in(x_0-\varepsilon,x_0+\varepsilon)$ y $\phi(x)=0$ en otro caso.
\end{lemma}

\begin{proof}
La demostración la vamos a hacer por construcción. Sea $\funcion{g}{\R}{\R}$ definida por:
\[
g(x)=\left\{
\begin{array}{cc}
e^{1/x} & x<0 \\
0 & x\geq 0
\end{array}
\right.
\]
Tenemos que $\limite{x}{0^-}{g(x)}=0$, luego $g$ es continua. Veamos que de hecho $g\in C^{\infty}$. Su derivada es:
\[
g'(x)=\left\{
\begin{array}{cc}
e^{1/x}\left(-\frac{1}{x^2}\right) & x<0 \\
0 & x> 0
\end{array}
\right.
\]
Mediante un proceso iterativo llegamos a:
\[
g^{n)}(x)=\left\{
\begin{array}{cc}
e^{1/x}\frac{R(x)}{x^{2n}} & x<0 \\
0 & x> 0
\end{array}
\right.
\]
donde $R(x)$ es un cierto polinomio que no nos interesa calcular. Queremos ver que 
\[\limite{x}{0^-}{g^{n)}(x)}=0,\]
con lo cual, $g\in C^n$. Haciendo el cambio $y=-\frac{1}{x}$, el anterior límite equivale a 
\[
\limite{y}{+\infty}{\frac{y^{2n}}{e^y}}=0 \Rightarrow g\in C^{n} \;\forall n \in\N \Rightarrow g\in C^{\infty}
\]
Con lo anterior, solo nos queda definir la función buscada de forma que sea una función test, es decir, la definimos como:
\[
\phi(x)=g(x-(x_0+\varepsilon))g((x_0-\varepsilon)-x) \espacio \varepsilon>0 
\]

\end{proof}


\begin{theorem}
\label{theorem:1.3}
Sea $f\in \continuas$ tal que
\[
\int f(x)\phi(x)dx=0 \hspace{1cm} \forall \phi \in \soportecompacto
\]
Entonces $f(x)=0 \;$  $\forall x\in [a,b]$.
\end{theorem}

\begin{proof}
Sea $\bar{x}\in \xcero$ y supongamos por reducción al absurdo que $f(\bar{x})\neq 0$. Podemos suponer $f(\bar{x})>0$. Aplicando el teorema de conservación del signo, obtenemos $\varepsilon>0$ tal que $f(x)>0 \text{ si } x \in (\bar{x}-\varepsilon,\bar{x}+\varepsilon)$.

Por el lema \ref{existenciaphi}, existe una función test $\phi$ tal que $\phi(x)>0$ si $x\in(\bar{x}-\varepsilon, \bar{x}+\varepsilon)$ y $0$ en otro caso. Luego:
\[
0=\int_{a}^{b}f(x)\phi(x)dx=\int_{\bar{x}-\varepsilon}^{\bar{x}+\varepsilon}f(x)\phi(x)dx>0 \Rightarrow f(\bar{x})=0
\]
Como $\bar{x}$ era arbitrario, tenemos que $f(\bar{x})=0 \; \forall \bar{x}\in\xcero$, y por la continuidad de $f$ podemos extenderlo a los extremos también, es decir, $f(a)=f(b)=0$.

\end{proof}

\subsection{Cálculo de extremales}

Sea $\Omega\subset\R^3$, definamos $F:\Omega \longrightarrow \R$ tal que $(x,y,p)\longmapsto F(x,y,p)$. Supongamos que $F\in C^1(\Omega)$  respecto de las dos últimas variables, es decir, existen $\derivada{F}{y}$ y $\derivada{F}{p}$, continuas. 

Usando la función anterior, podemos definir el siguiente funcional:

\begin{equation}\label{funcional}
L(y) = \int_{a}^{b}\fvariaciones dx 
\end{equation}


Nuestro objetivo en este apartado será encontrar \textit{extremales} de ese funcional, es decir, máximos o mínimos.

\begin{notacion}
Normalmente, a las derivadas parciales las denotaremos por:
\[
\derivada{F}{x}=F_x
\]
\end{notacion}

Los \textit{extremales} los buscaremos entre los elementos de un conjunto de funciones cumpliendo ciertas propiedades:

\begin{definition}
\label{espaciofuncionesbuenas}

Sea $\Omega\subset\R^3$ y $F:\Omega\longrightarrow \R$ funcional en las condiciones anteriores. Definimos entonces el siguiente conjunto:

\begin{equation}\label{espaciofunciones}
D=\{y\in\continuasabierto\cap\continuas[1]: \text{ se cumplen (a),(b) y (c)}\}
\end{equation}

\begin{enumerate}[(a)]
\item $(x,y(x),y'(x))\in\Omega \espacio \forall x\in\xcero$
\item $y(a)=y_0$ e $y(b)=y_1$ (\textit{Condición de contorno})
\item $\displaystyle\int_{a}^b\Big|\fvariaciones\Big|dx<+\infty$ 
\end{enumerate}

\end{definition}

El siguiente teorema nos proporcionará una condición sobre las derivadas parciales de $F$, que nos ayudará a buscar \textit{extremales}. Para su demostración necesitaremos el siguiente lema:

\begin{lemma}
\label{lemmatecnico}
Sea $\{s_n\}\longrightarrow 0$ una sucesión de números reales y $\phi\in \soportecompacto$, existe $n_0\in\N$ tal que si $y\in D$, entonces $y+s_n\phi\in D \espacio \forall n\geq n_0$.
\end{lemma}
\begin{proof}

Sea $\phi\in\soportecompacto$ y $K=\supp\phi$.Tenemos que comprobar que $y+s_n\phi$ cumple las condiciones de la definición \ref{espaciofuncionesbuenas}.

\begin{enumerate}[(a)]
\item Razonemos por reducción al absurdo. Supongamos que hay infinitos valores de $n\in\mathbb{N}$ tales que $(x_n,y(x_n)+s_n\phi(x_n),y'(x_n)+s_n\phi'(x_n))\notin \Omega$ para algún $x_n\in(a,b)$. Tomando una parcial si es necesario, podemos suponer una sucesión $\{x_n\}$ tal que para cada $n\in\mathbb{N}$, $(x_n,y(x_n)+s_n\phi(x_n),y'(x_n)+s_n\phi'(x_n))\notin \Omega \Rightarrow x_n\in\supp\phi$, ya que si no estuviera, tendríamos que $\phi(x_n)=0$ y $(x_n,y(x_n),y'(x_n))\in \Omega$. Como el soporte es compacto, podemos suponer que $\{x_n\}\longrightarrow\bar{x}\in\supp\phi\in(a,b)$. Si tomamos límite:
\[
\limitemasinfinito{n}{(x_n,y(x_n)+s_n\phi(x_n),y'(x_n)+s_n\phi'(x_n))}=(\bar{x},y(\bar{x}),y'(\bar{x}))\notin\Omega 
\]
porque el complementario de $\Omega$ es cerrado y el límite se queda fuera, llegando así a un absurdo. Por tanto, debe existir un $n_0\in\mathbb{N}$ tal que $(x_n,y(x_n)+s_n\phi(x_n),y'(x_n)+s_n\phi'(x_n))\in \Omega\quad\forall n\geq n_0$.
\item Evidente, ya que $\phi(a)=\phi(b)=0$.
\item Tomando $n\geq n_0$,de (a) tenemos:
\[
\integral{a}{b}{\Big|F(x,y(x)+s_n\phi(x),y'(x)+s_n\phi'(x))\Big|dx}=\]
\[
=\integral{(a,b)\backslash K}{}{\Big|F(...)\Big|dx}+\integral{K}{}{\Big|F(...)\Big|dx} < +\infty
\]
El primer término es finito ya que, fuera de $K$, $\phi=0$, luego ese término coincide con $\integral{(a,b)\backslash K}{}{\Big|F(x,y(x),y'(x))\Big|dx}$, que
es finito porque $y\in D$.
El segundo término lo es porque ser la integal de una función continua en un compacto. 
\end{enumerate}
\end{proof}
Lo que nos asegura este lema es que podamos sumar una perturbación \textit{pequeña} a nuestro extremal sin \textit{salirnos} de $D$.

\begin{theorem}
\label{theorem:1.7}
Si $\bar{y}\in D$ es un extremal, entonces:
\[
\integral{a}{b}{F_y(x,\bar{y}(x), \bar{y}'(x))}\phi(x)dx+\integral{a}{b}{F_p(x,\bar{y}(x), \bar{y}'(x))}\phi'(x)dx=0 \espacio \forall \phi\in \soportecompacto
\]
A $\bar{y}$ se le suele llamar \textbf{función crítica}.
\end{theorem}

\begin{proof}
Sean $\bar{y}\in D$ extremal y $\phi\in\soportecompacto$. Definimos el funcional $g:\R\longrightarrow\R$ tal que $g(s)=L(\bar{y}+s\phi)$.

Por el lema anterior, existe $\varepsilon>0$ tal que $g$ está bien definida en $(-\varepsilon,\varepsilon)$.

Ahora queremos derivar $g$ respecto de $s$, pero necesitamos que esté definida en un intervalo cerrado (por el teorema \ref{derivadaparametro}). Para ello, tomamos un intervalo cerrado $J$ de forma que $\supp (\phi)\subset J\subset [a,b]$. 

Derivamos $g$ respecto de $s$:
\[
g'(s)=\left(
\integral{[a,b]}{}{F(x,\bar{y}+s\phi(x),\bar{y}'+s\phi'(x))dx}
\right)'=
\]
\[
=\left(
\integral{[a,b]\backslash J}{}{F(x,\bar{y}+s\phi(x),\bar{y}'+s\phi'(x))dx}+
\integral{J}{}{F(x,\bar{y}+s\phi(x),\bar{y}'+s\phi'(x))dx}
\right)'
\]
Usando que $\integral{[a,b]\backslash J}{}{F(x,\bar{y}+s\phi(x),\bar{y}'+s\phi'(x))dx}\phi(x)$ es una constante (por que $\bar{y}\in D$ y $\phi = 0$ en $[a,b]\backslash J$) cuya derivada es cero, obtenemos que:
\[
g'(s)= \integral{J}{}{\Big(F_y(x,\bar{y}+s\phi(x),\bar{y}'+s\phi'(x))\phi(x)+F_p(x,\bar{y}+s\phi(x),\bar{y}'+s\phi'(x))\phi'(x)\Big)dx}
\]
Como $\phi=0=\phi'$ fuera de $J$
\[
g'(s)= \integral{[a,b]}{}{\Big(F_y(x,\bar{y}+s\phi(x),\bar{y}'+s\phi'(x))\phi(x)+F_p(x,\bar{y}+s\phi(x),\bar{y}'+s\phi'(x))\phi'(x)\Big)dx}
\]
Si evaluamos ahora $g'$ en 0 y usamos que $\bar{y}$ es extremal, tenemos:
\[
g'(0)=\integral{a}{b}{\Big(F_y(x,\bar{y},\bar{y}')\phi+F_p(x,\bar{y},\bar{y}')\phi'\Big)dx}=0 
\]
\end{proof}

\subsection{Ecuación de Euler}

Usando el teorema anterior vamos a llegar a una ecuación diferencial de segundo orden que nos ayudará a resolver este problema. Definimos $Z(x)=F_p(x,y(x),y'(x))$, nuestro objetivo ahora es imponer condiciones suficientes para que $Z(x)\in C^1(a,b)$, para poder derivarla y obtener una ecuación diferencial en $y'$.

Para continuar necesitamos un lema previo:

\begin{lemma}
Sea $Z\in C^1(x_0,x_1)$, entonces:
\[
\integral{a}{b}{Z(x)\phi'(x)dx}=-\integral{a}{b}{Z'(x)\phi(x)dx} \espacio \forall \phi\in\soportecompacto
\]
\end{lemma}

\begin{proof}

Resolviendo la integal por partes tenemos:
\[
\left.\begin{array}{cc}
u=Z(x) & du=Z'(x)dx\\
dv=\phi'(x)dx & v=\phi
\end{array}\right\} \Rightarrow\integral{a}{b}{Z(x)\phi'(x)dx}=Z(x)\phi(x)\Big|_a^b-\integral{a}{b}{Z'(x)\phi(x)dx}=
\]
\[
=-\integral{a}{b}{Z'(x)\phi(x)dx}
\]

\end{proof}

Este lema nos permite \enquote{intercambiar la derivada de sitio}.
Usando ahora el Teorema \ref{theorem:1.7} (podemos usarlo porque $y$ es función crítica) y el lema anterior, tenemos:
\[
0=\integral{a}{b}{F_y(x,y, y')}\phi(x)dx+\integral{a}{b}{F_p(x,y,y')}\phi'(x)dx= 
\]
\[
=\integral{a}{b}{F_y(x,y, y')}\phi(x)dx+\integral{a}{b}{Z(x)}\phi'(x)dx=
\]
\[
=\integral{a}{b}{F_y(x,y, y')}\phi(x)dx-\integral{a}{b}{Z'(x)}\phi(x)dx=
\]
\[
=\integral{a}{b}{\Big(F_y(x,y, y')-Z'(x)\Big)}\phi(x)dx=0 \espacio \forall\phi\in\soportecompacto
\]
Y usando ahora el Teorema \ref{theorem:1.3} nos queda:
\[
F_y(x,y,y')-Z'(x)=0 \espacio \forall x \in\xcero
\]
Que denoteramos por:
\[
\frac{d}{dx}F_p-F_y(x,y,y')=0 \espacio \textbf{(Ecuación de Euler)}
\]
Las condiciones sobre $F$ se pueden rebajar con el siguiente teorema:

\begin{theorem} 
\label{theorem:12}
Si $F\in C^1_{yp}$, $y'\in C^1$, función crítica, entonces:
\[
Z(x)=F_p(x,y(x),y'(x))\in C^1
\]
\[
Z'(x)=F_y(x,y(x),y'(x))
\]
\end{theorem}

Ya tenemos la ecuación que queremos resolver. La demostración del teorema es consecuencia de los anteriores resultados.

\begin{definition}
Una función $\phi\in\soportecompacto$ admite primitiva si existe otra función $\Psi\in\soportecompacto$ tal que $\Phi'=\phi$. 
\end{definition}

\begin{lemma}
\label{lemma:13}
Sea $\phi\in\mathcal{D}(a,b)$, entonces:
\[
\phi \text{ admite primitiva } \Longleftrightarrow \integral{a}{b}{\phi(x)dx}=0
\]
\end{lemma}

\begin{proof}
\hfill\\
$(\Rightarrow)$ Por hipótesis, supongamos que existe $\Psi$ tal que $\phi=\Psi'$. Ahora solo tenemos que integrarla y usar que $\Psi$ vale 0 en los extremos por ser una función de soporte compacto:
\[
\integral{a}{b}{\phi(s)ds}=\integral{a}{b}{\Psi'(s)ds}=\Psi\Big|_a^b=0
\]
$(\Leftarrow)$  Si $\integral{a}{b}{\phi(s)ds}=0$, entonces tenemos la siguiente situación: $\supp\phi\subset[a',b']$ tal que $a<a'\leq b'<b$.

\begin{center}
\includegraphics[scale=0.4]{./img/testfuncion.png}
\end{center}

Definimos $\Psi(x)=\integral{a'}{x}{\phi(s)ds}$ y tenemos $\Psi'=\phi$. Falta probar que $\Psi\in\mathcal{D}(a,b)$.
Si $x\leq a'$, claramente $\Psi(x)=0$ \big($\phi(x)=0\quad\forall x\in(a,a')$\big). Si $x\geq b'$,
\[
\int_{a'}^x\phi(s)ds=\int_{a'}^{b'}\phi(s)ds=\int_{a}^{b}\phi(s)ds=0
\]
Por tanto $\supp\Psi\subset[a',b']$.

\end{proof}

\begin{lemma}
\label{lemma:14}
Sea $f\in C(a,b)$ tal que $\integral{a}{b}{f(x)\phi'(x)dx=0}\espacio\forall\phi\in\mathcal{D}(a,b)\Longrightarrow f \text{ es constante.}$
\end{lemma}

\begin{proof}
Sea $x_0\in(a,b)$, tomamos una función $\phi_{x_0}\in \soportecompacto$ cumpliendo la tesis del lema \ref{existenciaphi}. Podemos tomarla de forma que $\integral{a}{b}{\phi_{x_0}(s)ds}=1$ (multiplicándola por cierta constante). 

Definimos $\Psi(x)=\integral{a}{x}{\phi_{x_0}(s)ds}$ y tomamos $c$ de forma que:
\[
\integral{a}{b}{f(s)\Psi'(s)}=c\integral{a}{b}{\Psi'(s)ds}=c\Psi\Big|_a^b=c
\]
Tomamos $\phi\in\soportecompacto$ y veamos que la función $\phi-\lambda\Psi'$ tiene primitiva para cierto $\lambda\in\R$. Supongamos que tiene media 0, y despejemos $\lambda$:
\[
\integral{a}{b}{\phi(s)-\lambda\Psi'(s)}=0 \Longrightarrow \lambda=\integral{a}{b}{\phi(s)ds}
\]
Haciéndo esa elección de $\lambda$, $\phi-\lambda\Psi'$ tiene primitiva por el lema \ref{lemma:13}.
\[
0=\integral{a}{b}{f(s)(\phi(s)-\lambda\Psi'(s))ds}=\integral{a}{b}{f(s)\phi(s)ds}-\lambda\integral{a}{b}{f(s)\Psi'(s)ds}=
\]
\[
=\integral{a}{b}{f(s)\phi(s)ds}-c\integral{a}{b}{\phi(s)ds}=\integral{a}{b}{(f(s)-c)\phi(s)ds} \Rightarrow f(s)-c=0 \Rightarrow f\equiv c
\]
\end{proof}

\begin{lemma}
Sean $f$,$g$ funciones continuas en $(a,b)$, entonces es equivalente:
\[
\integral{a}{b}{g\phi}+\integral{a}{b}{f\phi'}=0 \espacio \forall\phi\in\soportecompacto \Longleftrightarrow f\in C^1(a,b), g=f'
\]
\end{lemma}

\begin{proof}

($\Rightarrow$)Tomamos $x_0\in(a,b)$ y definimos $\tilde{f}(x)=\integral{a}{x}{g(s)ds}\Rightarrow \tilde{f}\in C^1$.
\[
\integral{a}{b}{\tilde{f}(s)\phi'(s)+g(s)\phi(s)ds}=\integral{a}{b}{\tilde{f}(s)\phi'(s)+\tilde{f}'(s)\phi(s)ds}=\integral{a}{b}{(\tilde{f}\phi)}=\tilde{f}\phi\Big|_a^b=0
\]
Restando la expresión de la hipótesis menos la anterior, obtenemos:
\[
0=\integral{a}{b}{f(s)\phi'(s)ds}-\integral{a}{b}{\tilde{f}(s)\phi'(s)ds}=\integral{a}{b}{(f-\tilde{f})(s)\phi'(s)ds}
\]
Y usnado el lema \ref{lemma:14}, tenemos que $f-\tilde{f}$ es constante, luego $f\in C^1$ ya que $\tilde{f}$ también pertenece a $C^1$.

($\Leftarrow$) 
\[
0=f\phi(x)\Big|_a^b=\integral{a}{b}{(f(x)\phi(x))'dx}=\integral{a}{b}{f(x)\phi'(x)dx+f'(x)\phi(x)dx}
\]
\end{proof}

\section{Problema de Braquistocrona}

Este problema se centra principalmente en averiguar que forma (curva) tenemos que darle a un tobogán para que este sea el más rápido.

A esa curva la vamos a denotar por $Y(t)$ y vamos a suponer que  pertenece a $C^1(0,L)$, es decir, que no tenga picos. Además, vamos a suponer que el tobogán tiene altura máximo 1, y mínima 0, es decir, $Y(0)=1$ e $Y(L)=0$. También necesitamos que el tobogán tenga sentido, es decir, que no tenga subidas ni bajadas muy bruscas, luego necesitamos imponer $Y(x)<1 \espacio \forall x \in (0,L)$.

Recordemos primero algunas nociones de física. Vamos a denotar por $(x(t),y(t))$ a la posición de una persona en el tobogán en el instante $t\in[0,L]$, por $m$ a su masa y por $g$ a la gravedad. Recordemos que la expresión de la energía es $mgy(t)$, la de la velocidad es $v(t)=\sqrt{x'(t)^2+y'(t)^2}$ y la de la energía cinética es $\frac{mv(t)^2}{2}$.

Si hacemos el tobogán suficientemente suave, por el teorema de conservación de la energía, se cumple:

\begin{equation}
\label{equationff}
mgy(t)+\frac{m}{2}(x'(t)^2+y'(t)^2)\equiv cte
\end{equation}
En el primer momento nos dejamos caer, luego en $t=0$, $(\ref{equationff})=mg$

Tenemos entonces $y(t)=Y(x(t)) \Rightarrow y'(t)=Y'(x(t))x'(t)$, sustituyendo en (\ref{equationff}):
\[
mgY(x(t))+\frac{m}{2}x'(t)^2+\frac{m}{2}(Y'(x(t))x'(t))^2=mg
\]
\[
x'(t)^2\left(1+Y'(x(t))^2\right)=2g\left(1-Y(x(t))\right)\Rightarrow x'(t)=\sqrt{2g}\sqrt{\frac{1-Y(x(t))}{1+Y'(x(t))^2}}
\]
Estamos buscando el tiempo de llegada, $T$, ¿cómo lo hacemos? Aplicamos un truco típico de ecuaciones diferenciales:
\[
T = \integral{0}{T}{dt}=\integral{0}{T}{\frac{x'(t)}{x'(t)}dt}=
\frac{1}{\sqrt{2g}}\integral{0}{T}{\frac{\sqrt{1+Y'(x(t))^2}}{\sqrt{1-Y(x(t))}}x'(t)dt}
\]
Que tras el cambio de variable $x=x(t)$ nos queda:
\[
T=\frac{1}{\sqrt{2g}}\integral{0}{L}{\frac{\sqrt{1+Y'(x)^2}}{\sqrt{1-Y(x)}}dx}
\]
Definimos ahora el conjunto de funciones donde vamos a buscar nuestro mínimo:
\[
D=\{y\in C^1(0,L),y(0)=1,y(L)=0, y'(x)<1 \espacio \forall x \in(0,L)\}
\]
Y definimos nuestro funcional:
\[
L(y)=\integral{0}{L}{\frac{\sqrt{1+Y'(x)^2}}{\sqrt{1-Y(x)}}dx}
\]
Usando la notación del principio, tenemos una función $F$ tal que $F(x,y,p)=\sqrt{\frac{1+p^2}{1-y}}$

Usando el Teorema \ref{theorem:12} podemos definir:
\[
Z(x)=\derivada{F}{p}(x,y,y')=\frac{1}{\sqrt{1-y(x)}}\frac{y'(x)}{\sqrt{1+y'(x)^2}} \in C^1(0,T)
\]
Ahora, en lugar de despejar $y'(x)$, vamos a ver que $y\in C^2$. Definiendo $\Psi(y')=\frac{y'}{\sqrt{1+y'^2}}=Z(x)\sqrt{1+y'^2}$. Como $\Psi(s)=\frac{s}{\sqrt{1+s^2}}$ tiene inversa de clase 1, tenemos:
\[
y'(x)=\Psi^{-1}(Z(x)\sqrt{1-y(x)})\Rightarrow y\in C^2
\]
En general, tenemos que si $F$ no depende de $x$, podemos hacer:
\[D(x)=F(y(x),y'(x))-Z(x)y'(x)\Rightarrow D'(x)=
\]
\[=\derivada{F}{y}(y(x),y'(x))y'(x)+\derivada{F}{p}(y(X),y'(x))y''(x)-Z'(x)y'(x)-Z(x)y''(x)=\]
\[
=y'(Z'(x)-Z'(x))=0
\]
Es decir, para alguna constante $C\in\R$, tenemos que:
\[
F(y(x),y'(x))-Z(x)y'=C
\]
En nuestro caso particular, nos quedaría:
\[
\frac{\sqrt{1+y'(x)^2}}{1-y(x)}-\frac{1}{\sqrt{1-y(x)}}\frac{y'(x)^2}{\sqrt{1+y'(x)^2}}=0 \Rightarrow \frac{1}{\sqrt{1-y(x)}}\frac{1}{\sqrt{1+y'(x)^2}}=C
\]
Esa ecuación diferencial, es de variables separadas, deberíamos resolverla con las condiciones iniciales $y(0)=1,y(L)=0$, pero la solución es trascendente, es decir, no tiene expresión explícita, es un cicloide.

\begin{figure}[!ht]
   \center
  \includegraphics[scale=0.5]{img/cicloide.png}
  \caption{Ejemplo de cicloide}
\end{figure}

\section{Relación de ejercicios}

\begin{ejercicio}
Encontrar la curva en la que el siguiente funcional podría alcanzar su extremo:
\[
I(y)=\integral{1}{2}{\Big(y'(x)^2-2xy(x)\Big)dx}
\]
con condiciones de contorno $y(1)=0,y(2)=-1$.
\end{ejercicio}
\textbf{Solución:}
Definimos $F(x,y,p)=p^2-2xy$ y obtenemos:
\[
\derivada{F}{p}=2p, \espacio \derivada{F}{y}=-2x \Rightarrow Z(x)=2y'(x), \espacio Z'(x)=-2x
\]
Resolviendo la ultima ecuación, obteniendo:
\[
Z(x)=-x^2+C \Rightarrow 2y'(x)=-x^2+C \Rightarrow y'(x)=\frac{-x^2+C}{2}\Rightarrow y(x)=-\frac{x^3}{6}+\frac{c}{2}x+D
\]
Usando las condiciones de contorno, podemos calcular el valor de $C$ y $D$, obteniendo:
\[
y(x)=\frac{x-x^3}{6}
\]
\begin{ejercicio}
Encuentra las curvas que unen $(1,3)$ con $(2,5)$, que puedan ser extremos del funcional:
\[
I(y)=\integral{1}{2}{\Big(y'(x)+x^2y'(x)^2\Big)dx}
\]
\end{ejercicio}

\textbf{Solución} $y(x)=-\frac{4}{x}+7$

\begin{ejercicio}
  Sean $\Phi\in \mathcal{C}^1(\mathbb{R}^3), y(x)\in \mathcal{C}^2\xcero, y'(x) \neq 0$. Dado el funcional
  \[
    F[y] = \int_{a}^{b}{\Phi(x,y(x),y'(x))dx}
  \]
  demuéstrese la equivalencia de las dos formas siguientes de las
  ecuaciones de Euler-Lagrange.
  \begin{itemize}
  \item[$a)$] $\frac{\partial\Phi}{\partial y}-\frac{d}{dx}\frac{\partial\Phi}{\partial p} = 0 $
  \item[$b)$] $\frac{\partial\Phi}{\partial x} - \frac{d}{dx}(\Phi-y'\frac{\partial\Phi}{\partial p}) = 0$
  \end{itemize}
  \begin{proof}
    Comenzamos viendo el caso $a) \implies b)$. Del apartado $a)$ sabemos que
    \[
      \frac{\partial\Phi}{\partial y} = \frac{d}{dx}\frac{\partial\Phi}{\partial p}
    \]
    y del enunciado sabemos que $\Phi$ es de calse $\mathcal{C}^1$
    luego $\frac{\partial\Phi}{\partial p}$ es de $\mathcal{C}^1$.

    Cuando derivamos $\Phi(x,y(x),y'(x))$ obtenemos
    \begin{align*}
      \frac{d\Phi}{dx}(x,y(x),y'(x)) & = \frac{\partial\Phi}{\partial x}(x, y(x), y'(x)) \\
                          & + \frac{\partial\Phi}{\partial y}(x, y(x), y'(x))y'(x) \\
                          & + \frac{\partial\Phi}{\partial p}(x, y(x), y'(x))y''(x)
    \end{align*}
    Recordemos que $\frac{\partial\Phi}{\partial p}(x, y(x), y'(x)) = z(x)$ luego
    \begin{align*}
      \frac{d}{dx}(\Phi-y'\frac{\partial\Phi}{\partial p}) & = \Phi_x + \Phi_yy' + \Phi_py'' - y''z -y'z'
    \end{align*}
    Tenemos que el 3º y 4º termino son opuestos, al igual que el 2º y 5º, luego obtenemos
        \begin{align*}
      \frac{d}{dx}(\Phi-y'\frac{\partial\Phi}{\partial p}) = \Phi_x
        \end{align*}

        que es lo que queríamos.

        $b)\implies a)$
        
        Todos los pasos que hemos dado son reversibles pero
        necesitamos ver que
        $\frac{\partial\Phi}{\partial p}(x, y(x), y'(x))$ es $\mathcal{C}^1$.

        Llamando $H(x) = \Phi-y'(x)z(x)$, por hipótesis tenemos que
        $H$ es derivable. Despejando tenemos que

        \[
          z = \frac{\Phi-H}{y'}
        \]

        luego $z$ es derivable por ser cociente de derivables ($y'\neq 0$).

        \begin{align*}
          \Phi_x & = \Phi_x + \Phi_yy' + \Phi_py'' - y''z -y'z' \\
                 & \implies y'(\Phi_y-z') = 0
        \end{align*}

        e $y' \neq 0$ por hipótesis, luego $\Phi_y - z' = 0$.
  \end{proof}
\end{ejercicio}
\begin{ejercicio}
Obténgase la forma que adopta la ecuación de Euler-Lagrange en los siguientes casos particulares:

\begin{enumerate}[a)]
\item $\Phi$ sólo depende de $y'$.
    \begin{proof}
      Usando el apartado a) del ejercicio anterior y sustituyendo $\frac{\partial\Phi}{\partial y}=0$ obtenemos
      \[
        \frac{d}{dx}\Phi_p = 0
      \]

      De donde deducimos que $\Phi_p(y'(x))$ es una constante.
    \end{proof}
\item $\Phi$ no depende de $y$.
\item $\Phi$ no depende explícitamente de $x$.
\item $\Phi=G(x,y)\sqrt{1+y'^2}$.
\end{enumerate}
\end{ejercicio}

\begin{ejercicio}
Aplíquense los resultados anteriores a los ejemplos siguientes:
\begin{enumerate}[a)]
\item $\mathcal{F}(y(x))=\integral{}{}{y(2x-y)dx}$, $y(0)=0$, $y(\pi/2)=\pi/2$.
\item $\mathcal{F}[y(x)] =\integral{}{}{y(2x-y)dx}, \quad y(0) = 0, \quad y(\pi/2) = \pi/2$

    Definimos $F(x,y,p) = y(2x-y)$. Se cumple

    \begin{align*}
      F_p &= 0\\
      z(x) &= 0\\
      F_y(x,y(x)) &= 2y(x)-2x
    \end{align*}

    De aquí obtenemos $2x-2y = 0 \implies y(x) = x$. Ahora tenemos que
    comprobar que la solución cumple las condiciones de contorno.

    \[
      y(0) = 0, \quad y(\pi/2) = \pi/2
    \]
    
    Luego se cumplen ambas condiciones.
\item $\mathcal{F}(y(x))=\integral{}{}{(y^2+2xyy')dx}$, $y(a)=A$, $y(b)=B$.
\item $\mathcal{F}(y(x))=\integral{}{}{y'(1+x^2y')dx}$, $y(1)=3$, $y(2)=5$.
\end{enumerate}
\end{ejercicio}

 
%\section{Cálculo de variaciones}

\subsection{Herramientas previas y repaso}

Necesitaremos recordar algunas nociones y teoremas básicos sobre derivabilidad:

\begin{definition}
\label{funciondiferenciable}

Sea $\funcion{f}{X}{Y}$ con $X,Y$ espacios normados. Se dice que $f$ es \textbf{diferenciable} en un punto $a\in\overset{\circ}{A}$, si existe una aplicación lineal continua $\funcion{T}{X}{Y}$ verifiando:

\[
\limite{x}{a}{\frac{\norm{f(x)-f(a)-T(x-a)}}{\norm{x-a}}}
\]

$T$ se denota por $Df(a)$ y se llama \textit{diferencial de $f$}.
\end{definition}
%\chapter{Espacios de Sobolev}

A lo largo de los temas 1 y 2, hemos construido una serie de herramientas básicas en la resolución de ciertos modelos matemáticos. A continuación, vamos a enlazar los conceptos vistos anteriormente y desarrollar los primeros modelos de esta asignatura.

\section{Enlace}

Antes de definir los \textit{espacios de Sobolev}, recordemos algunas propiedades del espacio 
\[L^2(a,b)=\displaystyle\{\funcion{f}{(a,b)}{\R}\;:\; \integral{a}{b}{|f(x)|^2dx<+\infty}\displaystyle\}\]
La norma de este espacio venía dada por $\norm{f}_2=\sqrt{\integral{a}{b}{f(x)^2dx}}$ e inducía el siguiente producto escalar: $<f,g>=\integral{a}{b}{f(x)g(x)dx}$ (nótese que no usamos el conjugado de $g$ ya que estamos trabajando sobre $\R$, no sobre $\C$). Además, $L^2(a,b)\subset L^1(a,b)$ ya que 
\[
\integral{a}{b}{f(x)}=\integral{a}{b}{|f(x)|\cdot 1dx}\leq \left(\integral{a}{b}{f(x)^2dx}\right)^{1/2}\left(b-a\right)^{1/2}
\]
donde hemos usado la desigualdad de Schwartz para el último paso. Con estas propiedades en mente, podemos definir el primer espacio de Sobolev:
\[
\sobolev{1}=\conjunto{f\in\lebesgue{2}: f \text{ tiene derivada débil }f' \text{ y } f'\in\lebesgue{2}}
\]
Este espacio tiene la particularidad de que también cuenta con un producto escalar:
\[
<f,g>=\integral{a}{b}{f(x)g(x)dx}+\integral{a}{b}{f'(x)g'(x)dx}
\]
Por supuesto, $\sobolev{1}$ es un espacio de Hilbert con la norma $\norm{f}=\sqrt{\norm{f}^2_2+\norm{f'}_2^2}$. Es importante recordar que si $f\in\sobolev{1}$, no tiene por qué ser continua (como muestra \textit{The devil staircase}), pero podemos elegir un representante continuo en su clase de equivalencia, es decir, $\sobolev{1}\hookrightarrow C[a,b]$. De hecho:
\begin{prop}\label{inclusion continua}
La aplicación inclusión $\funcion{i}{\sobolev{1}}{C[a,b]}$ es continua.
\end{prop}
\begin{proof}
Pendiente.
\end{proof}
Además, si $f\in C^1[a,b]$, tiene derivada clásica, luego también tiene derivada débil, luego $f\in\sobolev{1}$. Resumiendo, hemos construido la siguiente cadena:
\[
C^1[a,b]\subset\sobolev{^1}\subset C[a,b]\subset \lebesgue{2}
\]
Definimos ahora el siguiente espacio de Sobolev:
\[
\sobolev{2}=\conjunto{f\in\lebesgue{2}: \exists f',f''\in\lebesgue{2}\text{ (débiles)}}
\]
Usando la propiedad \ref{derivadanesima}, tenemos que $\sobolev{2}\subset C^1[a,b]$ y repitiendo el argumento anterior $C^2[a,b]\subset \sobolev{2}$, es decir:
\[
C^2[a,b]\subset \sobolev{2}\subset C^1[a,b]
\]
Repitiendo este proceso iterativamente, podemos definir el $n$-ésimo espacio de Sobolev:
\[
\sobolev{n}=\conjunto{f\in\lebesgue{2}: \exists f',\dots,f^{n)}\in\lebesgue{2} \text{ (débiles)}}
\]
con la propiedad:
\[
C^{n}[a,b]\subset\sobolev{n}\subset C^{n-1}[a,b]\subset\sobolev{n-1}\subset\dots\subset\sobolev{1}\subset\lebesgue{2}
\]

¿Qué tienen de particular estos espacios? Que son de Hilbert, es decir, vamos a poder usar todas las herramientes que tenemos de Análisis Funcional para resolver algunos problemas como veremos en los siguientes apartados.

\begin{definition}
Diremos que $\funcion{F}{[a,b]\times\R^2}{\R}$ es una \textit{función de Carathéodory} si cumple:
\begin{enumerate}[(a)]
\item Para casi todo punto $x\in(a,b)$, $F(x,y,p)$ es continua en $(y,p)$.
\item Para casi todo punto $(y,p)\in\R^2$, la función $x\mapsto F(x,y,p)$ es medible.
\item Dado $K\subset[a,b]\times\R^2$ compacto, existe una función $m_k(x)\in\lebesgue{1}$ tal que $\valorabsoluto{F(x,y,p)}\leq m_k(x)$ $\forall(x,y,p)\in K$.
\end{enumerate}
\end{definition}

Para que la teoría que vamos a desarrollar a continuación tenga sentido, vamos a imponer de ahora en adelante que $F$, $F_y$ y $F_p$ sean de Carathéodory. Haciendo uso de la propiedad (c), podemos definir el funcional $\funcion{L}{\sobolev{1}}{\R}$:
\[
L(y)=\integral{a}{b}{F(x,y(x),y'(x))dx}
\]
que está bien definido ya que que $F$ es medible y está acotada por una función que es integrable. Tomando $\phi\in\soportecompacto$, podemos definir otra función $\funcion{g}{[a,b]}{\R}$ por $g(s)=L(y+s\phi)$, derivarla respecto a $s$ y evaluarla en 0 (a esta expresión la llamamos \textit{derivada de y a lo largo de $\phi$}):
\[
g'(0)=DL_y(\phi)=\integral{a}{b}{F_y(x,y(x),y'(x)dx}+\integral{a}{b}{F_p(x,y(x),y'(x)dx}
\]
\newpage

\section{Modelos de cuerdas}

Procedemos a la introducción del \textit{modelo de cuerdas}. Comenzaremos realizando el planteamiento más simple, densidad constante y extremos fijos. Posteriormente, eliminaremos la primera hipótesis y resolveremos dos casos: en el primero partimos de una función de densidad dada explícitamente. Por el contrario, en el segundo caso, lo resolveremos dada una función de densidad arbitraria. A continuación, supondremos que el puente se haya \textit{sujeto} por varias cuerdas elásticas. Finalmente, no supondremos fijos los extremos.

Aunque ya hemos desarrollado una cantidad de resultados considerable, todavía precisamos de ciertos resultados, mayoritariamente del \textit{Análisis Funcional}, que iremos introduciendo al mismo tiempo que el modelo. Como estamos trabajando sobre espacios de Sobolev, que son de Hilbert, podremos usarlos sin mucha complicación.

\begin{figure}[h]
   \center
  \includegraphics[scale=0.5]{img/cuerdanormal.png}
\end{figure}

\subsection{Extremos fijos y densidad constante}

Supongamos tener una \textit{cuerda elástica}, colgada entre dos puntos, 0 y 1, es decir, nuestra cuerda está representada por una función $\funcion{y}{[0,1]}{\R}$ en $\sobolev{1}$ tal que $y(0)=y(1)=1$. El objetivo del modelo será encontrar la \textit{cuerda} de mínima energía. En este caso, vamos a suponer que la \textit{densidad} de la cuerda es una constante $m\in\R^+$. 

Denotamos por $E_p$ a la energía potencial y por $E_e$ a la energía elástica:
\[
E_e=\frac{1}{2}\integral{0}{1}{y'(x)^2dx} \espacio\espacio E_p=\integral{0}{1}{my(x)dx}
\]
Minimizar la energía total de la cuerda, es lo mismo que minimizar el funcional:
\[
L(y)=\frac{1}{2}\integral{0}{1}{y'(x)^2dx}+\integral{0}{1}{my(x)dx}
\]
Usando la notación usual:
\[
L(y)=\integral{0}{1}{F(x,y,p)dx} \; \text{ donde } \;\; F(x,y,p)=\frac{1}{2}p^2+my
\]
Supongamos que $y\in\sobolev{1}$ es un punto crítico del funcional $L:\sobolev{1}\longrightarrow\R$ para poder usar la teoría de la \textit{ecuación de Euler}.
Si calculamos las derivadas parciales de $F(x,y,p)$:
\[
Z(x)=F_p(x,y,p)=p=y'(x) \espacio Z'(x)=F_y(x,y,p)=m
\]
vemos que $Z(x)$ tiene derivada débil ($Z'(x)$), como consecuencia, $y'(x)$ es continua (proposición \ref{representantecontinuo}). Pero además, $Z'(x)=(y'(x))'=y''(x)=m$, continua, por lo tanto $y\in C^2[0,1]$. En resumen, tenemos que resolver la siguiente EDO de segundo orden:
\[
\left\{
\begin{array}{rl}
y''(x) & = m \\
y(0) & = 0 \\
y(1) & = 0
\end{array}
\right.
\]
Si recordamos algo de ecuaciones diferenciales, nos damos cuenta de que las condiciones iniciales están \textit{mal planteadas}. Para poder resolver la ecuación, necesitamos dos conciones sobre el mismo punto: una en $y$ y otra en $y'$. Lo podemos solucionar usando el llamado \textit{método de tiro}, que consiste en darle un valor arbitrario a la condición que nos falta, resolver la ecuación y despejar el valor posteriormente. Asumiendo que $y'(0)=\alpha\in\R$, nos queda:
\[
\left\{
\begin{array}{rl}
y''(x) & = m \\
y'(0) & = \alpha \\
y(0) & = 0
\end{array}
\right.
\]
con solución $y(x)=\integral{0}{x}{\left(\alpha+msds\right)}=m\frac{x^2}{2}+\alpha x$. Usando $y(1)=0$, obtenemos que $\alpha=-\frac{m}{2}$. Por lo que la solución del modelo es:
\[
y(x)=\frac{m}{2}x(x-1) \;\; \forall x \in[0,1]
\] 

\subsection{Extremos fijos y densidad no constante}

El planteamiento es igual al anterior, pero suponemos que la densidad en lugar de ser constante es la función $\funcion{q}{[0,1]}{\R}$ dada por:
\[
q(x)=\left\{\begin{array}{cc}
1 & x \in[0,\frac{1}{2}] \\
\frac{1}{2} & x \in(\frac{1}{2}, 1] 
\end{array}
\right.
\]
Cabe resaltar que $q(x)$ no es continua (presenta un salto en $x=\frac{1}{2}$, pero no importa a la hora de resolverlo. En este caso, el funcional viene dado por:
\[
L(y)=\integral{0}{1}{\frac{1}{2}y'(x)^2+q(x)y(x)dx}
\]
con $F(x,y,p)=\frac{1}{2}p^2+q(x)y$, $F_p(x,y,p)=p$, $F_y(x,y,p)=q(x)$. La ecuación diferencial a resolver es (usando de nuevo el método de tiro):
\[
\left\{
\begin{array}{rl}
y''(x) & = q(x) \\
y(0)  = 0,y'(0) & = \alpha \\
y(1)= 0, y'(1) & = \beta
\end{array}
\right.
\]
Tenemos el problema de que $y\notin C^2[0,1]$. Intentemos arreglarlo:
\begin{itemize}
\item Si $x\in(0,\frac{1}{2}) \Rightarrow y''(x)=1 \Rightarrow y\in C^2(0,\frac{1}{2}) \Rightarrow y'(x)=\integral{0}{x}{1dx}+y'(0)=x+\alpha$\\
 $\Rightarrow y(x)=\integral{0}{x}{x+\alpha dx}=\frac{x^2}{2}+x\alpha$.
\item Si $x\in(\frac{1}{2},1) \Rightarrow y''(x)=1/2 \Rightarrow y\in C^2(\frac{1}{2},1) \Rightarrow y'(x)=y'(1)-\integral{x}{1}{\frac{1}{2}dx}=\beta+\frac{x}{2}-\frac{1}{2}$\\
$\Rightarrow y(1)-y(x)=\integral{x}{1}{\frac{x}{2}-\frac{1}{2}+\beta dx}\Rightarrow y(x)=\frac{x^2}{2}-\frac{x}{2}+\beta(x-1)+\frac{1}{4}$
\end{itemize}
Quedando:
\begin{equation}\label{y}
y(x)=\left\{
\begin{array}{cc}
\frac{x^2}{2}+x\alpha & x\in[0,1/2]) \\
\frac{x^2}{2}-\frac{x}{2}+\beta(x-1)+\frac{1}{4} & x\in(1/2,1]
\end{array}
\right.
\end{equation}

\begin{equation}\label{yprima}
y'(x)=\left\{
\begin{array}{cc}
x+\alpha & x\in[0,1/2]) \\
\frac{x}{2}-\frac{1}{2}+\beta & x\in(1/2,1]
\end{array}
\right.
\end{equation}

Ahora tenemos que resolver el sistema de $\alpha$ y $\beta$. Obtendremos dos ecuaciones de imponer que $y$ e $y'$ sean continuas en $\frac{1}{2}$, es decir, de que las dos partes evaluadas en ese punto coincidan. Por lo que a partir de \eqref{y} y \eqref{yprima} conseguimos,evaluando en $\frac{1}{2}$ e igualando:
\begin{equation}
\left\{
\begin{array}{cc}
\alpha+\beta & =0 \\
\alpha-\beta & =-\frac{3}{4}
\end{array}
\right.
\end{equation}
Resolviendo el sistema, $\alpha=-\frac{3}{8}$, $\beta=\frac{3}{8}$. Quedando finalmente la siguiente solución al modelo: (esta mal)
\begin{equation}\label{y}
y(x)=\left\{
\begin{array}{cc}
\frac{x^2}{2}-\frac{x}{4} & x\in[0,1/2]) \\
\frac{x^2}{2}-\frac{x}{2}+\frac{1}{4} & x\in(1/2,1]
\end{array}
\right.
\end{equation}

\textbf{ESTAS CUENTAS NO ESTAN ACTUALIZADAS, AUNQUE ESTAN MAL :D}

\begin{figure}[h]
   \center
  \includegraphics[scale=0.6]{img/puenteflotante.png}
\end{figure}

Que como vemos, floa, asi que algo hay mal :D.

\subsection{Extremos fijos y densidad arbitraria}

El planteamiento es igual al caso más simple, pero ahora la función de densidad, $q(x)$, la consideraremos en $C^\infty$. Sin la expresión de la función $q(x)$, no podemos resolver el problema de forma explícita, pero sí podemos asegurar la existencia de solución. El siguiente procedimiento lo repetiremos varias veces: definimos un espacio en donde buscaremos nuestra solución ($\sobolevcero[0,1]{1}$), plantearemos el problema (condición de extremal), definiremos un producto escalar y demostraremos que el espacio definido con ese producto escalar es de Hilbert, para usar el Teorema de Riesz-Fréchet.

Comencemos definiendo nuestro espacio de soluciones:
\[
\sobolevcero[a,b]{1}=\conjunto{y\in\sobolev{1}: y(a)=y(b)=0}
\]
Recordemos que en este espacio están las funciones de $\lebesgue{2}$, que tienen derivada débil. Por lo tanto, por la proposición \ref{representantecontinuo} podemos elegir una $y$ que sea continua.
\begin{prop}
El espacio vectorial $\sobolevcero{1}$, es cerrado.
\end{prop}   
\begin{proof}
Sea una secuencia de funciones convergentes en $\sobolevcero{1}$, $f_n\longrightarrow f$. Usando la proposición \ref{inclusion continua} y $\sobolevcero{1}\subset\sobolev{1}$, tenemos que $f_n(a)\longrightarrow f(a)$ y $f_n(b)\longrightarrow f(b)$. Luego $f_n(a)=f_n(b)=0 \;\forall n\in\N \Rightarrow f(a)=f(b)=0 \Rightarrow f\in\sobolevcero{1}$.\\
\end{proof}

Una vez definido nuestro nuevo espacio, pasamos a resolver el modelo. Como de costumbre, suponemos que $y\in\sobolevcero[0,1]{1}$ es extremal y usamos la teoría de Euler:
\[
L(y)=\frac{1}{2}\integral{0}{1}{y'(x)^2dx}+\integral{0}{1}{q(x)y(x)dx}
\]
Usando la notación usual:
\[
L(y)=\integral{0}{1}{F(x,y,p)dx} \; \text{ donde } \;\; F(x,y,p)=\frac{1}{2}p^2+q(x)y
\]
Calculando sus derivadas parciales
\[
F_p(x,y,p)=p \espacio F_y(x,y,p)=q(x)
\]
podemos calcular $\funcion{DL_y}{\sobolevcero[0,1]{1}}{\R}$ y usar la concidición de extremal de $y$ (teorema \ref{theorem:1.7}):
\[
DL_y(\phi)=\integral{0}{1}{y'(x)\phi'(x)dx}+\integral{0}{1}{q(x)\phi(x)dx}=0 \espacio \forall \phi\in\sobolevcero[0,1]{1}
\]
Si lo miramos como una derivada débil, vemos que si $Z(x)=y'(x) \Rightarrow Z'(x)=q(x) \Rightarrow y''(x)=q(x)$, en sentido débil.
\begin{definition}
\label{soluciondebil}
$y\in\sobolevcero{1}$ es \textit{solución débil} de $y''(x)=q(x) \; \forall x 
\in [a,b]$, si se verifica:
\[
\integral{0}{1}{y'(x)\phi'(x)dx}+\integral{0}{1}{q(x)\phi(x)dx}=0 \espacio \forall \phi\in\sobolevcero[0,1]{1}
\]
\end{definition}
Cabe resaltar, el hecho de que en la anterior definición, $y$ está en $\sobolevcero[0,1]{1}$ y estamos resolviendo una EDO de segundo orden, es decir, no sabemos nada sobre la segunda derivada de $y$ (solo de la primera).

Definimos un producto escalar en $\sobolevcero[a,b]{1}$ y veamos que es de Hilbert:
\begin{prop}
El espacio $\sobolevcero[a,b]{1}$ con el siguiente producto escalar:
\[
f,g\in\sobolevcero[a,b]{1}, \espacio <f,g>=\integral{a}{b}{f'(x)g'(x)dx}
\]
es un espacio de Hilbert.
\end{prop}
\begin{proof}
Sea $f_n\in\sobolevcero{1} \;\forall n\in\N$ una sucesión de Cauchy, entonces:
\[
\forall \varepsilon>0 \; \exists n_0: \; n,m\geq n_0 \; \norm{f_n-f_m}_{\sobolevcero{1}}<\varepsilon
\]
Recordemos que si tenemos un producto escalar definido, podemos expresar la norma de un elemento del espaico como la raíz cuadrada del producto escalar de el elemento consigo mismo, es decir:
\[
\norm{f}_{\sobolevcero{1}}=\sqrt{<f,f>}
\]
Usando esa propiedad:
\[
\varepsilon > \norm{f_n-f_m}_{\sobolevcero{1}}=\sqrt{<f_n-f_m,f_n-f_m>}=\sqrt{\integral{a}{b}{(f_n(x)-f_m(x))^2dx}}=\norm{f_n'-f_m'}_2
\]
Luego $f_n'$ converge a una cierta $g$ en $\lebesgue{2}$. Definiendo:
\[
\tilde{f_n}(x)=\integral{a}{b}{f_n'(s)ds} \Rightarrow \tilde{f_n}(x) \longrightarrow \integral{a}{x}{g(s)ds} \Rightarrow f(x) = \limitemasinfinito{n}{\tilde{f_n}(x)}
\]
Ya tenemos nuestro candidato a límite, comprobemos que pertenece a $\sobolevcero{1}$:
\begin{itemize}
\item $f(a)=0$ y $f(b)=\integral{a}{b}{g(s)ds}=f_n'(b)-f_n'(a)=0$.
\item $f$ tiene derivada débil (igual a $g$) por el teorema \ref{fundamentalcalculo}.
\end{itemize}
Luego $f\in\sobolevcero{1}$.
Por último, vemos que converge:
\[
\norm{f_n-f}_{\sobolevcero{1}}=\norm{f_n'-g}_2\longrightarrow 0
\]
\end{proof}

Una vez comprobado que el espacio es de Hilbert, solo nos falta recordar el Teorema de Riesz-Fréchet:
\begin{theorem}[Riesz-Fréchet]
\label{riesz-frechet}
Si $H$ es un espacio de Hilbert y $f\in H^*$, existe $y\in H$ tal que:
\[
f(x)=<x,y> \; \forall x\in H
\]
\end{theorem}

Sea ahora $\phi\in\sobolevcero{1}$, definimos $\funcion{R}{\sobolevcero{1}}{\R}$ por $R(\phi)=-\integral{a}{b}{q(x)\phi(x)dx}\in\left(\sobolevcero{1}\right)^*$. Por el teorema de Riesz-Fréchet, existe $y\in\sobolevcero{1}$ tal que $R(\phi)=<y,\phi>$.

Si desarrollamos la última expresión, nos damos cuenta de que es igual a la condición en la definición \ref{soluciondebil}, que es justo lo que buscábamos.

\subsection{Puente sujeto por cuerdas}

Al modelo del puente anterior le vamos a añadir unas cuerdas elásticas para soportarlo, cada una con una constante de elasticidad distinta, dado por $K(x)\geq 0$.
Si $K(x)=0$ en algún punto $x\in[0,1]$, significa que en ese punto no hay cuerda sujetando a la de abajo.
\begin{figure}[H]
   \center
  \includegraphics[scale=0.6]{img/puentecuerdas.png}
\end{figure}
En esta nueva versión del modelo tenemos que tener en cuentra otra energía más, la aportada por los cables, $E_c$:
\[
E_c=\frac{1}{2}\integral{0}{1}{K(x)y^2(x)dx}
\]
Con lo que el funcional quedaría:
\[
L(y)=\frac{1}{2}\integral{0}{1}{y'(x)^2dx}+\frac{1}{2}\integral{0}{1}{K(x)y(x)^2dx}+\integral{0}{1}{q(x)y(x)dx}
\]
Usando la notación usual:
\[
L(y)=\integral{0}{1}{F(x,y,p)dx} \; \text{ donde } \;\; F(x,y,p)=\frac{1}{2}p^2+\frac{1}{2}K(x)y^2+q(x)y
\]
Suponiendo que $y\in\sobolevcero[0,1]{1}$ es extremal, la condición de punto crítico es:
\[
DL_y(\phi)=\integral{0}{1}{y'(x)\phi'(x)dx}+\integral{0}{1}{\left(K(x)y(x)+q(x)\right)\phi(x)dx}=0 \;\; \forall \phi\in\sobolevcero[0,1]{1}
\]
Denotando $Z(x)=y'(x)$ y viendo la expresión anterior en sentido débil, tenemos que $Z'(x)=K(x)y(x)+q(x)$ es su derivada débil. Luego la EDO de orden 2 de este modelo es (que la obtenemos a partir de la ecuación de Euler):
\[
-y''(x)=+K(x)y(x)+q(x)=0
\]
Ahora, al igual que en el modelo anterior, vamos a definir un cierto producto escalar de forma que el teorema de Riesz-Frechét nos de la existencia de solución.
\begin{prop}
El espacio $\sobolevcero[a,b]{1}$ con el siguiente producto escalar:
\[
f,g\in\sobolevcero[a,b]{1}, \espacio <f,g>=\integral{a}{b}{f'(x)g'(x)dx}+\integral{a}{b}{K(x)f(x)g(x)dx}
\]
con $K\in\lebesgue{2}$, $K(x)\geq 0$ $\forall x\in[a,b]$, es un espacio de Hilbert.
\end{prop}
\begin{proof}

\end{proof}


\chapter{Teorema de Lax-Milgran}

Para dejar el puente libre, primero necesitamos ver nuestro funcional $L$ desde otro punto de vista. Para ello, vamos a ver una serie de definiciones y propiedades abstractas sobre los \textit{funcionales cuadráticos}, lo que nos va a permitir enunciar el teorema de Lax-Milgran, que nos dará la existencia de solución en esta última versión del modelo.

\begin{definition}\label{formabilineal}
Dados un cuerpo $K$ y un $K-$espacio vectorial $V$, una forma bilineal es una aplicación $\funcion{f}{V\times V}{K}$ que verifica:
\begin{enumerate}[(a)]
\item $f(u_1+u_2,v)=f(u_1,v)+f(u_2,v)$
\item $f(u, v_1+v_2)=f(u,v_1)+f(u,v_2)$
\item $f(au,v)=af(u,v)$
\item $f(u,av)=af(u,v)$
\end{enumerate}
para cualquier $a\in K$ y $u,v,u_1,u_2,v_1,v_2\in V$.

Una propiedad que necesitaremos es:
\[
f\left(\sum_ia_iu_i,\sum_jb_jv_h\right)=\sum_i\sum_ja_ib_jf(u_i,v_j)
\]

\end{definition}

\begin{definition}
Una forma bilineal $\funcion{f}{V\times V}{K}$ se dice simétrica si verifica:
\[
f(u,v)=f(v,u) \espacio \forall u,v\in V
\]
\end{definition}

\begin{definition}
Sea $H$ un espacio de Hilbert arbitrario, $\funcion{A}{H\times H}{\R}$ una forma bilineal, continua y simétrica y $\funcion{R}{H}{\R}$ una aplicación lineal y continua. Llamamos funcional cuadrático a la aplicación $\funcion{L}{H}{\R}$ dada por
\[
L(y)=\frac{1}{2}A(y,y)-R(y) \espacio\forall y\in H
\]
\end{definition}

\begin{definition}
Sea $\funcion{L}{H}{\R}$ una forma cuadrática. Diremos que es coerciva si existe una constante $\alpha\in\R^+$ tal que:
\[
A(u,u)\geq \alpha\norm{u}^2 \espacio \forall u\in H
\]
\end{definition}

\begin{prop}
Sea un espacio de Hilbert $H$ y $\funcion{L}{H}{\R}$ una forma cuadrátrica en ese espacio.
Entonces, la aplicación $\funcion{\norm{\cdot}_H}{H}{\R^+_0}$ definida por 
\[
\norm{u}_H=\sqrt{A(u,u)} \espacio \forall u\in H
\]
donde $\funcion{A}{H\times H}{\R}$ es la forma bilineal asociada a $L$, es una norma, y es equivalente a la norma natural de $H$.
\end{prop}
\begin{proof}
El hecho de que $\norm{\cdot}_H$ es una norma es sencillo de comprobar ya que $A$ define un producto escalar en $H$.

Sea $\funcion{\norm{\cdot}}{H}{\R}$ la norma de $H$. Necesitamos encontrar dos constante $c_1,c_2\in\R^+$ tal que 
\[
c_1\norm{u}\leq \norm{u}_H \leq c_2\norm{u}
\]
La constante $c_2$ la obtenemos por continuidad de la norma $\norm{\cdot}_H$ y la constante $c_1$ de la coercividad de $L$.

\end{proof}
A partir de ahora consideramos $\norm{\cdot}_H$ como nuestra norma por defecto en $H$.
\begin{prop}
Todo funcional cuadrático coercivo está acotado inferiormente.
\end{prop}
\begin{proof}
Sea $\funcion{L}{H}{\R}$ un funcional cuadrático, donde $\funcion{A}{H\times H}{\R}$ es su forma bilineal, continua y simétrica y $\funcion{R}{H}{\R}$ es su aplicación lineal y continua.
Vamos a ver que podemos acotar inferiormente la expresión de $L(y)$, con $y\in H$, por una función parabólica hacia arriba.

Como $R$ es continua, entonces $\norm{R(y)}\leq \norm{R}\norm{y} \Rightarrow -\norm{R(y)} \geq -\norm{R}\norm{y}$. Además, como $L$ es coerciva, existe cierta constante $C\in\R^+$ tal que $A(y,y)\geq C\norm{y}^2$. Combinando esas dos acotaciones, obtenemos:
\[
L(y)=\frac{1}{2}A(y,y)-R(y)\geq \frac{1}{2}C\norm{y}^2-\norm{R}\norm{y}
\]
Para ver con más claridad la parábola, hacemos el cambio $s=\norm{y}$, obteniendo una nueva función $P$ dependiente de las dos constantes $C$ y $\norm{R}$: $P(s)=\frac{1}{2}s^2-s\norm{R}$. Derivando e igualando a 0, obtenemos fácilmente que el mínimo de la parábola $P$ se alcanza en $s=\frac{\norm{R}}{C}$. Luego:
\[
L(y)\geq \frac{\norm{R}}{C} \espacio \forall y\in H 
\]

\end{proof}

\begin{theorem}[Lax-Milgran]
\label{laxmilgran}
Sea $\funcion{L}{H}{\R}$ un funcional cuadrático. Si $L$ es coercivo, entonces tiene mínimo absoluto en $H$.
\end{theorem}

\begin{prop}[condición de punto crítico]
\label{puntocritico}
Sea $\funcion{L}{H}{\R}$ un funcional cuadrático coercivo, $y\in H$. Entonces existe $y\in H$ verificando:
\[
A(\phi,y)-R(\phi)=0 \espacio \forall\phi\in\sobolev{}
\]

Además, el punto $y$ es el mínimo de $L$.
\end{prop}
\begin{proof}
Calculemos primero la expresión explícita de $g(s)$:
\[
g(s)=L(y+s\phi)=\frac{1}{2}A(y+s\phi,y+s\phi)-R(y+s\phi)
\]
Desarrollamos el primer término (usando la última propiedad en la definición \ref{formabilineal}):
\[
A(y+s\phi,y+s\phi)=A(y,y)+sA(y,\phi)+sA(\phi,y)+s^2A(\phi,\phi)=s^2A(\phi,\phi)+2sA(\phi,\phi)+A(y,y)
\]
Para el segundo término, usamos que $R$ es lineal:
\[
R(y+s\phi)=R(y)+sR(\phi)
\]
Quedándo:
\[
g(s)=\frac{1}{2}\left(s^2A(\phi,y)+2sA(\phi,\phi)+A(y,y)\right)-R(y)-sR(\phi)
\]
Derivando respecto de $s$:
\[
g(s)=sA(\phi,\phi)+A(\phi,y)-R(\phi) \Rightarrow g'(0)=A(\phi,y)-R(\phi)
\]
Luego nuestra condición de punto crítico es:
\[
A(\phi,y)-R(\phi)=0 \espacio \forall\phi\in\sobolev{}
\]

Para le existencia, simplemente tenemos que definir un producto escalar y usar el teorema de Riesz, como en los ejemplos anteriores. Sea $<<u,v>>=A(u,v)$ un producto escalar en $H$, por el teorema \ref{riesz-frechet}, existe $y\in H$ verificando:
\[
<<y,\phi>>=R(\phi) \espacio \forall \phi\in\\sobolev{}
\]

Para ver que es mínimo, tomamos $\funcion{g}{\R}{\R}$ definida por $g(s)=L(y+s(z-y))$. Notemos que $y+s(z-y)$ es la recta que una a $z$ con $y$. Tenemos que ver
\[
L(z)>L(y) \espacio \forall z\in H
\]
Que si nos fijamos, es equivalente a ver que
\[
g(1)>g(0)
\]
Que ya lo tenemos porque $s=0$ era el punto mínimo de la parábola.
\end{proof}

Una vez planteada toda la teoría necesaria, podemos volver a nuestro modelo del puente. Recordemos que nuestro funcional  $\funcion{L}{\sobolev{1}}{\R}$ venía dado por la expresión:
\[
L(y)=\frac{1}{2}\integral{a}{b}{y'(x)^2dx}+\frac{1}{2}\integral{a}{b}{K(x)y(x)^2dx}+\integral{a}{b}{q(x)y(x)dx}
\]
donde $q,K\in\lebesgue{\infty}$ y $K(x)\geq k_0 >0\ \casipordoquier$. La última condición la imponemos porque al dejar los extremos sueltos (lo hemos hecho al considerar como dominio de $L$ el espacio $\sobolev{1}$ en lugar de $\sobolevcero{1}$) necesitemos que el puente siga enganchado por las cuerdas.

Para poder usar la teoría desarrollada anteriormente, necesitamos ver que nuestro funcional es cuadrático. Definimos nuestra forma bilineal, continua y simétrica, y nuestra aplicación lineal y continua:
\[
A(u,v)=\integral{a}{b}{u'(x)v'(x)dx}+\integral{a}{b}{K(x)u(x)v(x)dx}, \; \espacio R(u)=-\integral{a}{b}{q(x)u(x)dx} \;\;\forall u,v\in\sobolev{1}
\]
Pero también necesitamos que $L$ sea coerciva. Usando la hipótesis sobre $K(x)$ es fácil de ver:
\[
A(u,u)=\integral{a}{b}{u'(x)^2dx}+\integral{a}{b}{K(x)u(x)^2dx}\geq\normasobolev{1}{u}^2+k_0\normasobolev{1}{u}^2\geq(1+k_0)\normasobolev{1}{u}^2
\]
Por lo tanto, sabemos que existe mínimo en $\sobolev{1}$. Además de saber que existe, queremos saber algo más sobre su expresión. Como hemos visto antes, los puntos críticos están caracterizados por la expresión:
\[
y\in\sobolev{1} \text{ punto critico } \Leftrightarrow A(y,\phi)=R(\phi) \espacio \forall \phi\in\sobolev{1}
\]
es decir,
\[
\integral{a}{b}{y'(x)\phi'(x)dx}+\integral{a}{b}{K(x)y(x)\phi(x)dx}=-\integral{a}{b}{q(x)\phi(x)dx} \espacio  \forall\phi\in\sobolev{1}
\]
Si en lugar de considerar $\phi$ en $\sobolev{1}$, la consideramos en $\soportecompacto$. Vemos que llamando $z=y'$ y agrupando los términos que tienen $\phi$, llegamos a:
\[
\integral{a}{b}{z(x)\phi'(x)dx}=-\integral{a}{b}{(K(x)y(x)+q(x))\phi(x)dx} \espacio  \forall\phi\in\sobolev{1}
\]

lo que significa que la derivada débil de $z$ es justamente $ky+q$. En particular z está en $\sobolev{1}$, porque todas las funciones que aparecen están en $\lebesgue{2}$. Luego debemos resolver la siguiente ecuación diferencial de orden 2 en $y$:
\[
y''(x)=K(x)y(x)+q(x) \espacio \forall x \in[a,b]
\]
Si imaginamos que $K$ y $q$ son continuas, esa ecuación tendría un gran número de soluciones. De hecho, el conjunto de soluciones sería un espacio biparamétrico (porque la solución depende de las condiciones iniciales sobre $y$ e $y'$). Tenemos que intentar limitarlas de alguna forma.

\begin{prop}\label{productoh1}
Sea $y\in\sobolev{1}$ (el representante continuo) y $\phi\in\mathcal{D}(\R)$, entonces $y\phi\in\sobolev{1}$.
\end{prop}
\begin{proof}
Para ver que $y\phi\in\sobolev{1}$, necesitamos encontrar su derivada débil. ¿Cuál es nuestro candidato a derivada débil? La derivada clásica, es decir, $z=y'\phi+y\phi'$. Para comprobarlo, tenemos que ver que para toda $\psi\in\soportecompacto$ se tiene:

\[
\integral{a}{b}{(y'(x)\phi(x)+y(x)\phi'(x))\psi(x)dx}+\integral{a}{b}{y(x)\phi(x)\psi'(x)dx}=0
\]
Reordenando los términos que tienen $y$ e $y'$, obtenemos:
\[
\integral{a}{b}{[\phi(x)\psi'(x)+\phi'(x)\psi(x)]y(x)dx}+\integral{a}{b}{[\phi(x)\psi(x)]y'(x)dx}=0
\]
Se cumple que si
$\phi \in \mathcal{D}(\mathbb{R}), \psi \in \mathcal{D}((a,b))$
entonce $\phi\psi \in \mathcal{D}((a,b))$ y
\[
  (\phi\psi)' = \phi'\psi + \phi\psi'
\]
Ahora por ser $y \in \mathcal{H}^1(a,b)$ y
$\hat\phi \in \mathcal{\mathbb{R}}$ tenemos que se cumple
\[
\integral{a}{b}{y'(x)\hat\phi(x)}+\integral{a}{b}{y(x)\hat\phi'(x)}=y(x)\hat\phi(x)\Big|^b_a
\]
una expresión general. Tomando $\hat\phi = \phi\psi$ tenemos que 

\[
\integral{a}{b}{y'(x)\hat\phi(x)}+\integral{a}{b}{y(x)\hat\phi'(x)}=y(x)\hat\phi(x)\Big|^b_a = 0
\]
ya que $\phi\psi(b) = \phi\psi(a) = 0$ y obtenemos así el resultado que buscabamos.
\end{proof}

Volviendo al problema original teníamos que

\begin{equation}
  \label{eq:probleminicial}
  \integral{a}{b}{y'(x)\phi(x)}+\integral{a}{b}{k(x)y(x)\phi(x)}=-\integral{a}{b}{q(x)\phi(x)}
\end{equation}


Considerando $\phi \in \mathcal{D}(\mathbb{R}), \phi \in \mathcal{H}^1$ y $z = y'$

obtenemos en la expresión anterior

\begin{equation}
  \label{eq:resparcial}
  \integral{a}{b}{z\phi'}=z\phi(x)\Big|^b_a - \integral{a}{b}{z'\phi}
\end{equation}

Como estabamos resolviendo el problema $y''=ky + q$ tenemos que se cumple

\[
\integral{a}{b}{z'\phi} = \integral{a}{b}{k(x)y(x)+q(x)\phi}
\]

y usando la igualdad que nos da \ref{eq:probleminicial} despejamos en
\ref{eq:resparcial} y obtenemos
\[
y'\phi\Big|^b_a = 0
\]

Si tomamos una $\phi$ como la siguiente

\centerline{\includegraphics[scale=0.5]{img/caso1.png}} 

de la expresión $ 0 = y'\phi\Big|^b_a = y'(b)\phi(b)-y'(a)\phi(a)$
obtenemos $y'(b) = 0$. Tomando como $\phi$

\centerline{\includegraphics[scale=0.5]{img/caso2.png}} 

obtenemos $y'(a) = 0$. Dado que $z=y'\in \mathcal{H}^1$ entonces $y\in \mathcal{H}^2$ y
tomando el representante continuo obtenemos lo que se conoce como la condición de Neumann.

\section{La delta de Dirac y el espacio $H^{-1}$}

Ahora vamos a ver otro caso del modelo, donde colgamos una masa $m$ en un punto $\alpha\in(a,b)$ del puente. La energía de este objeto es $my(\alpha)$, luego la expresión de nuestro funcional $\funcion{L}{\sobolevcero{1}}{\R}$ es:
\[
L(y)=\frac{1}{2}\integral{a}{b}{y'(x)^2dx}+\frac{1}{2}\integral{a}{b}{q(x)y(x)^2dx}+my(\alpha)
\]
donde $y\in\sobolevcero{1}$, $q(x)\geq 0$ y  $q\in\lebesgue{\infty}$. La intuición nos dice que al colgar la masa $m$, se formará un pico en la cuerda, como muestra el siguiente dibujo:

\begin{center}
\begin{tikzpicture}
\draw (0,0) -- (5,0);
% parabola
\draw[scale=1,domain=0:2.5,smooth,variable=\x,red] plot ({\x}, {-0.2*\x});
\draw[scale=1,domain=2.5:5,smooth,variable=\x,red] plot ({\x}, {0.2*(\x-5)});

% etiquetas
\draw[dashed] (0,-0.7) -- (0,.2);
\draw (0,-0.7) node[anchor=north] {$a$};
\draw[dashed] (5,-.7) -- (5,.2);
\draw (5,-0.7) node[anchor=north] {$b$};
\draw (2.5,-0.1) -- (2.5,.1);
\draw (2.5,0) node[anchor=north] {$\alpha$};
\draw (3,-0.7) node[anchor=north] {$m$};

% masa
\fill[blue!40!white, draw=black] (2.25,-1) rectangle (2.75,-0.5);
\end{tikzpicture}
\end{center}
La forma bilineal $A$ es la misma que en el anterior modelo:
\[
A(y,y)=\frac{1}{2}\integral{a}{b}{y'(x)^2dx}+\frac{1}{2}\integral{a}{b}{q(x)y(x)^2dx}
\]
y como estamos en $\sobolevcero{1}$, $\integral{a}{b}{y'(x)^2dx}$ es una norma, luego $A$ es coercivo ya que restando el último término (que es positivo) obtenemos:
\[
A(y,y)\geq \frac{1}{2}\integral{a}{b}{y'(x)^2dx} = \frac{1}{2}\norm{u}^2
\] 
La aplicación lineal y continua $\funcion{R}{\sobolevcero{1}}{\R}$, viene dada por $R(y)=-my(\alpha)$, que tiene sentido al elegir el represante continuo porque $\sobolev{1}\subset \mathcal{C}(a,b)$. Por el teorema de Lax-Milgran (\ref{laxmilgran}) y la proposicion \ref{puntocritico}, existe $y\in\sobolevcero{1}$ de forma que
\[
A(y,\phi)=R(\phi) \espacio \forall \phi\in\sobolevcero{1}
\]
Lo que vamos a hacer ahora es calcular o ver qué condiciones verifica la función $y$. Usando la condición de punto crítico:
\begin{equation}
\label{formulageneral}
\integral{a}{b}{y'(x)\phi'(x)dx}+\integral{a}{b}{q(x)y(x)\phi(x)dx}=-m\phi(\alpha)
\end{equation}
Ahora hacemos una distinción de casos con $\phi$ perteneciendo a diferentes clases de Schwartz:
\begin{itemize}[-]
\item si $\phi\in\mathcal{D}(a,\alpha)$:
\[
\integral{a}{b}{y'(x)\phi'(x)dx}+\integral{a}{b}{q(x)y(x)\phi(x)dx}=-m\phi(\alpha)=0
\]
ya que $\alpha\notin(a,\alpha) \Rightarrow \phi(\alpha)=0$. Si denotamos $z=y'$, entonces $z\in\sobolev[a,\alpha]{1}\subset\mathcal{C}[a,\alpha]$ (porque la expresión anterior coincide con la definición de derivada débil para $z$).
\item si $\phi\in\mathcal{D}(\alpha,b)$, de igual forma conseguimos que $z\in\sobolevcero[\alpha,b]{1}\subset\mathcal{C}[\alpha,b]$. 
\end{itemize}
Fijaros que ese razonamiento no implica que $z\in\mathcal{C}[a,b]$, ya que el representante continuo usado en $[a,\alpha]$ y $[\alpha,b]$ son distintos. Lo que si implica es que $z\in\mathcal{C}\left([a,\alpha]\right)\cap \mathcal{C}\left([\alpha,b]\right)$ y existen los límites laterales de $\alpha$:
\[
z(\alpha^-)=\limite{x}{\alpha^-}{z(x)} \espacio z(\alpha^+)=\limite{x}{\alpha^+}{z(x)}
\]
Además sabemos que $z'=qy \;\; a.e \; x \in(a,\alpha)\cup(\alpha,b)$. Si volvemos a la fórmula general \eqref{formulageneral}:
\begin{equation}
\label{formulageneral2}
\integral{a}{b}{z(x)\phi'(x)dx}+\integral{a}{b}{q(x)y(x)\phi(x)dx}=-m\phi(\alpha) \espacio \forall\phi\in\sobolevcero{1}
\end{equation}
Ahora queremos desarrollar esa expresión. Tomamos $\phi\in\soportecompacto$, como $z$ no es continua en $\alpha$, la integral del primer término hay que escribirla en dos partes:
\[
\integral{a}{b}{z(x)\phi'(x)dx}=\integral{a}{\alpha}{z(x)\phi'(x)dx}+\integral{\alpha}{b}{z(x)\phi'(x)dx}
\]
Aplicamos derivación por partes a cada término:
\[
\integral{a}{\alpha}{z(x)\phi'(x)dx}=z(x)\phi(x)\Big|_a^\alpha-\integral{a}{\alpha}{z'(x)\phi(x)dx}=z(\alpha^-)\phi(\alpha)-\integral{a}{\alpha}{z'(x)\phi(x)dx}
\]
\[
\integral{\alpha}{b}{z(x)\phi'(x)dx}=z(x)\phi(x)\Big|_\alpha^b-\integral{\alpha}{b}{z'(x)\phi(x)dx}=-z(\alpha^+)\phi(\alpha)-\integral{\alpha}{b}{z'(x)\phi(x)dx}
\]
Combinando los resultados y sustituyendo en \eqref{formulageneral2}:
\[
z(\alpha^-)\phi(\alpha)-z(\alpha^+)\phi(\alpha)-\integral{a}{b}{z'(x)\phi(x)dx}=-\integral{a}{b}{z'(x)\phi(x)dx}-m\phi(\alpha)
\]
Cancelando ámbos términos:
\[
\left(z(\alpha^-)-z(\alpha^+)\right)\phi(\alpha)=-m\phi(\alpha)
\]
Como la función de Schwartz escogida es arbitraria, podemos suponer que vale 1 en $\alpha$. Usando eso y que $z=y'$:
\[
y'(\alpha^-)-y'(\alpha^+)=-m \Rightarrow y'(\alpha^+)-y'(\alpha^-)=m  
\]
Es decir, el cambio de pendiente que se da entre $\alpha^-$ y $\alpha^+$ es igual a $m$, el peso de la masa.

\section{Sturm-Liouville: caso 1}

Vamos a generalizar el caso anterior para obtener una ecuación diferencial con varias condiciones. Sea $\funcion{L}{\sobolev{1}}{\R}$ el funcional dado por:
\[
L(y)=\frac{1}{2}\integral{a}{b}{y'(x)^2dx}+\frac{1}{2}\integral{a}{b}{q(x)y(x)^2dx}-R(y)
\]
con $q\in\lebesgue{\infty}$, $0<q_0\leq q(x)\leq q_1 \; \casipordoquier$. 

De donde podemos definir la siguiente forma cuadrática: 
\[
A(y,\phi)=\integral{a}{b}{y'(x)\phi'(x)dx}+\integral{a}{b}{q(x)y(x)^2\phi(x)dx}
\]
Una vez más, la teoría de Lax-Milgran nos dice que existe $y$ en $\sobolev{1}$ cumpliendo $A(y,\phi)=R(\phi)$, es decir, $ \forall R\in \sobolev{-1}=\left(\sobolev{1}\right)^*$:
\begin{equation}
\label{condicion1}
\integral{a}{b}{y'(x)\phi'(x)dx}+\integral{a}{b}{q(x)y(x)\phi(x)dx}=R(\phi) \;\; \forall \phi\in\sobolev{1}
\end{equation}

Una vez planteado el problema y visto que tiene solución, tenemos que intentar determinar $y$ mediante alguna ecuación diferencial. Como vamos a determinar $y$ mediante una ecuación diferencial, necesitamos buscar condiciones de regularidad sobre ella porque la teoría de ecuaciones diferenciales nos dice que hay solución siempre que los elementos de la ecuación sean continuos. Nótese que, por ahora, solo sabemos que $y\in\continuas$ e $y'\in\lebesgue{2}$.
   
Ahora vamos a empezar con el primer caso. Partimos de la condición \eqref{condicion1} y recordamos que tenemos la cadena de espacios:
\[
\lebesgue{2} = \sobolev{0} \subset \sobolev{-1}
\]

Donde la indentificación $\sobolev{0}\longhookrightarrow\sobolev{-1}$ viene dada por la aplicación
\[
\begin{array}{ll}
R: & \lebesgue{2} \longrightarrow \sobolev{-1} \\
  &\;\;\; r \;\;\; \;\;\;\; \longmapsto    \funcion{R_r}{\sobolev{2}}{\R} \\
  & \hspace*{3cm} y \mapsto R_r(y)=\integral{a}{b}{r(x)y(x)dx} 
\end{array}
\]

Nótese ahora que si denotamos $z=y'$, la ecuación \eqref{condicion1} nos dice que se verifica:
\begin{equation}
\label{condicion2}
\integral{a}{b}{z(x)\phi'(x)dx}+\integral{a}{b}{q(x)y(x)\phi(x)dx}=\integral{a}{b}{r(x)\phi(x)dx}
\end{equation}
Y agrupando:
\[
\integral{a}{b}{z(x)\phi'(x)dx}+\integral{a}{b}{\left(q(x)y(x)-r(x)\right)\phi(x)dx}=0
\]
Luego $z$ tiene derivada débil y coincide con $z'=qy-r$. Como $z'=y''$, nos queda la EDO:
\[
y''-qy=-r \Rightarrow -y''+qy=r
\]
Ahora, como $z$ tiene derivada débil, puedo tomar su representante continuo $\bar{z}\in\continuas$, es decir, podemos suponer directamente que $z\in\continuas$. Pero claro, como $z=y'$, entonces $y'\in\continuas$, lo que implica que $y\in\continuas[1]$. En particular, tenemos definido cuanto vale $y'(a)$ e $y'(b)$ (antes no lo teníamos porque solo sabíamos que $y\in\lebesgue{2}$). Luego la ecuación $-y''+qy=r$ se verifica en $\sobolev{2}$ (porque $z\in\sobolev{1}$, luego $y\in\sobolev{2}$).

Se puede comprobar fácilmente que si $r,q\in\continuas$ entonces $y\in\continuas[2]$, debido a que en ese caso, $y''$ sería continua (se expresaría como diferencia de continuas y producto de continuas).

Como la expresión \eqref{condicion2} es válida para toda $\phi\in\sobolev{1}$, podemos tomar una $\phi\in\mathcal{D}(\R)$ y como $z\in\sobolev{2}$, la proposición \ref{productoh1} nos dice que $z\phi\in\sobolev{1}$ y $\left(z\phi\right)'=z\phi'+z'\phi$. Usando integración por partes en el primer término:
\[
\integral{a}{b}{z(x)\phi'(x)dx}=z(x)\phi(x)\Big|_a^b-\integral{a}{b}{z'(x)\phi(x)dx}=
\]
\[
=z(b)\phi(b)-z(a)\phi(a)-\integral{a}{b}{(q(x)y(x)-r(x))\phi'(x)dx}
\]
Sustituyendo ahora lo obtenido en la expresión \eqref{condicion2} y simplificando, llegamos a:
\[
z(b)\phi(b)-z(a)\phi(a)=0 \espacio \forall \phi \in\mathcal{D}(\R)
\]
Si ahora tomamos una función $\phi$ de la siguiente forma:

\centerline{\includegraphics[scale=0.5]{img/caso1.png}} 

Entonces $\phi(a)=0$ y $\phi(b)=1$, de donde queda que $z(b)=0$. Tomando otra $\phi$ de esta forma
\centerline{\includegraphics[scale=0.5]{img/caso2.png}} 

obtenemos que $z(a)=0$. Es decir, $y'(a)=y'(b)=0$.

\section{Sturm-Liouville: caso 2}

Para el siguiente caso tenemos que definir primero la \textit{delta de Dirac}.

\medskip

\begin{definition}
Dado $\alpha\in(a,b)$, definimos la función $\funcion{\delta_\alpha}{\sobolev{-1}}{\R}$
\[
\delta_\alpha(y)=y(\alpha), \espacio \delta_\alpha\in\sobolev{1}, \;\; \delta_\alpha\notin\sobolev{0}
\]
A $\delta_\alpha$ la llamaremos \textit{delta de Dirac}.
\end{definition}
\begin{remark}
Nótese que esta función simplemente es otra forma de denotar la evaluación en $\alpha$, que usamos para indicar en qué punto $\alpha$, de una cuerda $y$, colgamos una masa $c$. 
\end{remark}

Ahora pasamos al siguiente caso, donde suponemos que $R(\phi)=c\delta_\alpha(\phi)=c\phi(\alpha) \espacio \forall \phi\in\sobolev{1}\subset\continuas$, con $a<\alpha<b$. Con esta nueva $R$, vemos otra vez la ecuación \ref{condicion1} con $z=y'$:
\[
\integral{a}{b}{z(x)\phi'(x)dx}+\integral{a}{b}{q(x)y(x)\phi(x)dx}=c\phi(\alpha)
\]
donde solo sabemos que $z\in\lebesgue{2}$ e $y\in\sobolev{1}$ y $z$ no tiene por qué tener derivada débil ya que la expresión anterior no coincide con la de la definición. 

Si ahora considero el intervalo $(a,\alpha)$ y tomo $\phi\in\mathcal{D}(a,\alpha)$, tenemos que $\phi(\alpha)=0$ y
\[
\integral{a}{b}{z(x)\phi'(x)dx}+\integral{a}{b}{q(x)y(x)dx}=0
\]
Luego $z\in\sobolev[a,\alpha]{1}$, que tiene un representante continuo $z_1\in\mathcal{C}[a,\alpha]$. De forma análoga, obtenemos $z_2\in\mathcal{C}[\alpha,b]$. Si definimos $z'$ como la unión de $z_1'$ y $z_2'$:
\[
z'(x)= \left\{
\begin{array}{cc}
z_1'(x) & x\in(a,\alpha) \\
z_2'(x) & x\in(\alpha,b)
\end{array}
\right.
\]
¿y qué pasa con $z'(\alpha)$? No pasa nada, porque $\{\alpha\}$ tiene medida nula asi que no hace falta definirla en ese punto. Además, $z'\in L^2(a,\alpha)$ y $z'\in L^2(\alpha,b)$, luego $z'\in\lebesgue{2}$ por la misma razón. 

Como $z'$ no es la derivada débil de $z$ en $(a,b)$, solo podemos decir que
\[
\integral{a}{b}{z'(x)\phi(x)dx}+\integral{a}{b}{z(x)\phi'(x)dx}=0 \espacio \forall \phi\in\mathcal{D}(a,\alpha) \text{ ó } \forall \phi\in\mathcal{D}(\alpha,b)
\]
Y en ese caso, $z'(x)=-q(x)y(x) \;\; \casipordoquier$. 

Como vemos, este es el origen de la cuestión, tenemos una función $z'$, que sabemos que es una derivada, pero es una derivada \textit{extraña}, ya que es débil en $(a,\alpha)$ y $(\alpha,b)$, coincide en casi todo punto con $-q(x)y(x)$, pero no es una derivada débil en $(a,b)$. 

En principio, no sabemos si $z$ es continua, lo único que sabemos es que $z_1$ y $z_2$ sí lo son, y se cumple:
\[
\begin{array}{cc}
z(x)=z_1(x) & \text{ a.e } x\in(a,\alpha) \\
z(x)=z_2(x) & \text{ a.e } x\in(\alpha,b)
\end{array}
\]
Usando esa propiedad, podemos definir otra función, $z_3$, como sigue:
\[
z_3(x)=\left\{
\begin{array}{cc}
z_1(x) & x\in[a,\alpha) \\
z_2(x) & x\in(\alpha,b]
\end{array}
\right.
\]
Fijaros que $z_3$, por el carácter local de la continuidad, es continua en $(a,\alpha)$ y $(\alpha,b)$. De hecho, es continua en $a$ y $b$ también porque coincide con $z$ en ese entorno. Además, se verifica:
\[
\begin{array}{c}
\limite{x}{\alpha^-}{z_3(x)}=\limite{x}{\alpha^-}{z_1(x)}=z_1(\alpha)\\
\limite{x}{\alpha^+}{z_3(x)}=\limite{x}{\alpha^+}{z_2(x)}=z_2(\alpha)
\end{array}
\]
Es decir, $z_3$ no es continua en $[a,b]$, pero sí tienes límites laterales. Además, la discontinuidad que presenta en $\alpha$ es evitable (ojo, que no es seguro que siempre la tenga). De ahora en adelante, vamos a denotar simplemente por $z$ a $z_3$, ya que no tiene mucho sentido arrastrar ambas notaciones. 

Ahora vamos a razonar con $z_1$, que es una solución en $(a,\alpha)$, luego por la condición de punto crítico, tenemos que:
\[
\integral{a}{b}{z_1(x)\phi'(x)dx}+\integral{a}{b}{q(x)y(x)\phi(x)dx}=0 \espacio \forall\phi\in\mathcal{D}(a,\alpha)
\]
Si consideramos ahora $\phi$ sobre $\mathcal{D}(\R)$ en lugar de sobre $\mathcal{D}(a,\alpha)$, tenemos que:
\[
\integral{a}{\alpha}{z_1(x)\phi'(x)dx}+\integral{a}{\alpha}{q(x)y(x)\phi(x)dx}=0
\] 
Aplicando integración por partes al primer término:
\[
\integral{a}{\alpha}{z_1(x)\phi'(x)dx}=z_1(x)\phi(x)\Big|_a^\alpha-\integral{a}{\alpha}{q(x)y(x)\phi(x)dx}
\]
Cogiendo la $\phi$ de forma que $\phi(a)=1$ y $\phi(x)=0\quad\forall x\geq\alpha$, y sustituyendo en la expresión anterior:
\[
-z_1(a)\phi(a)-\integral{a}{\alpha}{q(x)y(x)\phi(x)dx}+\integral{a}{\alpha}{q(x)y(x)\phi(x)dx}=0 \Rightarrow -z_1(a)\phi(a)=0
\]
Es decir, $z_1(a)=0$. De forma análoga, se prueba que $z_2(b)=0$. Eso implica que $y'(a)=y'(b)=0$, luego: 
\[
\left.
\begin{array}{cc}
y\in\sobolev[a,\alpha]{2}\\
y\in\sobolev[\alpha,b]{2}
\end{array}
\right\} \Rightarrow
\left.
\begin{array}{cc}
y\in\mathcal{C}^1[a,\alpha]\\
y\in\mathcal{C}^1[\alpha,b]
\end{array}
\right\}
\]
En conclusión, hemos llegado a la ecuación diferencial
\[
-y''(x)+q(x)y(x)=0
\]
que sólo se cumple en $(a,\alpha)$ y $(\alpha,b)$. Quedaría ver qué pasa en el extremo, aunque más o menos se ve que simplemente no van a coincidir las derivadas laterales: $y'(\alpha^+)-y'(\alpha^-)=-c$

\section{Sturm-Liouville: caso 3}

Podemos combinar los dos casos anteriores, para resolver el caso donde $R$ vene dado por:
\[
R(x)=r(x)+\sum_{i=1}^nc_i\delta_{\alpha_i} \espacio a<\alpha_1<\cdots<\alpha_n<b
\]
donde $y\in\sobolev{1}$, $y\in\mathcal{H}^2\left((a,\alpha_1)\cup\dots\cup(\alpha_n,b)\right)$, $y\in\mathcal{C}^1\left((a,\alpha_1)\cup\dots\cup(\alpha_n,b)\right)$, y por supuesto $y\in\mathcal{C}[a,b]$. La ecuación diferencial de este caso es la misma que en los dos anteriores:
\[
-y''(x)+p(x)y(x)=0
\]
La \textit{condición de contorno} también se verifica: $y'(a)=y'(b)=0$ y$y'(\alpha^+_i)-y'(\alpha^-_i)=-c_i$

\section{Sturm-Liouville: caso 4}

En este caso, en lugar de minimizar en $\sobolev{1}$, podemos hacerlo en $\sobolevcero{1}$. La única diferencia es que la concición $y'(a)=y'(b)=0$ se sustituye por
\[
y(a)=y(b)=0
\]
Además, basta tomar $0\leq q(x)\leq q_1$. En este caso también podemos suponer que 
\[
R(x)=r(x)+\sum_{i=1}^nc_i\delta_{\alpha_i} \espacio a<\alpha_1<\cdots<\alpha_n<b
\]
verificándose casi lo mismo que en el caso anterior: $y\in\sobolev{1}$, $y\in\mathcal{H}^2\left((a,\alpha_1)\cup\dots\cup(\alpha_n,b)\right)$, $y\in\mathcal{C}^1\left((a,\alpha_1)\cup\dots\cup(\alpha_n,b)\right)$, $y\in\mathcal{C}[a,b]$ pero $y(a)=y(b)=0$.

\section{Sturm-Liouville: caso 5}

En este caso vamos a añadir un peso, por lo que nuestro funcional cambia un poco:
\[
L(y)=\frac{1}{2}\integral{a}{b}{p(x)y'(x)^2dx}+\frac{1}{2}\integral{a}{b}{q(x)y(x)^2dx}-R(y)
\]
Con las condiciones: $0<p_0\leq p(x) \leq p_1$, $0<q_0\leq q(x) \leq q_1$.
Con este planteamiento, la ecuación que se verifica es:
\[
\integral{a}{b}{p(x)y'(x)\phi'(x)dx}+\integral{a}{b}{q(x)y(x)\phi(x)dx}=R(\phi) \espacio \forall \phi\in\sobolev{1}
\]
con $y\in\sobolev{1}$, $y\in\mathcal{C}[a,b]$. Si tomo $z(x)=p(x)y'(x)$, ¿cuáles son las condiciones de regularidad?
\begin{enumerate}[(a)]
\item Si $R(x)=r(x)$, entonces $z\in\sobolev{1}$ y $z\in\continuas$. Además:
\[
-(p(x)y'(x))'+q(x)y(x)=r(x)
\]
Y $z(a)=z(b)=0$.
\item Si cambiamos $\sobolev{1}$ por $\sobolevcero{1}$, la condición $z(a)=z(b)=0$ la tenemos que cambiar por $y(a)=y(b)=0$. Y en este caso, basta con que se cumpla $0\geq q(x)\leq q_1$.
\item Si $R(x)=r(x)+\sum_{i=1}^nc_i\delta_{\alpha_i} \espacio a<\alpha_1<\cdots<\alpha_n<b$ e $y\in\sobolev{}$. Luego $z\in\sobolev[a,\alpha_1]{1}$, $\cdots$, $z\in\sobolev[\alpha_n,b]{1}$, e igual con las continuas: $z\in\mathcal{C}[a,\alpha_1]$, $\cdots$, $z\in\mathcal{C}[\alpha_n,b]$. La ecuación que s e verifica es:
\[
(p(x)y'(x))'+q(x)y(x)=r(x)
\]
Con la condición de contorno $z(a)=z(b)=0$ (si $\sobolev{}$) ó $y(a)=y(b)=0$ (si $\sobolevcero{1}$). Las condiciones en las derivadas laterales son:
\[
z(\alpha^+_i)-z(\alpha_i^-)=-c_i
\]
\end{enumerate}

\section{Sturm-Liouville: caso regular (1)}

En este caso tenemos $y\in\sobolev{1}$ y nuestro funcional (sin peso) es:
\[
L(y)=\frac{1}{2}\integral{a}{b}{y'(x)^2dx}+\frac{1}{2}\integral{a}{b}{q(x)y(x)^2dx}-R(y)
\]
La diferencia es que vamos a considerar $R=r(x)$ con $r\in\continuas$, $q\in\continuas$ con 
\[
\begin{array}{rr}
0<q_0\leq q(x)\leq q_1 & \text{ si } \sobolev{}\\
0\leq q(x) \leq q_1    & \text{ si } \sobolevcero{1}
\end{array}
\]
Ahora aqui hay una demostracion  pero no sé ni de que es ¿?.

\section{Sturm-Liouville: caso regular (2)}

Igual que el anterior, pero añadiendo deltas de Dirac:
\[
R(\phi)=r(x)+\sum_{i=1}^nc_i\delta_{\alpha_i} \espacio a<\alpha_1<\cdots<\alpha_n<b
\]
¿?

\section{Sturm-Liouville: caso regular (3)}
igual pero con peso.
%\documentclass[12pt]{article}
 
\usepackage[margin=1in]{geometry} 
\usepackage{amsmath,amsthm,amssymb}
\usepackage[spanish]{babel}
\usepackage[utf8]{inputenc}
\usepackage{tikz-cd}
\usepackage{amsmath}
\usepackage[shortlabels]{enumitem}
\usepackage{mathtools}

% cosas entre comillas 
\usepackage{csquotes}

\usepackage{tikz}


\usepackage{xcolor}

%\usepackage{config}

\newtheorem{theorem}{Teorema}[section]
\newtheorem{lemma}[theorem]{Lema}
\newtheorem{prop}[theorem]{Proposición}
\newtheorem{coro}[theorem]{Corolario}
\newtheorem{conj}[theorem]{Conjetura}
\newtheorem{ejercicio}{Ejercicio}[section]
\newtheorem*{ejercicio*}{Ejercicio}
\theoremstyle{definition}
\newtheorem{definition}[theorem]{Definición}
\newtheorem{example}[theorem]{Ejemplo}
\theoremstyle{remark}
\newtheorem{remark}[theorem]{Nota}
\newtheorem{notacion}[theorem]{Notación}
\newcommand{\continuas}[1][]{C^{ #1 }[a,b]}
\newcommand{\continuasabierto}[1][]{C^{ #1 }(a,b)}
\newcommand{\soportecompacto}{\mathcal{D}(a,b)}
\newcommand{\xcero}{(a,b)}
\newcommand{\xcerocerrado}{[a,b]}
\newcommand{\fvariaciones}{F(x,y(x),y'(x))}

\begin{document}

\section{Ejercicios tema 1}

\begin{ejercicio}
  Sean $\Phi\in \mathcal{C}^1(\mathbb{R}^3), y(x)\in \mathcal{C}^2\xcero, y'(x) \neq 0$. Dado el funcional

  \[
    F[y] = \int_{a}^{b}{\Phi(x,y(x),y'(x))dx}
  \]

  demuéstrese la equivalencia de las dos formas siguientes de las
  ecuaciones de Euler-Lagrange.

  \begin{itemize}
  \item $\frac{\partial\Phi}{\partial y}-\frac{d}{dx}\frac{\partial\Phi}{\partial p} = 0 $
  \item $\frac{\partial\Phi}{\partial x} - \frac{d}{dx}(\Phi-y'\frac{\partial\Phi}{\partial p}) = 0$
  \end{itemize}

  \begin{proof}
    Comenzamos viendo el caso $a) \implies b)$. Para ello en primer
    lugar tenemos que comprobar que efectivamente podemos derivar
    $\Phi$ respecto al tercer parámetro. Del apartado $a)$ sabemos que

    \[
      \frac{\partial\Phi}{\partial y} = \frac{d}{dx}\frac{\partial\Phi}{\partial p}
    \]
    
    y del enunciado sabemos que $\Phi$ es de calse $\mathcal{C}^1$
    luego $\frac{\partial\Phi}{\partial p}$ es de $\mathcal{C}^1$.

    Cuando derivamos $\Phi(x,y(x),y'(x))$ obtenemos

    \begin{align*}
      \Phi(x,y(x),y'(x))' & = \frac{\partial\Phi}{\partial x}(x, y(x), y'(x)) \\
                          & = \frac{\partial\Phi}{\partial y}(x, y(x), y'(x))y'(x) \\
                          & = \frac{\partial\Phi}{\partial p}(x, y(x), y'(x))y''(x)
    \end{align*}

    Recordemos que $\frac{\partial\Phi}{\partial p}(x, y(x), y'(x)) = z(x)$ luego

    \begin{align*}
      \frac{d}{dx}(\Phi-y'\frac{\partial\Phi}{\partial p}) & = \Phi_x + \Phi_yy' + \Phi_py'' - y''z -y'z'
    \end{align*}

    Tenemos que el 3º y 4º termino son iguales t el 2º y 4º son
    iguales entre ellos luego obtenemos

        \begin{align*}
      \frac{d}{dx}(\Phi-y'\frac{\partial\Phi}{\partial p}) = \Phi_x
        \end{align*}

        que es lo que queríamos.

        $b)\implies a)$
        
        Todos los pasos que hemos dado son reversibles pero
        necesitamos ver que
        $\frac{\partial\Phi}{\partial p}(x, y(x), y'(x))$ es $\mathcal{C}^1$.

        Llamando $H(x) = \Phi-y'(x)z(x)$, por hipótesis tenemos que
        $H$ es derivable. Despejando tenemos que

        \[
          z = \frac{\Phi-H}{y'}
        \]

        luego se verifica cómo queríamos.

        \begin{align*}
          \Phi_x & = \Phi_x + \Phi_yy' + \Phi_py'' - y''z -y'z' \\
                 & \implies y'(\Phi_Y-z') = 0
        \end{align*}

        e $y' \neq 0$ por hipótesis luego $\Phi_Y - z' = 0$.
  \end{proof}
\end{ejercicio}

\end{document}
\end{document}


% codigo para dibujar funciones tests
%\begin{tikzpicture}[scale=0.5]
%\node[inner sep=0pt] (russell) at (0,0)
%    {\includegraphics[scale=1]{img/flat.png}};
%\draw (-8,-7.75)-- (8,-7.75);
%\draw[-,dashed] (-5,0)-- (5,0); 
%\draw (-6,0) node[anchor=north] {$1$};
%\draw[-,dashed] (-7,-8.25) -- (-7,-7.25);
%\draw (-7,-8.5) node [anchor=north] {$a$};
%\draw[-,dashed] (0,-8.25) -- (0,-7.25);
%\draw (0,-8.5) node [anchor=north] {$b$};
%\end{tikzpicture}

