\subsection*{Examen matemapli 26 de marzo}
\begin{ejercicio}
Sea $E(x)$ la función parte entera, entonces su derivada débil vale...
\end{ejercicio}
Es una función escalera, luego no tiene representante continuo, luego no tiene derivada débil.

\begin{ejercicio}
Las funciones 
\[
f(x)=\left\{
\begin{array}{cc}
0 & x\notin \Z \\
1 & x\in \Z 
\end{array}
\right. \espacio 
g(x)=\left\{
\begin{array}{cc}
1 & x\notin \Z \\
0 & x\in \Z 
\end{array}
\right.
\]
\begin{enumerate}[(a)]
\item Son iguales en $L_{loc}^1(\R)$.
\item No están en $L_{loc}^1(\R)$.
\item $f$ está en $L_{loc}^1(\R)$  pero $g$ no.
\item Están en $L_{loc}^1(\R)$ pero son diferentes.
\end{enumerate}
\end{ejercicio}

Las dos son localmente integrables pero tienen representantes continuos distintos, luego es la D.

\begin{ejercicio}
Sea $\funcion{f}{(0,1)}{\R}$ definida por
\[
f(x)=\left\{
\begin{array}{cc}
1 & |x|<\frac{1}{2} \\
0 & |x|\geq \frac{1}{2} 
\end{array}
\right.
\]
\begin{enumerate}[(a)]
\item No tiene derivada débil en $\left(-\frac{1}{2},\frac{1}{2}\right)$.
\item Está en $\sobolevcero[-\frac{1}{2},\frac{1}{2}]{1}$.
\item Está en $\sobolev[-\frac{1}{2},\frac{1}{2}]{1}$ pero no en $\sobolevcero[-\frac{1}{2},\frac{1}{2}]{1}$.
\item Ninguna de las anteriores es correcta.
\end{enumerate}
\end{ejercicio}

La $A$ es falsa porque tiene derivada débil en ese intervalo. Hacer aclaración en los apuntes sobre $\sobolevcero{1}$ evaluando en el representante continuo. Es la $C$ entonces, esta en el sobolevc 1 pero eno el soole 0.

\begin{ejercicio}
EL funcional $L(y)=\integral{0}{1}{\left(y'(x)x^2-xy(x)\right)dx}$...
\begin{enumerate}[(a)]
\item No tiene puntos críticos en $\sobolev[0,1]{1}$.
\item El punto crítico es la función $y\equiv\frac{-1}{2}$ que está en $\sobolev[0,1]{1}$ pero no en $\sobolevcero[0,1]{1}$.
\item No está definido en $\sobolev[0,1]{1}$.
\item Tiene puntos críticos en $\sobolevcero[0,1]{1}$ pero no en $\sobolev[0,1]{1}$.
\end{enumerate}
\end{ejercicio}

Si partimos de la ecuación de Euler, podemos usar la condición de punto crítico y llegas a que si se cumple la ecuación todo $x$ en $(0,1)$ es 0.

Ni idea, le damos a la A.