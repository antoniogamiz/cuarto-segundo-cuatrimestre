% !TeX root = master.tex
\documentclass[12pt]{report}
 
\usepackage[margin=1in]{geometry} 
\usepackage{amsmath,amsthm,amssymb}
\usepackage[utf8]{inputenc}
\usepackage{amsmath}
\usepackage[shortlabels]{enumitem}
\usepackage{mathtools}
\usepackage{amsfonts}
\usepackage{float}
\usepackage{epigraph}
\usepackage{lipsum}


\usepackage[spanish]{babel}
\usepackage{tikz}
\usetikzlibrary{babel}

% cosas entre comillas 
\usepackage{csquotes}

\usepackage{xcolor}

\usepackage{../texmf/tex/latex/config/config}

\newtheorem{theorem}{Teorema}
\newtheorem{lemma}[theorem]{Lema}
\newtheorem{prop}[theorem]{Proposición}
\newtheorem{coro}[theorem]{Corolario}
\newtheorem{conj}[theorem]{Conjetura}
\newtheorem{ejercicio}{Ejercicio}
\newtheorem*{ejercicio*}{Ejercicio}
\theoremstyle{definition}
\newtheorem{definition}[theorem]{Definición}
\newtheorem{example}[theorem]{Ejemplo}
\theoremstyle{remark}
\newtheorem{remark}[theorem]{Nota}
\newtheorem{notacion}[theorem]{Notación}


\begin{document}
\begin{ejercicio}
Estudia si la función $\funcion{f}{\R}{\R}$ definida por
\[
f(x):=\left\{
\begin{array}{cc}
x^2 & |x|\leq 1 \\
|x| & |x| > 1
\end{array}
\right.
\]
es lipschitziana y, en caso de serlo,  calcula una constante de Lipschitz óptima para $f$.
\end{ejercicio}

Estudiamos si su derivada está acotada:
\[
f'(x):=\left\{
\begin{array}{cc}
2x & |x|\leq 1 \\
1 & x > 1 \\
-1 & x < 1
\end{array}
\right.
\]
Que vemos que podemos acotarla por 2, y que esa es la constante más pequeña con la que podemos acotarla, luego es la óptima.

\begin{ejercicio}
Estudia si la función 
\[
\funcion{f}{\R}{\R}, f(x)=36x\sin(x+6)
\]
es lipschitziana y, en caso de serlo,  calcula una constante de Lipschitz óptima para $f$.
\end{ejercicio}

Igual que antes, calculamos su derivada:
\[
f'(x)=36\sin(x+6)+36x\cos(x+6)
\]
Que vemos que no podemos acotar por el término $x$ que aparece, luego no es lipschitz.

\begin{ejercicio}
Estudia si la función 
\[
\funcion{f}{\R^2}{\R}, f(x,y)=\sin(x)\arctan(y)
\]
es lipschitziana y, en caso de serlo,  calcula una constante de Lipschitz para $f$ cuando se usa en $\R^2$ la norma del máximo.
\end{ejercicio}
Calculamos las dos derivadas parciales y las intentamos acotar:
\[
\derivada{f}{x}(x,y)=\cos(x)\arctan(y)\Rightarrow \norm{\derivada{f}{x}}_\infty \leq \frac{\pi}{2} 
\]
\[
\derivada{f}{y}(x,y)= \frac{\sin(x)}{1+y^2} \Rightarrow \norm{\derivada{f}{y}}_\infty \leq 1 
\]
Luego es lipschitziana con constante $L=\frac{\pi}{2}$.

\begin{ejercicio}
¿Cuál de los siguientes funciones \textbf{no} es localmente lipschitziana en $\R$?
\begin{enumerate}[(a)]
\item $f(x)=\log(x^2+1)$
\item $f(x)=x+|x|$
\item $f(x)=|x|^{7/6}$
\item $f(x)=|x|^{8/7}$
\item $f(x)=x|x|$
\item $f(x)=|x|^{6/7}$
\end{enumerate}
\end{ejercicio}

Si nos fijamos en la $(f)$ y la derivamos:
\[
f'(x):=\left\{
\begin{array}{cc}
x^{-1/7} & x \geq 0 \\
-x^{-1/7} & x < 0
\end{array}
\right.
\]
Como el exponente es negativo, en un entorno de $x=0$ no se va a poder acotar, luego no es localmente lipschitziana.

\begin{ejercicio}
Estudia si la función 
\[
\funcion{f}{\R}{\R}, f(x)=8\arctan(7x+3)
\]
es lipschitziana y, en caso de serlo,  calcula una constante de Lipschitz óptima para $f$.
\end{ejercicio}

Calculamos su derivad:
\[
f'(x)=\frac{56}{1+(7x+3)^2}
\]
Luego sí es lipschitziana con $L=56$.

\begin{ejercicio}
Sea $\funcion{f}{A\subset \R^d}{\R}$ una función lipschitziana cuyo valor absoluto está acotado inferiormente:
\[
R=\inf\{|f(x)|:x\in A\}>0
\]
y cuya constante de Lipschitz óptima es $L_f$. Estudia si
\[
\funcion{F}{A}{\R}, \;\; F(x)=\frac{1}{f(x)}
\]
es lipschitziana y, en caso de serlo, calcula una constante de Lipschitz para $F$.
\end{ejercicio}

\[
\norm{F(x)-F(y)}=\norm{\frac{1}{F(x)}-\frac{1}{F(y)}}=\norm{\frac{F(y)-F(x)}{F(x)F(y)}}\leq \frac{L_f\norm{x-y}}{R^2}
\]
Luego sí lo es con constante $L=\frac{L_f}{R^2}$.

\begin{ejercicio}
Sean $\funcion{f}{A\subset\R^d}{\R^k}$, $\funcion{g}{A}{\R}$ dos funciones lipschitzianas con constantes de Lipschitz óptimas $L_f$ y $L_g$ (resp.). Estudia si el producto:
\[
\funcion{F}{A}{\R^k}, \;\; F(x):=g(x)f(x)
\]
es lipschitziana y, en caso de serlo, calcula una constante de Lipschitz para $F$.
\end{ejercicio}

No lo es, considerar $f(x)=x$, $g(x) \Rightarrow F(x)=x^2$ que no es lipstchitziana ya que su derivada es $F'(x)=2x$, que no está acotada.

\begin{ejercicio}
Sea $A\subset\R^d$ y sean $\funcion{f_1,f_2}{A}{\R^k}$ dos funciones lipschitzianas con constantes de Lipschitz óptimas $L_1$ y $L_2$ (resp.). Estudia si la combinación lineal
\[
\funcion{F}{A}{\R^k}, \;\; F(x)=af_1(x)+bf_2(x) \espacio a,b\in\R
\]
es lipschitziana y, en caso de serlo, calcula una constante de Lipschitz para $F$.
\end{ejercicio}
\[
|F(x)-F(y)|=|af_1(x)+bf_2(x)-af_1(y)-bf_2(y)|\leq|a||f_1(x)-f_2(x)|+|b||f_1(x)-f_2(y)|\leq
\]
\[
\leq|a|L_1|x-y|+|b|L_2|x-y|=(|a|L_1+|b|L_2)
\]

\begin{ejercicio}
Sean $\funcion{f}{\R^d}{\R^k}$ y $\funcion{g}{\R^k}{\R^n}$ dos funciones lipschitzianas con constantes de Lipschitz óptimas $L_f$ y $L_g$ (resp.). Estudia si la composición
\[
\funcion{F}{\R^d}{\R^n}, \;\; F(x)=g(f(x))
\]
es lipschitziana y, en caso de serlo, calcula una constante de Lipschitz para $F$.
\end{ejercicio}

\[
\norm{g(f(x_1))-g(f(x_2))}\leq L_g\norm{f(x_1)-f(x_2)}\leq L_fL_g\norm{x_1-x_2} \espacio x_1,x_2\in \R^d
\]

Luego sí lo es, con constante $L$ igual a $L_fL_g$.

\begin{ejercicio}
Sea $\funcion{g}{\R^d}{\R^d}$ globalmente lipschitziana y $A\in\mathcal{M}_d(\R)$ (una matriz cuadrada). ¿Podemos asegurar que la función $\funcion{F}{\R\times\R^d}{\R^d}$, $F(t,x)=g(tAx)$ es globalmente lipschitziana?
\end{ejercicio}

Si no estuviera la $t$ sí que lo es (lo han dicho los demás, no me he enterado por qué).
Está en el archivo que han pasado del año pasado.
\end{document}
