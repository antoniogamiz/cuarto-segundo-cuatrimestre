% !TeX root = master.tex
\documentclass[12pt]{article}
 
\usepackage[margin=1in]{geometry} 
\usepackage{amsmath,amsthm,amssymb}
\usepackage[spanish]{babel}
\usepackage[utf8]{inputenc}
\usepackage{tikz-cd}
\usepackage{amsmath}
\usepackage[shortlabels]{enumitem}
\usepackage{mathtools}

% cosas entre comillas 
\usepackage{csquotes}

\usepackage{tikz}

\decimalpoint
\usepackage{xcolor}

\usepackage{personalcommands}

\newtheorem{theorem}{Teorema}[section]
\newtheorem{lemma}[theorem]{Lema}
\newtheorem{prop}[theorem]{Proposición}
\newtheorem{coro}[theorem]{Corolario}
\newtheorem{conj}[theorem]{Conjetura}
\newtheorem{ejercicio}{Ejercicio}
\newtheorem*{ejercicio*}{Ejercicio}
\theoremstyle{definition}
\newtheorem{definition}[theorem]{Definición}
\newtheorem{example}[theorem]{Ejemplo}
\theoremstyle{remark}
\newtheorem{remark}[theorem]{Nota}
\newtheorem{notacion}[theorem]{Notación}
\newcommand{\continuas}[1][]{C^{ #1 }[a,b]} 

\begin{document}

\begin{ejercicio}
Demuestra que la solución maximal del siguiente PVI está acotada (en el futuro y en el pasado):
\[
\left.
\begin{array}{c}
x'=(1-x^2-y^2)(x+y)\\
y'=(1-x^2-y^2)(x-y)\\
x(0)=\frac{1}{3}, \;\; y(0)=-\frac{2}{3}
\end{array}
\right\}
\]
\end{ejercicio}

\begin{ejercicio}
En cada uno de los problemas de valores iniciales siguientes, demuestra que la solución maximal está acotada en el futuro.
\end{ejercicio}

Denotamos por $\varphi$ a la solución maximal en cada uno de los respectivos apartados.

\begin{enumerate}[(a)]
\item $
\left\{
\begin{array}{l}
x'=\frac{x^2-25}{1+t^2}\\
x(0)=4
\end{array}
\right.
$

Como $f(t,x)\in\mathcal{C}^1$, es localmente lipschitziana y por tanto verifica la propiedad de unicidad global. Los puntos de equilibrio de $f$ son $x_1=-5$ y $x_2=5$. Como $x_0\in[-5,5]$, entonces $\varphi$ cumple que $\varphi(t)\in[-5,5]$ $\forall t\geq 0$.
\item $
\left\{
\begin{array}{l}
x'=\frac{16}{1+t^2}x-x^3\\
x(0)=3
\end{array}
\right.
$

Como $f(t,x)\in\mathcal{C}^1$, es localmente lipschitziana y por tanto verifica la propiedad de unicidad global. Los puntos de equilibrio de $f$ son $x_1=0$ y $x_2=1$. Como $x_0\notin[0,1]$, usamos:
\[
\begin{array}{ll}
x_3=5:& f(t,5)<0 \espacio \forall t\in\R\\
x_4=0.1: & f(t,0.1)>0 \espacio \forall t\in\R
\end{array}
\]
Ahora se cumple que $x_0\in[0.1,5]$, siendo $[0.1,5]$ un conjunto invariante. Luego $\varphi$ cumple que $\varphi(t)\in[0.1,5]$ $\forall t\geq 0$.

\item $
\left\{
\begin{array}{l}
x'=\frac{1-x^3}{1+x^2}\\
x(0)=2
\end{array}
\right.
$

Como $f(t,x)\in\mathcal{C}^1$, es localmente lipschitziana y por tanto verifica la propiedad de unicidad global. Los puntos de equilibrio de $f$ son $x_1=-1$ y $x_2=1$. Como $x_0\notin[-1,1]$, usamos:
\[
\begin{array}{ll}
f(t,0)>0 \espacio \forall t\in\R\\
f(t,2)<0 \espacio \forall t\in\R
\end{array}
\]
Luego $\varphi$ cumple que $\varphi(t)\in(0,2)$ $\forall t\geq 0$.

\end{enumerate}

\begin{ejercicio}
Busca funciones guía que nos permitan asegurar que las soluciones de las siguientes ecuaciones diferenciales están acotadas en el futuro:
\end{ejercicio}
\begin{enumerate}[(a)]
\item $x'=\sin(t)-x^3$

Sea $V(x)=x^2$. Entonces:
\[
\dot{V}(x)=2x\sin(t)-2x^4\leq 0, \espacio \forall x\geq n_0
\]
Con $n_0\in\N$ suficientemente grande. Luego $V$ es una función guía y coerciva, luego $(a)$ está acotada en el futuro.

\item $
\left\{
\begin{array}{l}
x_1'=x_2\\
x_2'=-\sin(x_1)-x_1
\end{array}
\right.
$

Sea $V(x_1,x_2)=V_1(x_1)+V_2(x_2)$. Entonces:
\[
\dot{V}(x_1,x_2)=V_1'(x_1)x_2-V_2'(x_2)(\sin(x_1)+x_1)
\]
Para que $\dot{V}(x_1,x_2)\equiv 0$ tiene que darse:
\[
V_1'(x_1)=\sin(x_1)+x_1, \espacio V_2'(x_2)=x_2
\]
Luego, integrando:
\[
V_1(x_1)=-\cos(x_1)+\frac{x_1^2}{2}, \espacio V_2(x_2)=\frac{x_2^2}{2}
\]
Luego $V$ es una función guía y coerciva, luego $(b)$ está acotada en el futuro.
\item $
\left\{
\begin{array}{l}
x'_1=-x_1+x_2\\
x'_2=-x_2
\end{array}
\right.
$

Sea $V(x_1,x_2)=x_1^2+x_2^2$. Entonces:
\[
\dot{V}(x_1,x_2)=2x_1(-x_1+x_2)+2x_2(-x_2)=-x_1^2-x_2^2-(x_1-x_2)^2\leq 0 \espacio \forall (x_1,x_2)\in\R^2
\]
Luego $V$ es una función guía y coerciva, luego $(c)$ está acotada en el futuro.
\end{enumerate}

\begin{ejercicio}
Determina qué debe cumplir el parámetro $a>0$ para que:
\[
V(x,y)=x^2+axy+y^2
\]
sea una función guía coerciva para el sistema plano:
\[
\left\{
\begin{array}{l}
x'=-x+6y\\
y'=-20y
\end{array}
\right.
\]
(Debes indicar un intervalo). Además justifica la veracidad o falsedad de las siguientes afirmaciones:
\begin{enumerate}[(a)]
\item Todas las soluciones del sistema están acotadas en el futuro.
\item Todas las soluciones del sistema están acotadas en el pasado.
\item Existe una solución que está acotada (en el pasado y en el futuro).
\end{enumerate}
\end{ejercicio}

\end{document}