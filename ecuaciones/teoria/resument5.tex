% !TeX root = master.tex
\documentclass[12pt]{article}
 
\usepackage[margin=1in]{geometry} 
\usepackage{amsmath,amsthm,amssymb}
\usepackage[spanish]{babel}
\usepackage[utf8]{inputenc}
\usepackage{tikz-cd}
\usepackage{amsmath}
\usepackage[shortlabels]{enumitem}
\usepackage{mathtools}

% cosas entre comillas 
\usepackage{csquotes}

\usepackage{tikz}

\decimalpoint
\usepackage{xcolor}

\usepackage{personalcommands}

\newtheorem{theorem}{Teorema}
\newtheorem{lemma}[theorem]{Lema}
\newtheorem{prop}[theorem]{Proposición}
\newtheorem{coro}[theorem]{Corolario}
\newtheorem{conj}[theorem]{Conjetura}
\newtheorem{ejercicio}{Ejercicio}
\newtheorem*{ejercicio*}{Ejercicio}
\theoremstyle{definition}
\newtheorem{definition}[theorem]{Definición}
\newtheorem{example}[theorem]{Ejemplo}
\theoremstyle{remark}
\newtheorem{remark}[theorem]{Nota}
\newtheorem{notacion}[theorem]{Notación}
\newcommand{\continuas}[1][]{C^{ #1 }[a,b]} 
\newcommand*{\TickSize}{2pt}%

\newcommand*{\AxisMin}{0}%
\newcommand*{\AxisMax}{0}%

\newcommand*{\DrawHorizontalPhaseLine}[4][]{%
    % #1 = axis tick labels
    % #2 = right arrows positions as CSV
    % #3 = left arrow positions as CSV
    \gdef\AxisMin{0}%
    \gdef\AxisMax{0}%
    \edef\MyList{#2}% Allows for #1 to be both a macro or not
    \foreach \X in \MyList {
        \draw  (\X,\TickSize) -- (\X,-\TickSize) node [below] {$\X$};
        \ifnum\AxisMin>\X
            \xdef\AxisMin{\X}%
        \fi
        \ifnum\AxisMax<\X
            \xdef\AxisMax{\X}%
        \fi
    }

    \edef\MyList{#3}% Allows for #2 to be both a macro or not
    \foreach \X in \MyList {% Right arrows
        \draw [->] (\X-0.1,0) -- (\X,0);
        \ifnum\AxisMin>\X
            \xdef\AxisMin{\X}%
        \fi
        \ifnum\AxisMax<\X
            \xdef\AxisMax{\X}%
        \fi
    }

    \edef\MyList{#4}% Allows for #3 to be both a macro or not
    \foreach \X in \MyList {% Left arrows
        \draw [<-] (\X-0.1,0) -- (\X,0);
        \ifnum\AxisMin>\X
            \xdef\AxisMin{\X}%
        \fi
        \ifnum\AxisMax<\X
            \xdef\AxisMax{\X}%
        \fi
    }

    \draw  (\AxisMin-1,0) -- (\AxisMax+1,0) node [right] {#1};
}%
\begin{document}

\section{Conjuntos invariantes y acotación de soluciones}

\subsection{Definiciones}

Sea $\funcion{f}{D}{\R^d}$ con $D=(a,+\infty)\times\R^d$ y continua. Consideramos el problema:
\begin{equation}
x'=f(t,x) \tag{*}
\end{equation}

\begin{definition}
Diremos que un conjunto $B\subset\R^d$ es \textbf{positivamente invariante} para la ecuación $(*)$ si $\forall x_0\in B$ y $\forall t_0>a$ y para toda solución $\funcion{\varphi}{(\alpha,\beta)}{\R^d}$ de
\[
\left\{
\begin{array}{l}
x'=f(t,x)\\
x(t_0)=x_0
\end{array}
\right.
\]
se verifica que $\varphi(t)\in B \; \forall t \geq t_0$. De forma análoga se puede definir un conjunto \textbf{negativamente invariante}. Diremos que un conjunto es \textbf{invariante} cuando es positiva y negativamente invariante.
\end{definition}

\begin{definition}
Diremos que una solución $\funcion{\varphi}{(\alpha,\omega)}{\R^d}$ es \textbf{acotada en el futuro} si:
\begin{itemize}
\item $\omega=+\infty$
\item $\exists M \geq 0$ tal que $\norm{\varphi(t)}\leq M \;\; \forall t \geq t_0 > \alpha$
\end{itemize}
\end{definition}

\begin{definition}
Diremos que la ecuación diferencial $(*)$ es \textbf{estable en el sentido de Lagrange} si todas sus soluciones son acotadas en el futuro.
\end{definition}

\subsection{Acotación de soluciones de EDO escalares}

Consideramos el conjunto abierto $D=(a,+\infty)\times\R$ donde $a\geq -\infty$ y una función continua $\funcion{f}{D}{\R}$.

\begin{prop}
Supongamos que la EDO $x'=f(t,x)$ verifica la propiedad de unicidad en el futuro y sean $p_1,p_2\in\R$ ($p_1<p_2$) tales que 
\[
f(t,p_1)=f(t,p_2)=0 \espacio \forall t\in(a,+\infty)
\]
Sea $\funcion{\varphi}{(\alpha,\beta)}{\R}$ una solución maximal de
\[
\left\{
\begin{array}{l}
x'=f(t,x)\\
x(t_0)=x_0\in[p_1,p_2]
\end{array}
\right.
\]
Entonces $\varphi(t)\in[p_1,p_2] \;\; \forall t\in(t_0,\omega)$. Además $\omega=+\infty$.
\end{prop}

\begin{example}
Ver página 6 tema 5.1. Importante, para que obtener la unicidad en el futuro, puedes ver que verifica la propiedad  de unicidad global viendo que es localmente lipschitziana. Lo último puedes hacerlo o viendo que es continua, o haciéndolo a mano (encontrar un entorno abierto para todo punto donde puedas acotar $f$).
\end{example}

\begin{prop}
Sea $\funcion{\varphi}{(\alpha,\beta)}{\R}$ una solución maximal de
\[
\left\{
\begin{array}{l}
x'=f(t,x)\\
x(t_0)=x_0\in[p_1,p_2]
\end{array}
\right.
\]
Si existen $p_1,p_2\in\R$ ($p_1<p_2$) tales que 
\[
f(t,p_1)>0 \espacio f(t,p_2)<0 \espacio \forall t\geq t_0
\]
Entonces $\varphi(t)\in(p_1,p_2) \;\; \forall t\in(t_0,\omega)$. Además $\omega=+\infty$.
\end{prop}
\begin{example}
Esta proposición hay que usarla cuando los ceros son tope tochos difíciles de encontrar. Así que es más fácil probar a mano valores hasta que salga uno positivo y uno negativo. Ver pag9. tema5.1.
\end{example}

\subsection{Acotación de soluciones de EDO vectoriales}

Consideramos el conjunto abierto $D=(a,+\infty)\times\R^d$ donde $a\geq -\infty$ y una función continua $\funcion{f}{D}{\R^d}$.
\begin{equation}
x'=f(t,x) \tag{*}
\end{equation}

Vamos a denotar por $\funcion{\dot{V}}{D}{\R}$ a la función definida por $\dot{V}(t,f(t,x))=<\nabla V(x),f(t,x))>$.
\begin{definition}
Diremos que $V$ es una \textbf{función guía} de la EDO $(*)$ si $\dot{V}\leq 0 \espacio \forall(t,x)\in D$ (o al menos para valores de $x$ suficientemente grandes)
\end{definition}

\begin{lemma}
Sea $\funcion{\varphi}{(\alpha,\beta)}{\R^d}$ una solución de $(*)$. Si existe $V\in\mathcal{C}^1(\R^d)$ tal que
\[
\dot{V}(t,x)=0 \espacio \forall (t,x)\in D
\]
y $r=V(\varphi(t_0))$ entonces $\varphi(t)\in S_r \espacio \forall t\in(\alpha,\omega)$.
\end{lemma}

\begin{remark}
Yo el lema 2 que aparece en la pag6.t5.2 no lo entiendo así que no me lo he mirado.
\end{remark}

\begin{definition}
Diremos que una función $\funcion{V}{\R}{\R}$ continua es \textbf{coerciva} en $\R$ si y sólo si:
\[
\limitemasinfinito{|x|}{V(x)}=+\infty
\]
\end{definition}

\begin{example}
Todos los polinomios de grado par con coeficiente principal positivo son coercivos.
\end{example}

\begin{definition}[funciones con variables separadas]
Diremos que una función $\funcion{V}{\R^2}{\R}$ continua, que se expresa como $V(x,y)=V_1(x)+V_2(y)$ es coerciva si $V_1$ y $V_2$ lo son.
\end{definition}
\begin{remark}
Lo anterior se puede generalizar para cualquier número de $V_i$.
\end{remark}
\begin{prop}
Puede pasar que $V$ no sea de variables separadas pero sí sea cuadrática, es decir:
\[
V(x,y)=ax^2+bxy+cy^2+dx+ey+r
\]
Entonces:
\[
V \text{ coerciva } \Longleftrightarrow a,c>0 \espacio b^2<4ac
\]
\end{prop}

\begin{theorem}[acotación mediante funciones guía]
Si existe $V\in\mathcal{C}^1(\R^d)$ coerciva tal que $\dot{V}(t,x)\leq 0 \forall (t,x)\in D$ entonces las soluciones maximales de la EDO $(*)$ están acotadas en el futuro.
\end{theorem}

\begin{theorem}
Dada $\funcion{f}{\R\times\R^d}{\R^d}$ continua, si existe $V\in\mathcal{C}^1(\R^2)$ coerciva tal que
\[
\dot{V}(t,x)=0 \espacio \forall (t,x)\in\R\times\R^d
\]
entonces las soluciones maximales de $(*)$ están acotadas en el futuro y en el pasado.
\end{theorem}
\begin{example}
Importante el ejemplo en la página 6 del tema 5.3.
\end{example}

\section{Ecuaciones diferenciales de segundo orden}

\begin{enumerate}[(a)]
\item Ecuaciones conservativas:
\[
mx''+g(x)=0
\]
\item Ecuaciones disipativas:
\[
mx''cx'+g(x)=0 \espacio c>0
\]
\end{enumerate}

Esta parte básicamente consiste en coger a $g(x)$, integrarla, y ver si el resultado es coercivo. Por ejemlo, $x''+x'+\sinh(x)=0$

En este caso, $g(x)=\sinh(x)$, continua, luego:
\[
G(x)=\integral{0}{x}{g(z)dx}=\integral{0}{x}{\sinh(z)dz}=\cosh(x)-1
\]
Como se da $\limite{x}{-\infty}{G(x)}=0$, $G$ no es coerciva, luego las soluciones de la ecuación no están acotadas en el futuro.
\end{document}