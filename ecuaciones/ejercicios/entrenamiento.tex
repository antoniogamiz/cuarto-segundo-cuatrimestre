% !TeX root = master.tex
\documentclass[12pt]{article}
 
\usepackage[margin=1in]{geometry} 
\usepackage{amsmath,amsthm,amssymb}
\usepackage[spanish]{babel}
\usepackage[utf8]{inputenc}
\usepackage{tikz-cd}
\usepackage{amsmath}
\usepackage[shortlabels]{enumitem}
\usepackage{mathtools}

% cosas entre comillas 
\usepackage{csquotes}

\usepackage{tikz}


\usepackage{xcolor}

\usepackage{personalcommands}

\newtheorem{theorem}{Teorema}[section]
\newtheorem{lemma}[theorem]{Lema}
\newtheorem{prop}[theorem]{Proposición}
\newtheorem{coro}[theorem]{Corolario}
\newtheorem{conj}[theorem]{Conjetura}
\newtheorem{ejercicio}{Ejercicio}
\newtheorem*{ejercicio*}{Ejercicio}
\theoremstyle{definition}
\newtheorem{definition}[theorem]{Definición}
\newtheorem{example}[theorem]{Ejemplo}
\theoremstyle{remark}
\newtheorem{remark}[theorem]{Nota}
\newtheorem{notacion}[theorem]{Notación}
\newcommand{\continuas}[1][]{C^{ #1 }[a,b]}
\newcommand{\continuasabierto}[1][]{C^{ #1 }(a,b)}
\newcommand{\soportecompacto}{\mathcal{D}(a,b)}
\newcommand{\xcero}{(a,b)}
\newcommand{\xcerocerrado}{[a,b]}
\newcommand{\fvariaciones}{F(x,y(x),y'(x))}

\usepackage{float}
\usepackage{pdfpages}

\newcommand{\ejercicioo}[1]{
\textbf{Ejercicio #1 }

\begin{figure}[H]
\center
\includegraphics[scale=0.5]{img/#1.png}
\end{figure}

}

\begin{document}

\ejercicioo{1}

Es esta, ya que se cumple $b^2<4ac$ y $a,c>0$.

\medskip

\ejercicioo{2}

\begin{enumerate}[(a)]
\item $f(x,y)=\arctan(xy)$ Esta acotada en $\R$ luego lo verifica.
\item $f(x)=\frac{3x^4}{x^3+x+1}$ Luego esta no lo verifica.
\item $f(x)=\ln(x^2+1)$ Globalmente lipschitziana por tener derivada acotada en $\R$, luego lo verifica.
\item $f(x,y)=\sqrt{1+(2x+3y)^2}$ 
\end{enumerate}

\medskip

\ejercicioo{3}

\[
\dot{V}(x,y)=<(V_1'(x),V_2'(y)),f(x,y))>=-V_1'(x)x^3-V_1'(x)y^3+V_2'(y)x^5-V_2'(y)y^7
\]
Haciendo los dos términos del medio igual a 0:
\[
V_1'(x)y^3=V_2'(y)x^5
\]
Luego $V_1(x)=\frac{x^6}{6}$ y $V_2(y)=\frac{y^6}{6}$. Vemos que si multiplicamos $V_1$ y $V_2$ por 6, la desigualdad anterior se sigue cumpliendo.

\medskip

\ejercicioo{4}

\medskip

\ejercicioo{5}

Este tipo de ecuación es conservativa con $g(x)=x+\cos(x)-1$, integrando:
\[
G(x)=\integral{0}{x}{g(z)dz}=\frac{x^2}{2}+\sin(x)-x
\]
Y la función guía para las ecuaciones conservativas era:
\[
V(x,y)=G(x)+\frac{y^2}{2}
\]
Luego es la opción $(D)$.

\medskip

\ejercicioo{6}

Esta es la A pero no se por qué.

\medskip

\ejercicioo{7}

La $B$ es disipativa, luego su función guía no va a cumplir $\dot{V}(x,y)=0$. Las otras, al ser conservativas, como mínimo su función $G$ asociada tiene que ser coerciva, cosa que no pasa ni en $A$, ni en $E$, ni en $F$, ni en $C$. 

Luego es la $D$.

\end{document}