% !TeX root = master.tex
\documentclass[12pt]{article}
 
\usepackage[margin=1in]{geometry} 
\usepackage{amsmath,amsthm,amssymb}
\usepackage[spanish]{babel}
\usepackage[utf8]{inputenc}
\usepackage{tikz-cd}
\usepackage{amsmath}
\usepackage[shortlabels]{enumitem}
\usepackage{mathtools}

% cosas entre comillas 
\usepackage{csquotes}

\usepackage{tikz}


\usepackage{xcolor}

\usepackage{config}

\newtheorem{theorem}{Teorema}[section]
\newtheorem{lemma}[theorem]{Lema}
\newtheorem{prop}[theorem]{Proposición}
\newtheorem{coro}[theorem]{Corolario}
\newtheorem{conj}[theorem]{Conjetura}
\newtheorem{ejercicio}{Ejercicio}
\newtheorem*{ejercicio*}{Ejercicio}
\theoremstyle{definition}
\newtheorem{definition}[theorem]{Definición}
\newtheorem{example}[theorem]{Ejemplo}
\theoremstyle{remark}
\newtheorem{remark}[theorem]{Nota}
\newtheorem{notacion}[theorem]{Notación}
\newcommand{\continuas}[1][]{C^{ #1 }[a,b]}
\newcommand{\continuasabierto}[1][]{C^{ #1 }(a,b)}
\newcommand{\soportecompacto}{\mathcal{D}(a,b)}
\newcommand{\xcero}{(a,b)}
\newcommand{\xcerocerrado}{[a,b]}
\newcommand{\fvariaciones}{F(x,y(x),y'(x))}


\usepackage{pdfpages}
 

\begin{document}

\begin{ejercicio}
Sea $I\subset\R$ un intervalo no vacío y sea $\funcion{f}{I}{\R}$. Denotamos
\[
I^*=\{t\in I:\exists f'(t)\}
\]
Supongamos que existe una sucesión $t_n\in I^*$ tal que
\[
|f'(t_n)|\longrightarrow\infty
\]
Demuestra que la función $f$ no es globalmente lipschitziana en $I$.
\end{ejercicio}
Como la derivada de $f$ existe en $I^*$, $f$ también es continua en $I^*$. Usando la definición de derivada con la sucesión $t_n$:
\[
|f'(t_n)|=\limite{h}{0}{\frac{|f(t_n+h)-f(t_n)|}{h}}=\infty
\]
Luego $f$ no es globalmente lipschitziana porque no lo es localmente (como nos dice la caracterización).
\begin{ejercicio}
Sea $I\subset\R$ un intervalo no vacío y sea $\funcion{f}{I}{\R}$. Diremos que $f$ es una función de clase uno a trozos y escribiremos $f\in \mathcal{C}^1_T(I)$, si existe un recubrimiento finito del intervalo $I$:
\[
I=\bigcup_{j=1}^nI_j
\]
donde los subintervalos $I_j$ son cerrados relativos a $I$ y se cumple: $f_{|I_j}\in\mathcal{C}^1(I_j)$. Dada $f\in\mathcal{C}^1_T(I)$, si denotamos $I^*=\{t\in I:\exists f'(t)\}$, demuestra que las siguientes afirmaciones son equivalentes:
\begin{enumerate}[(a)]
\item La función $f$ es globalmente lipschitziana en $I$.
\item Existe $L\geq 0$ tal que $|f'(t)|\leq L \;\; \forall t\in I^*$
\end{enumerate}
\end{ejercicio}

$(a)\Rightarrow(b)$ Si $f$ es globalmente lipschitz se cumple que existe una constante $L$ tal que $|f(x)-f(y)|\leq L|x-y|$. Como cada intervalo $I_j$ es cerrado puedo encontrar una sucesión de puntos $x_n\longrightarrow y \;\; \forall y\in I_j\cap I^*$, luego:
\[
\limite{x_n}{y}{\frac{|f(x_n)-f(y)|}{|x_n-y|}}=|f'(y)|\leq L
\]
Por lo tango, para todo $t$ en $I^*$ se cumple que $|f'(t)|\leq L$.

$(b)\Rightarrow(a)$ Tomo $x_n$ sucesión de forma que $x_0=s$ y $x_n$ sea el mayor valor dentro de cada $I_j$ de manera que puedo ir enlazando entre los distintos $I_j$ y $x_i\longrightarrow t$ de manera que 
\[
|f(t)-f(s)|\leq\integral{s}{t}{|f'(\tau)|d\tau}\leq\displaystyle\sum\integral{x_i}{x_{i+1}}{|f'(\tau)|d\tau}\leq L\displaystyle\sum|x_{i+1}-x_i|=L|t-s|
\]
\begin{ejercicio}
Sea $D\subset\R^d$ un conjunto convexo y sea $\funcion{f}{D}{\R^d}$. Suponemos que existe un recubrimiento finito del conjunto $D$ formada por conjuntos cerrados:
\[
D=\bigcup_{i=1}^nC_i \espacio \bar{C_i}=C_i \;\;\forall i = 1,\cdots,n
\]
tales que para cada $i=1,\cdots,n$ la restricción $f_{|C_i}$, es lipschitziana (con constante de lipschitz $L_i$). Queremos demostrar que $f$ es lipschitziana en $D$. Para ello:
\begin{enumerate}[(a)]
\item Dados $x,y\in D$ demuestra que existe una partición del segmento $[x,y]$ tal que los nodos consecutivos están en un mismo conjunto cerrado del recubrimiento:
\[
\exists\{z_j\}_{j=1}^m\subset[x,y] \;\; z_1=x, \; z_m=y \espacio z_j,z_{j+1}\in C_i \;\;\;\forall j=1,\cdots,m-1
\]
\item Prueba que $\norm{f(x)-f(y)}\leq \displaystyle\sum_{j=1}^{m-1}\norm{f(z_j)-f(z_{j+1}}$
\item Deduce que $f$ es lipschitziana y determina su constante de Lipschitz.
\end{enumerate}
\end{ejercicio}
\begin{ejercicio}
Sea $D\subset\R^d$, $\funcion{f}{D}{\R^K}$, $\bar{D}\subset\R^k$, $\funcion{g}{\bar{D}}{\R^m}$ tales que las funciones $f$ y $g$ son globalmente lipschitzianas y además $f(D)\subset\bar{D}$. Demuestra que la composición $\funcion{g\circ f}{D}{\R^m}$ es globalmente lipschitziana.
\end{ejercicio}

Sean $L_1$ y $L_2$ las constantes de lipschitz de $f$ y $g$ respectivamente. Tenemos:
\[
\norm{f(g(x_1))-f(g(x_2))}\leq L_1\norm{g(x_1)-g(x_2)}\leq L_1L_2\norm{x_1-x_2} \espacio x_1,x_2\in D
\]
Luego $f\circ g$ es lipschitziana.
\begin{ejercicio}
Sea $D\subset\R\times\R^d$ abierto y sean $\funcion{f}{D}{\R^d}$ y $\funcion{g}{D}{\R^k}$ continuas. Además $f(t,x)$ y $g(t,x)$ son localmente lipschitzianas (respecto de la variable $x\in\R^d$). Decide sobre la validez de las siguientes afirmaciones:
\begin{enumerate}
\item Si $k=d$ entonces $f+g$ es localmente lipschitziana (respecto de la variable $x$).
\item Si $k=1$ entonces $f\cdot g$ es localmente lipschitziana (respecto de la variable $x$).
\end{enumerate}
\end{ejercicio}
Existe un entorno $U=U^\circ\subset D$ donde $f$ y $g$ son localmente lipschitzianas.

$(a)$
\[
\forall(t,x),(t,y)\in U \Rightarrow \norm{f(t,x)-f(t,y)+g(t,x)-g(t,y)}\leq
\]
\[
\leq L_f\norm{x-y}+L_g\norm{x-y}\leq (L_f+L_g)\norm{x-y} \Rightarrow f+g \text{ lipscihtz }
\]

$(b)$ De forma análoga al apartado anterior obtenemos:
\[
\norm{g(t,x)f(t,x)-g(t,y)f(t,y)}\leq (BL_f+AL_g)\norm{x-y}
\]
Donde $A$ y $B$ son las cotas de $f$ y $g$ respectivamente.

\begin{ejercicio}
Sean $I$ y $J$ dos intervalos de $\R$ y sean $\funcion{f}{I}{\R}$, $\funcion{g}{J}{\R}$ dos funciones localmente lipschitzianas. Estudia si podemos asegurar que las siguientes funciones son localmente lipschitzianas:
\begin{enumerate}[(a)]
\item $\funcion{F}{I\times J}{\R}$, $F(x,y)=f(x)+g(y)$

Para ver que $F$ es localmente lipschitziana, tenemos que encontrar un entorno $U_{xy}$  de $(x,y)\in I\times J$ tal que existe una constante $L_{xy}$ verificando:
\[
|F(x,y)-F(x',y')|\leq L_{xy}\norm{(x,y)-(x',y')} \espacio \forall (x',y')\in U_{xy}
\]

Si $(x,y)\in I\times J$, como $f$ y $g$ son localmente lipschitzianas, podemos encontrar dos entornos $U_x \subset I$, $U_y\subset J$ de $x$ e $y$ respectivamente, tal que se verifica la condición de lipschitz. Podemos tomar $U_{xy}=U_x\cap U_y$, teniendo:
\[
|f(x)-f(y)|\leq L_{x}|x-y|, \espacio |f(x)-g(y)|\leq L_{y}|x-y| \espacio \forall x,y\in U_{xy}
\]
Tenemos que ver que se verifica para $F$:
\[
|F(x_1,y_1)-F(x_2,y_2)|=|f(x_1)+g(y_1)-f(x_2)-g(y_2)|\leq
\]
\[
\leq |f(x_1)-f(x_2)|+|g(y_1)-g(y_2)|\leq L_x|x_1-x_2|+L_y|y_1-y_2|\leq
\]
\[
\leq L_{xy}(|x_1-x_2|+|y_1-y_2|)\leq L_{xy}\norm{(x_1,y_1)-(x_2,y_2)}
\]
donde $L_{xy}=\max{L_x,L_y}$.

\item $\funcion{G}{I\times J}{\R}$, $G(x,y)=f(x)\cdot g(y)$

Sean $[a,b]\subset I$, $[c,d]\subset J$ dos intervalos compactos. Como $f$ y $g$ son lipschitzianas, son continuas, por lo tanto, por el teorema de Weierstrass están acotadas en esos dos intervalos compactos por una constante $C$, que podemos suponerla común para ambas funciones.

Sean $L_f$ y $L_g$ las constantes de lipschitz de $f$ y $g$ en esos dos subintervalos. Entonces:
\[
|f(x_1)g(y_1)-f(x_2)g(y_2)|\leq |f(x_1)g(y_1)-f(x_2)g(y_1)|+|f(x_2)g(y_1)-f(x_2)g(y_2)|\leq
\]
\[
\leq C|f(x_1)-f(x_2)|+C|g(y_1)-g(y_2)|\leq C\left(L_f|x_1-x_2|+L_g|y_1-y_2|\right)\leq
\]
\[
\leq K\norm{(x_1,y_1)-(x_2,y_2)}, \espacio \forall (x_1,y_1),(x_2,y_2)\in [a,b]\times[c,d] 
\]
donde $K=2C\max{L_f,L_g}$.

\end{enumerate}
\end{ejercicio}
\end{document}